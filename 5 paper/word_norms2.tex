\documentclass[english,man]{apa6}

\usepackage{amssymb,amsmath}
\usepackage{ifxetex,ifluatex}
\usepackage{fixltx2e} % provides \textsubscript
\ifnum 0\ifxetex 1\fi\ifluatex 1\fi=0 % if pdftex
  \usepackage[T1]{fontenc}
  \usepackage[utf8]{inputenc}
\else % if luatex or xelatex
  \ifxetex
    \usepackage{mathspec}
    \usepackage{xltxtra,xunicode}
  \else
    \usepackage{fontspec}
  \fi
  \defaultfontfeatures{Mapping=tex-text,Scale=MatchLowercase}
  \newcommand{\euro}{€}
\fi
% use upquote if available, for straight quotes in verbatim environments
\IfFileExists{upquote.sty}{\usepackage{upquote}}{}
% use microtype if available
\IfFileExists{microtype.sty}{\usepackage{microtype}}{}

% Table formatting
\usepackage{longtable, booktabs}
\usepackage{lscape}
% \usepackage[counterclockwise]{rotating}   % Landscape page setup for large tables
\usepackage{multirow}		% Table styling
\usepackage{tabularx}		% Control Column width
\usepackage[flushleft]{threeparttable}	% Allows for three part tables with a specified notes section
\usepackage{threeparttablex}            % Lets threeparttable work with longtable

% Create new environments so endfloat can handle them
% \newenvironment{ltable}
%   {\begin{landscape}\begin{center}\begin{threeparttable}}
%   {\end{threeparttable}\end{center}\end{landscape}}

\newenvironment{lltable}
  {\begin{landscape}\begin{center}\begin{ThreePartTable}}
  {\end{ThreePartTable}\end{center}\end{landscape}}

  \usepackage{ifthen} % Only add declarations when endfloat package is loaded
  \ifthenelse{\equal{\string man}{\string man}}{%
   \DeclareDelayedFloatFlavor{ThreePartTable}{table} % Make endfloat play with longtable
   % \DeclareDelayedFloatFlavor{ltable}{table} % Make endfloat play with lscape
   \DeclareDelayedFloatFlavor{lltable}{table} % Make endfloat play with lscape & longtable
  }{}%



% The following enables adjusting longtable caption width to table width
% Solution found at http://golatex.de/longtable-mit-caption-so-breit-wie-die-tabelle-t15767.html
\makeatletter
\newcommand\LastLTentrywidth{1em}
\newlength\longtablewidth
\setlength{\longtablewidth}{1in}
\newcommand\getlongtablewidth{%
 \begingroup
  \ifcsname LT@\roman{LT@tables}\endcsname
  \global\longtablewidth=0pt
  \renewcommand\LT@entry[2]{\global\advance\longtablewidth by ##2\relax\gdef\LastLTentrywidth{##2}}%
  \@nameuse{LT@\roman{LT@tables}}%
  \fi
\endgroup}


\ifxetex
  \usepackage[setpagesize=false, % page size defined by xetex
              unicode=false, % unicode breaks when used with xetex
              xetex]{hyperref}
\else
  \usepackage[unicode=true]{hyperref}
\fi
\hypersetup{breaklinks=true,
            pdfauthor={},
            pdftitle={English Semantic Feature Production Norms: An Extended Database of 4,437 Concepts},
            colorlinks=true,
            citecolor=blue,
            urlcolor=blue,
            linkcolor=black,
            pdfborder={0 0 0}}
\urlstyle{same}  % don't use monospace font for urls

\setlength{\parindent}{0pt}
%\setlength{\parskip}{0pt plus 0pt minus 0pt}

\setlength{\emergencystretch}{3em}  % prevent overfull lines

\ifxetex
  \usepackage{polyglossia}
  \setmainlanguage{}
\else
  \usepackage[english]{babel}
\fi

% Manuscript styling
\captionsetup{font=singlespacing,justification=justified}
\usepackage{csquotes}
\usepackage{upgreek}

 % Line numbering
  \usepackage{lineno}
  \linenumbers


\usepackage{tikz} % Variable definition to generate author note

% fix for \tightlist problem in pandoc 1.14
\providecommand{\tightlist}{%
  \setlength{\itemsep}{0pt}\setlength{\parskip}{0pt}}

% Essential manuscript parts
  \title{English Semantic Feature Production Norms: An Extended Database of 4,437
Concepts}

  \shorttitle{Semantic Norms}


  \author{Erin M. Buchanan\textsuperscript{1}, K. D. Valentine\textsuperscript{2}, \& Nicholas P. Maxwell\textsuperscript{1}}

  % \def\affdep{{"", "", ""}}%
  % \def\affcity{{"", "", ""}}%

  \affiliation{
    \vspace{0.5cm}
          \textsuperscript{1} Missouri State University\\
          \textsuperscript{2} University of Missouri  }

  \authornote{
    Erin M. Buchanan is an Associate Professor of Quantitative Psychology at
    Missouri State University. K. D. Valentine is a Ph.D.~candidate at the
    University of Missouri. Nicholas P. Maxwell is a Masters' candidate at
    Missouri State University.
    
    We would like to thank Keith Hutchison and David Balota for their
    contributions to this project, including the funds to secure Mechanical
    Turk participants.
    
    Correspondence concerning this article should be addressed to Erin M.
    Buchanan, 901 S. National Ave, Springfield, MO 65897. E-mail:
    \href{mailto:erinbuchanan@missouristate.edu}{\nolinkurl{erinbuchanan@missouristate.edu}}
  }


  \abstract{The largest limiting factor in understanding memory and language
networks is often the availability of normed stimuli to use and explore
in experimental studies. In this study, we expand on three previous
semantic feature overlap norms to over 4,000 cue stimuli ranging from
nouns, verbs, adjectives, and other parts of speech. Participants in the
norming study were asked to provide feature components of each cue
stimuli, which were combined with the previous research using semantic
feature production procedures. In addition to expanding previous
research, this project explores different semantic overlap measurements
by coding each word feature listed by root and affixes to determine
different strengths of feature overlap. All information is provided in a
searchable database for easy access and utilization for future
researchers when designing experiments. The final database of cue-target
pairs was paired with the Semantic Priming Project to examine the
predictive ability of semanticity on semantic priming in tandem with
other psycholinguistic variables, such as association, thematics, and
frequency. Target concept frequency was the largest predictor of
semantic priming, follow by thematics and association. Root word cosine
was predictive of semantic priming, even after adjusting for the
previously mentioned psycholinguistic variables.}
  \keywords{semantics, word norms, database, psycholinguistics \\

    
  }





\usepackage{amsthm}
\newtheorem{theorem}{Theorem}
\newtheorem{lemma}{Lemma}
\theoremstyle{definition}
\newtheorem{definition}{Definition}
\newtheorem{corollary}{Corollary}
\newtheorem{proposition}{Proposition}
\theoremstyle{definition}
\newtheorem{example}{Example}
\theoremstyle{definition}
\newtheorem{exercise}{Exercise}
\theoremstyle{remark}
\newtheorem*{remark}{Remark}
\newtheorem*{solution}{Solution}
\begin{document}

\maketitle

\setcounter{secnumdepth}{0}



Semantic representations are the focus of a large area of research and
are a thing.

What are semantic feature production norms

Why are they important!

Previous work

\section{Method}\label{method}

\subsection{Participants}\label{participants}

Participants in the newly collected stimuli were gathered from Amazon's
Mechanical Turk, which is a large, diverse participant pool wherein
users can complete surveys for small sums of money (Buhrmester, Kwang,
\& Gosling, 2011). Answers can be screened for errors, and incorrect or
incomplete surveys can be rejected or discarded without payment. Each
participant was paid five cents for a survey, and they could complete
multiple Human Intelligence Tasks or HITS. Each HIT included five
concepts, and HITS would remain active until \emph{n} = 30 valid survey
answers were collected. HITS were usually rejected if they included
copied definitions from Wikipedia, \enquote{I don't know}, or writing a
paragraph about the concept. These answers were discarded, as described
below. Table \ref{tab:part-table} includes the sample sizes from the new
study (Mechanical Turk 2), as well as the sample sizes from the previous
study, as described in Buchanan, Holmes, Teasley, and Hutchison (2013).

\begin{table}[tbp]
\begin{center}
\begin{threeparttable}
\caption{\label{tab:part-table}}
\begin{tabular}{lccc}
\toprule
Institution & Total Participants & Concepts & Mean $N$\\
\midrule
University of Mississippi & 749 & 658 & 67.8\\
Missouri State University & 1420 & 720 & 71.4\\
Montana State University & 127 & 120 & 63.5\\
Mechanical Turk 1 & 571 & 310 & 60\\
Mechanical Turk 2 & 198 & 1914 & 30\\
\bottomrule
\addlinespace
\end{tabular}
\begin{tablenotes}[para]
\textit{Note.} 
\end{tablenotes}
\end{threeparttable}
\end{center}
\end{table}

\subsection{Materials}\label{materials}

The purpose of this second norming set was to expand the Buchanan et al.
(2013) norms to include all concepts from the Semantic Priming Project
({\textbf{???}}). Therefore, these concepts were the target of the
project. The original concept set was selected primarily from the
Nelson, McEvoy, and Schreiber (2004) database, with small overlaps in
the McRae, Cree, Seidenberg, and McNorgan (2005) and Vinson and
Vigliocco (2008) database sets for convergent validity. In the Semantic
Priming Project, cue-target pairs were shown to participants to examine
naming and lexical decision time priming across related and unrelated
pairs. The related pairs included first associate (most common response
to a cue) and other associates (second or greater common responses to
cues) as their target words. The original publication of concepts
included the cue words from the Semantic Priming Project, while this
project expanded to include missed cue words and all target words. The
addition of these concepts allowed for complete overlap between the
Semantic Priming Project and the feature production norms.

Concepts were labeled by part of speech using the English Lexicon
Project (Balota et al., 2007), the free association norms, and Google's
define search when necessary. When labelling these words, we used the
most common part of speech to categorize concepts. This choice was
predominately for simplicity of categorization, however, the
participants were shown concepts without the suggestion of which sense
to use for the word. Therefore, multiple senses are embedded into the
feature production norms, while the database is labeled with single
parts of speech. The other parts of speech can be found in the English
Lexicon Project or multiple other databases. This dataset was combined
with ({\textbf{???}}) and Vinson and Vigliocco (2008) feature production
norms, which was a combined total of 4437 concepts. 70.4\% of concepts
were nouns, 14.9\% adjectives, 12.4\% verbs, and 2.3\% were other forms
of speech, such as adverbs and conjunctions.

\subsection{Procedure}\label{procedure}

Each HIT was kept to five concepts, and usual survey response times were
five to seven minutes. Each HIT was open until thirty participants had
successfully completed the HIT and were paid the five cents for the HIT.
The survey instructions were copied from McRae et al. (2005)'s Appendix
B, which were also used in the previous publication of these norms.
Because the McRae et al. (2005) data was collected on paper, we modified
these instructions slightly. The original lines to write in responses
were changed to an online textbox response window. The detailed
instructions additionally no longer contained information about how a
participant should only consider the noun of the target concept, as the
words in our study included multiple forms of speech and senses.
Participants were encouraged to list the properties or features of each
concept in the following areas: physical (looks, sounds, and feels),
functional (uses), and categorical (belongings). The same examples used
previously (\emph{duck, cucumber, stove}) were included to aid in task
understanding and completion. Participants signed up for the HITS
through Amazon's Mechanical Turk website and completed the study within
the Mechanical Turk framework. Approved HITs were compensated through
the Mechanical Turk system. All answers were then combined into a larger
dataset.

\subsection{Data Processing}\label{data-processing}

The entire dataset, at each processing stage described here, can be
found at: \url{https://osf.io/cjyzw/}. On our OSF page, we have included
a detailed processing guide on how concepts were (re)examined for this
publication. This paper was written with \emph{R} markdown and
\emph{papaja} (Aust \& Barth, 2018). The markdown document allows an
interested reader to view the scripts that created the article in line
with the written text. However, the processing of the text documents was
performed on the raw files, and therefore, we have included the
processing guide for transparency of each stage.

First, each concept was separated into an individual text file that is
included as the \enquote{raw} data online. Each of these files was then
spell checked and corrected when the participant answer was obviously a
typo. As noted earlier, participants often tried to cut and paste
Wikipedia or other online dictionary sources into the their answers to
complete surveys quickly with minimal effort. These entries were easily
found by the formatting of the webpage that was included in their
answer. These answers were then discarded from the concept individual
text files. Next, each concept was processed for feature frequency. In
this stage, the raw frequency counts of each cue-feature combination
were calculated and put together into one large file. Cue-cue
combinations were discarded, as participants might write \enquote{a
zebra is a horse} when asked to define \emph{zebra}. English stop words
such as \emph{the, an, of} were then discarded, as well as terms that
were often used as part of a definition (\emph{like, means, describes}).

To create the final root cosine values, we then created a
\enquote{translated} column for each feature listed. This column
indicated the root word for each feature, along with the affixes that
were used in the original feature. For example, the original feature
\emph{cats} would be translated to \emph{cat} and \emph{s}, wherein
\emph{cat} would be the translated feature and the \emph{s} would be the
affix code. Multiple affix codes were often needed for features, as
\emph{beautifully} would have been translated to \emph{beauty},
\emph{ful}, and \emph{ly}. Often, the noun version of the feature would
be used for the translation or the most common part of speech for each
feature would be recorded. The sample size for the cue was added to this
dataset, as the sample sizes varied across experiment time, as shown in
Table \ref{tab:part-table}. Therefore, instead of using raw feature
frequency, we normalized each count into a percent of participants that
included that feature with each cue.

At this stage, the data was reduced to cue-feature combinations that
were listed by at least 16\% of participants (matching McRae et al.
(2005)'s procedure) or were in the top five features listed for that
cue. This calculation was performed on the translated normalized feature
percent. For example, \emph{beauty} may have been listed as
\emph{beauty, beautiful, beautifully, beautifulness}, and this feature
would have been listed three times in the dataset for the original cue.
The \emph{frequency\_feature} column indicates the frequency of the
original, unedited feature, while the \emph{frequency\_translated}
includes all combinations of \emph{beauty} into one overall feature.
Because non-nouns can be more difficult to create a feature list for, we
included the top five descriptors in addition to the 16\% listed
criteria, to ensure that each concept included at least five features.
Table \ref{tab:feature-table} indicates the average number of
cue-feature pairs found for each data collection site/time point and
part of speech for the cue word.

The parts of speech for the cue, original feature, and translated
feature were merged with this file as described above. Table
\ref{tab:percent-table} depicts the pattern of feature responses for
cue-feature part of speech combinations. This table includes the percent
of features listed for each cue-feature part of speech combination
(i.e., what is the percent of time that both the cue and feature were
both adjectives) for the original feature (raw) and translated feature
(root). Next, the normalized frequency percent average was calculated
along with their standard deviations. These columns indicate the
frequency percent that a cue-feature part of speech combination was
listed across participants (i.e., what is the average percent of
participants that listed an adjective feature for an adjective cue).
These two types of calculation describe the likelihood of seeing part of
speech combinations across the concepts, along with the likelihood of
those cue-feature part of speech combinations across participants.

\begin{table}[tbp]
\begin{center}
\begin{threeparttable}
\caption{\label{tab:feature-table}}
\begin{tabular}{lccccc}
\toprule
Institution & Adjective & Noun & Verb & Other & Total\\
\midrule
University of Mississippi & 5.57 (1.53) & 7.35 (4.05) & 5.33 (.87) & 6.01 (2.11) & 6.71 (3.44)\\
Missouri State University & 5.74 (1.56) & 6.85 (2.82) & 6.67 (2.08) & 7.45 (5.35) & 6.65 (2.92)\\
Montana State University & 5.81 (1.74) & 7.25 (3.35) & 5.59 (1.13) & 5.76 (1.74) & 6.69 (2.93)\\
Mechanical Turk 1 & 6.27 (2.28) & 7.74 (4.34) & 5.77 (1.17) & 5.57 (1.40) & 7.14 (3.79)\\
Mechanical Turk 2 & 5.76 (1.36) & 6.62 (1.85) & 5.92 (1.38) & 5.78 (1.17) & 6.38 (1.75)\\
Total & 5.78 (1.61) & 6.94 (2.88) & 5.67 (1.18) & 5.84 (1.71) & 6.57 (2.60)\\
\bottomrule
\addlinespace
\end{tabular}
\begin{tablenotes}[para]
\textit{Note.} 
\end{tablenotes}
\end{threeparttable}
\end{center}
\end{table}

\begin{table}[tbp]
\begin{center}
\begin{threeparttable}
\caption{\label{tab:percent-table}}
\begin{tabular}{llcccc}
\toprule
Cue Type & Feature Type & \% Raw & \% Root & $M$ Freq. Raw & $M$ Freq. Root\\
\midrule
Adjective & Adjective & 38.09 & 29.74 & 17.84 (16.47) & 30.02 (18.83)\\
 & Noun & 40.02 & 46.74 & 13.14 (14.96) & 29.71 (19.94)\\
 & Verb & 17.69 & 20.72 & 8.51 (9.78) & 26.88 (17.27)\\
 & Other & 4.2 & 2.8 & 15.17 (15.64) & 28.04 (15.54)\\
Noun & Adjective & 16.56 & 12.07 & 15.55 (15.17) & 31.20 (18.17)\\
 & Noun & 60.85 & 62.67 & 17.21 (17.01) & 33.26 (20.05)\\
 & Verb & 20.8 & 23.68 & 8.88 (9.73) & 31.01 (17.87)\\
 & Other & 1.79 & 1.58 & 17.06 (15.29) & 28.87 (17.14)\\
Verb & Adjective & 15.16 & 12.27 & 13.95 (13.98) & 30.03 (18.28)\\
 & Noun & 42.92 & 44.35 & 14.59 (14.92) & 29.59 (18.90)\\
 & Verb & 36.92 & 39.72 & 12.75 (14.85) & 30.43 (19.54)\\
 & Other & 5 & 3.66 & 19.16 (15.95) & 25.59 (19.54)\\
Other & Adjective & 20.8 & 20.32 & 16.61 (17.37) & 31.66 (19.51)\\
 & Noun & 42.74 & 39.03 & 16.77 (19.41) & 37.28 (25.94)\\
 & Verb & 19.66 & 23.93 & 7.18 (7.57) & 26.14 (19.38)\\
 & Other & 16.81 & 16.71 & 22.72 (16.69) & 30.70 (18.48)\\
Total & Adjective & 19.74 & 14.93 & 16.12 (15.57) & 30.75 (18.37)\\
 & Noun & 55.41 & 57.81 & 16.55 (16.74) & 32.58 (20.09)\\
 & Verb & 22.02 & 24.95 & 9.50 (10.91) & 30.29 (18.24)\\
 & Other & 2.82 & 2.31 & 17.76 (15.83) & 28.45 (16.83)\\
\bottomrule
\addlinespace
\end{tabular}
\begin{tablenotes}[para]
\textit{Note.} 
\end{tablenotes}
\end{threeparttable}
\end{center}
\end{table}

The top cue-feature combinations from Buchanan et al. (2013) and this
new data collection were then combined with the cue-feature combinations
from McRae et al. (2005) and Vinson and Vigliocco (2008). We did not
reduce their cue-feature combinations, but instead included them with
the cue-feature listed in their supplemental files with the feature in
the raw feature column. If features could be translated into root words
with affixes, the same procedure as described above was applied. The
final file then included the original dataset, cue, feature, translated
feature, frequency of the original feature, frequency of the translated
feature, sample size, normalized frequencies for the original and
translated feature. This file includes 69284 cue-feature combinations,
with 48925 from our dataset, and 24449 of which are unique
cue-translated feature combinations. Statisics in Tables
\ref{tab:feature-table} and \ref{tab:percent-table} only include
information from the reprocessed Buchanan et al. (2013) norms and the
new cues collected for this project. The final data processing step was
to code affixes found on the original features. A complete affix list
translation file can be found online in our OSF files. Table
\ref{tab:affix-table} displays the list of affix tags, common examples
for each type of affix, and the percent of affixes that fell into each
cateory. The percent values are calculated on the overall affix list, as
feature words could have up to three different affixes. Generally,
affixes were tagged in a one-to-one match, however, special care was
taken with numbers and verb tenses. Features like cat\emph{s} would be
coded as a number affix, while features like walk\emph{s} would be coded
as a third person verb. In the final words file found online, we
additionally added forward strength (FSG) and backward strength (BSG)
for investigation into association overlap. The last few columns
indicate the word list a concept was originally normed in to allow for
matching to the original raw files on the OSF page, along with the code
for each school and time point of collection.

\begin{table}[tbp]
\begin{center}
\begin{threeparttable}
\caption{\label{tab:affix-table}}
\begin{tabular}{llc}
\toprule
Affix Tag & Example & Percent\\
\midrule
Actions/Processes & ion, ment, ble, ate, ize & 8.21\\
Characteristic & y, ous, nt, ful, ive, wise & 22.72\\
Location & under, sub, mid, inter & 0.44\\
Magnitude & er, est, over, super, extra & 1.31\\
Not & less, dis, un, non, in , im, ab & 2.76\\
Number & s, uni, bi, tri, semi & 28.31\\
Opposites/Wrong & mis, anti, de & 0.13\\
Past Tense & ed & 8.03\\
Person/Object & er, or, men, person, ess, ist & 7.23\\
Present Participle & ing & 14.03\\
Slang & bros, bike, bbq, diff, h2o & 0.12\\
Third Person & s & 6.16\\
Time & fore, pre, post, re & 0.54\\
\bottomrule
\addlinespace
\end{tabular}
\begin{tablenotes}[para]
\textit{Note.} 
\end{tablenotes}
\end{threeparttable}
\end{center}
\end{table}

This procedure is a slight departure from our previous work, as we
previously argued to keep some morphologically similar features separate
if they denoted different concepts. For example, \emph{act} and
\emph{actor} were separated because each feature explained a separate
component of the cue word (i.e., noun and gender). The original
processing in Buchanan et al. (2013) combined features that overlapped
in cue sets by 80\%. In this reprocessing and update, we translated all
words to a root form, and coded these translations, thus, allowing for
the exploration of the affect of affixes on semantic feature overlap.
Both forms of the feature are provided for flexibility in calculating
overlap by using the original feature (raw), the translated feature
(root), and the affix overlap by code (affix). Cosine values were
calculated for each of these feature sets by using the following
formula:

\[
\frac{\sum_{i=1}^{n} A_i \times B_i} {\sqrt{\sum_{i=1}^{n} A_i^2} \times \sqrt{\sum_{i=1}^{n} B_i^2}}
\]

This formula is similar to a dot-product correlation, where \(A_i\) and
\(B_i\) indicate the overlapping feature normalized frequency between
cue A and cue B. The \emph{i} subscript denotes the current cue, and
when features match, the frequencies are multiplied together and summed
across all matches (\(\Sigma\)). For the denominator, the feature
normalized frequency is first squared and summed from \emph{i} to
\emph{n} features for cue A and B. The square root of these summation
values is then multiplied together. In essence, the numerator calculates
the overlap of feature frequency for matching features, while the
denominator accounts for the entire feature frequency set for each cue.
Cosine values range from 0 (no overlapping features) to 1 (complete
overlapping features). With nearly five thousand cue words, just under
twenty-five million cue-cue cosine combinations can be calculated. In
the datasets presented online, we only included cue-cue combinations
with a feature overlap of at least two features, in order to reduce the
large quantity of zero and very low cosine values. This procedure
additionally allowed for online presentation of the data, as millions of
lines was not feasible for our server. The complete feature list, along
with our code to calculate cosine, can be used to obtain values not
presented in our data if necessary.

\subsection{Website}\label{website}

In addition to our OSF page, we present a revamped website for this data
at \url{http://www.wordnorms.com/}. The single words page includes
information about each of the cue words including cue set size,
concreteness, word frequency from multiple sources, length, full part of
speech, orthographic/phonographic neighborhood, and number of phonemes,
syllables, and morphemes. These values were taken from Nelson et al.
(2004), Balota et al. (2007), and ({\textbf{???}}). A definition of each
of these variables is provided along with the minimum, maximum, mean,
and standard deviation of numeric values. The table is programmed using
Shiny apps (Chang, Cheng, Allaire, Xie, \& McPherson, 2017). Shiny is an
\emph{R} package that allows the creation of dynamic graphical user
interfaces for interactive web applications. The advantage to using
Shiny applications is data manipulation and visualization with the
additional bonus of up to date statistics for provided data (i.e., as
typos are fixed or data is updated, the web app will display the most
recent caculations). In addition to the variable table, users can search
and save filtered output using our Shiny search app. With this app, you
can filter for specific variable ranges and save the output in a csv or
Excel file. The complete data is also provided for download.

On the word pairs page, all information about word-pair statistics can
be found. A second variable table is provided with semantic and
associative statistics. This dataset includes the cue and target words
from this project (cue-cue combinations), the root, raw, and affix
cosines described above, as well as the original Buchanan et al. (2013)
cosines. Additional semantic information includes Latent Semantic
Analysis (LSA; {\textbf{???}}) and JCN ({\textbf{???}}) values provided
in the Maki, McKinley, and Thompson (2004) norms, along with FSG and BSG
from the Nelson et al. (2004) norms for association. The descriptions,
minimum, maximum, mean, and standard deviations of these values are
provided in the app. Again, the search app includes all of these stimuli
for cue-cue combinations with two or more features in common, where you
can filter this data for experimental stimuli creation. The separation
of single and word-pair data (as well as cosine calculation reduction to
cues with two features in common) was practical, as the applications run
slowly as a factor of the number of rows and columns. On each page, we
link the data, applications, and source code so that others may use and
manipulate our work depending on their data creation or visualization
goals.

\section{Results}\label{results}

An examination of the results of the cue-feature lists indicated that
the new data collection is similar to the previous semantic feature
production norms. As shown in Table \ref{tab:feature-table}, the new
Mechanical Turk data showed roughly the same number of listed features
for each cue concepts, usually between five to seven features. Table
\ref{tab:percent-table} portrayed that adjective cues generally included
other adjectives or nouns as features, while noun cues were
predominately described by other nouns. Verb cues included a large
feature list of nouns, but then was equally split between adjectives,
other verbs, and other categories. Lastly, the other cue types generally
elicited nouns and verbs. Normalized percent frequencies were generally
between seven and twenty percent of the participant sampling listing
features when examining the raw words. These words included multiple
forms, as the percent increased to around thirty percent when features
were translated into their root words. Indeed, nearly half of the 48925
cue-feature pairs were repeated, as 24449 cue-feature pairs were unique
when examining translated features.

36030 affix values were found, which was for 4407 cue concepts. 33052
first affixes were found, with 2832 second place affixes, and 146 third
place affixes. Table \ref{tab:affix-table} shows the distribution of
these affix values. Generally, numbers were the largest category of
affixes indicating that participants often indicated the quantity of the
feature when describing the cue word. The second largest affix category
was characteristics which often indicated the switch to or from a noun
form of the feature word. Verb tenses (past tense, present participle,
and third person) comprised a large set of affixes indicating the type
of concept or when a concept might be doing an action for a cue. Persons
and objects were also indicated about 7\% of the time, while actions and
processes of the cue were mentioned about 8\% of the time.

\subsection{Convergent Validity}\label{convergent-validity}

To examine the validity of cosine values, we calculated the average
cosine score between the new processing of the data for each of the
three feature production norms used in this project. Overlapping cues in
each of three database sets were found (\emph{n} = 188), and the average
cosine between their feature sets was examined. Buchanan et al. (2013)
and the new dataset are listed the subscript B, while McRae et al.
(2005) is M and V for Vinson and Vigliocco (2008). For root cosine
values, we found high overlap between all three datasets: \(M_{BM}\) =
.67 (\emph{SD} = .14), \(M_{BV}\) = .66 (\emph{SD} = .18), and
\(M_{MV}\) = .72 (\emph{SD} = .11). The raw cosine values also
overlapped well, even though the McRae et al. (2005) and Vinson and
Vigliocco (2008) datasets were already mostly preprocessed for word
stems: \(M_{BM}\) = .55 (\emph{SD} = .15), \(M_{BV}\) = .54 (\emph{SD} =
.20), and \(M_{MV}\) = .45 (\emph{SD} = .19). Last, the affix cosines
overlapped well between BM datasets, \(M_{BM}\) = .43 (\emph{SD} = .29),
but did not overlpa with the V datasets: \(M_{BV}\) = .04 (\emph{SD} =
.14), and \(M_{MV}\) = .09 (\emph{SD} = .19). Last, the correlation
between root, raw, affix, old cosine, LSA, and JCN were calculated to
examine convergent validity. As shown in Table
\ref{tab:correlation-table}, the intercorrelations between the cosine
measures are high, especially between our previous work and this
dataset. JCN is backwards coded, as zero values indicate close semantic
neighbors (low dictionary distance) and high values indicate low
semantic relation. The small negative correlations replicate previous
findings. LSA values showed small positive correlations with cosine
values, indicating some overlap with thematic information and semantic
feature overlap (Maki \& Buchanan, 2008). These correlations are
slightly different than our previous publication, likely because we
restricted this cosine set to values with at least two features in
common. The results are simiilar, where LSA and JCN correlations are
lower than LSA-COS and JCN-COS, but these values indicate that themes
and dictionary distance are similarly related to feature overlap.

\begin{table}[tbp]
\begin{center}
\begin{threeparttable}
\caption{\label{tab:correlation-table}}
\begin{tabular}{lcccccccc}
\toprule
  & Root & Raw & Affix & Previous COS & JCN & LSA & FSG & BSG\\
\midrule
Root & 1 &  &  &  &  &  &  & \\
Raw & 0.93 & 1 &  &  &  &  &  & \\
Affix & 0.5 & 0.53 & 1 &  &  &  &  & \\
Previous COS & 0.94 & 0.91 & 0.49 & 1 &  &  &  & \\
JCN & -0.18 & -0.22 & -0.17 & -0.22 & 1 &  &  & \\
LSA & 0.18 & 0.15 & 0.1 & 0.21 & -0.06 & 1 &  & \\
FSG & 0.06 & 0.04 & 0.08 & 0.1 & -0.15 & 0.24 & 1 & \\
BSG & 0.14 & 0.15 & 0.17 & 0.18 & -0.18 & 0.26 & 0.31 & 1\\
\bottomrule
\addlinespace
\end{tabular}
\begin{tablenotes}[para]
\textit{Note.} 
\end{tablenotes}
\end{threeparttable}
\end{center}
\end{table}

\subsection{Divergent Validity}\label{divergent-validity}

Adjective Noun Verb Other 51.85612 36.47896 32.14514 44.44444 Adjective
Noun Verb Other 0.1180191 0.1090159 0.1076604 0.1334710 Adjective Noun
Verb Other 0.1495435 0.1379780 0.1344069 0.1827018 Adjective Noun Verb
Other 0.010 0.010 0.010 0.013 Adjective Noun Verb Other 0.939 0.913
0.936 0.878

\begin{table}[tbp]
\begin{center}
\begin{threeparttable}
\caption{\label{tab:divergent-table}Percent and Mean Overlap between cue-feature lists from the free association norms}
\begin{tabular}{lccccc}
\toprule
  & \% Overlap & $M$ FSG & $SD$ FSG & Min & Max\\
\midrule
Adjective & 51.86 & .12 & .15 & .01 & .94\\
Noun & 36.48 & .11 & .14 & .01 & .91\\
Verb & 32.15 & .11 & .13 & .01 & .94\\
Other & 44.44 & .13 & .18 & .01 & .88\\
Total & 37.47 & .11 & .14 & .01 & .94\\
\bottomrule
\end{tabular}
\end{threeparttable}
\end{center}
\end{table}

When collecting semantic feature production norms, there can be a
concern that the information produced will simply mimic the free
association norms, and thus, be a representation of association
(context) rather than semanticity (meaning). Table
\ref{tab:divergent-table} portrays the overlap with the Nelson et al.
(2004) norms. The percent of time a cue-feature combination was present
in the free association norms was calculated, along with the average FSG
for those overlapping pairs. These values were calculated on the
complete dataset with the McRae et al. (2005) and Vinson and Vigliocco
(2008) norms, as we are presenting them as a combined dataset, on the
translated cue-feature set only. The overall overlap between the
database cue-feature sets and the free association cue-target sets was
approximately 37\%, ranging from 32\% for verbs and nearly 52\% for
adjectives. Similiar to our previous results, the range of the FSG was
large (.01 - .94), however, the average FSG was low for these
overlapping pairs, \emph{M} = .11 (\emph{SD} = .14). These results
indicated that while it will always be difficult to separate association
and meaning, the dataset presented here represents a low association
when examining overlapping values, and more than 60\% of the data is
completely separate from the free association norms. The limitation to
this finding is the removal of idiosyncratic responses from the Nelson
et al. (2004) norms, but even if these were to be included in some form,
the average FSG would still be quite low when comparing cue-feature
lists to cue-target lists.

\subsection{Semantic Priming
Prediction}\label{semantic-priming-prediction}

A secondary goal of this

\section{Discussion}\label{discussion}

With these values, we believe we have captured some of the nuisance that
is often discarded in this type of research. (affixes)

\newpage

\section{References}\label{references}

\setlength{\parindent}{-0.5in} \setlength{\leftskip}{0.5in}

\hypertarget{refs}{}
\hypertarget{ref-R-papaja}{}
Aust, F., \& Barth, M. (2018). \emph{papaja: Create APA manuscripts with
R Markdown}. Retrieved from \url{https://github.com/crsh/papaja}

\hypertarget{ref-Balota2007}{}
Balota, D. A., Yap, M. J., Cortese, M. J., Hutchison, K. A., Kessler,
B., Loftis, B., \ldots{} Treiman, R. (2007). The English Lexicon
Project. \emph{Behavior Research Methods}, \emph{39}(3), 445--459.
doi:\href{https://doi.org/10.3758/BF03193014}{10.3758/BF03193014}

\hypertarget{ref-Buchanan2013}{}
Buchanan, E. M., Holmes, J. L., Teasley, M. L., \& Hutchison, K. A.
(2013). English semantic word-pair norms and a searchable Web portal for
experimental stimulus creation. \emph{Behavior Research Methods},
\emph{45}(3), 746--757.
doi:\href{https://doi.org/10.3758/s13428-012-0284-z}{10.3758/s13428-012-0284-z}

\hypertarget{ref-Buhrmester2011}{}
Buhrmester, M., Kwang, T., \& Gosling, S. D. (2011). Amazon's Mechanical
Turk. \emph{Perspectives on Psychological Science}, \emph{6}(1), 3--5.
doi:\href{https://doi.org/10.1177/1745691610393980}{10.1177/1745691610393980}

\hypertarget{ref-R-shiny}{}
Chang, W., Cheng, J., Allaire, J., Xie, Y., \& McPherson, J. (2017).
\emph{Shiny: Web application framework for r}. Retrieved from
\url{https://CRAN.R-project.org/package=shiny}

\hypertarget{ref-Maki2008}{}
Maki, W. S., \& Buchanan, E. M. (2008). Latent structure in measures of
associative, semantic, and thematic knowledge. \emph{Psychonomic
Bulletin \& Review}, \emph{15}(3), 598--603.
doi:\href{https://doi.org/10.3758/PBR.15.3.598}{10.3758/PBR.15.3.598}

\hypertarget{ref-Maki2004}{}
Maki, W. S., McKinley, L. N., \& Thompson, A. G. (2004). Semantic
distance norms computed from an electronic dictionary (WordNet).
\emph{Behavior Research Methods, Instruments, \& Computers},
\emph{36}(3), 421--431.
doi:\href{https://doi.org/10.3758/BF03195590}{10.3758/BF03195590}

\hypertarget{ref-McRae2005}{}
McRae, K., Cree, G. S., Seidenberg, M. S., \& McNorgan, C. (2005).
Semantic feature production norms for a large set of living and
nonliving things. \emph{Behavior Research Methods}, \emph{37}(4),
547--559.
doi:\href{https://doi.org/10.3758/BF03192726}{10.3758/BF03192726}

\hypertarget{ref-Nelson2004}{}
Nelson, D. L., McEvoy, C. L., \& Schreiber, T. A. (2004). The University
of South Florida free association, rhyme, and word fragment norms.
\emph{Behavior Research Methods, Instruments, \& Computers},
\emph{36}(3), 402--407.
doi:\href{https://doi.org/10.3758/BF03195588}{10.3758/BF03195588}

\hypertarget{ref-Vinson2008}{}
Vinson, D. P., \& Vigliocco, G. (2008). Semantic feature production
norms for a large set of objects and events. \emph{Behavior Research
Methods}, \emph{40}(1), 183--190.
doi:\href{https://doi.org/10.3758/BRM.40.1.183}{10.3758/BRM.40.1.183}






\end{document}
