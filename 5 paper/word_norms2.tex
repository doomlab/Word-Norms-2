\documentclass[english,man]{apa6}

\usepackage{amssymb,amsmath}
\usepackage{ifxetex,ifluatex}
\usepackage{fixltx2e} % provides \textsubscript
\ifnum 0\ifxetex 1\fi\ifluatex 1\fi=0 % if pdftex
  \usepackage[T1]{fontenc}
  \usepackage[utf8]{inputenc}
\else % if luatex or xelatex
  \ifxetex
    \usepackage{mathspec}
    \usepackage{xltxtra,xunicode}
  \else
    \usepackage{fontspec}
  \fi
  \defaultfontfeatures{Mapping=tex-text,Scale=MatchLowercase}
  \newcommand{\euro}{€}
\fi
% use upquote if available, for straight quotes in verbatim environments
\IfFileExists{upquote.sty}{\usepackage{upquote}}{}
% use microtype if available
\IfFileExists{microtype.sty}{\usepackage{microtype}}{}

% Table formatting
\usepackage{longtable, booktabs}
\usepackage{lscape}
% \usepackage[counterclockwise]{rotating}   % Landscape page setup for large tables
\usepackage{multirow}		% Table styling
\usepackage{tabularx}		% Control Column width
\usepackage[flushleft]{threeparttable}	% Allows for three part tables with a specified notes section
\usepackage{threeparttablex}            % Lets threeparttable work with longtable

% Create new environments so endfloat can handle them
% \newenvironment{ltable}
%   {\begin{landscape}\begin{center}\begin{threeparttable}}
%   {\end{threeparttable}\end{center}\end{landscape}}

\newenvironment{lltable}
  {\begin{landscape}\begin{center}\begin{ThreePartTable}}
  {\end{ThreePartTable}\end{center}\end{landscape}}

  \usepackage{ifthen} % Only add declarations when endfloat package is loaded
  \ifthenelse{\equal{\string man}{\string man}}{%
   \DeclareDelayedFloatFlavor{ThreePartTable}{table} % Make endfloat play with longtable
   % \DeclareDelayedFloatFlavor{ltable}{table} % Make endfloat play with lscape
   \DeclareDelayedFloatFlavor{lltable}{table} % Make endfloat play with lscape & longtable
  }{}%



% The following enables adjusting longtable caption width to table width
% Solution found at http://golatex.de/longtable-mit-caption-so-breit-wie-die-tabelle-t15767.html
\makeatletter
\newcommand\LastLTentrywidth{1em}
\newlength\longtablewidth
\setlength{\longtablewidth}{1in}
\newcommand\getlongtablewidth{%
 \begingroup
  \ifcsname LT@\roman{LT@tables}\endcsname
  \global\longtablewidth=0pt
  \renewcommand\LT@entry[2]{\global\advance\longtablewidth by ##2\relax\gdef\LastLTentrywidth{##2}}%
  \@nameuse{LT@\roman{LT@tables}}%
  \fi
\endgroup}


\ifxetex
  \usepackage[setpagesize=false, % page size defined by xetex
              unicode=false, % unicode breaks when used with xetex
              xetex]{hyperref}
\else
  \usepackage[unicode=true]{hyperref}
\fi
\hypersetup{breaklinks=true,
            pdfauthor={},
            pdftitle={Word Norms 2 This Paper Will Get Finished},
            colorlinks=true,
            citecolor=blue,
            urlcolor=blue,
            linkcolor=black,
            pdfborder={0 0 0}}
\urlstyle{same}  % don't use monospace font for urls

\setlength{\parindent}{0pt}
%\setlength{\parskip}{0pt plus 0pt minus 0pt}

\setlength{\emergencystretch}{3em}  % prevent overfull lines

\ifxetex
  \usepackage{polyglossia}
  \setmainlanguage{}
\else
  \usepackage[english]{babel}
\fi

% Manuscript styling
\captionsetup{font=singlespacing,justification=justified}
\usepackage{csquotes}
\usepackage{upgreek}

 % Line numbering
  \usepackage{lineno}
  \linenumbers


\usepackage{tikz} % Variable definition to generate author note

% fix for \tightlist problem in pandoc 1.14
\providecommand{\tightlist}{%
  \setlength{\itemsep}{0pt}\setlength{\parskip}{0pt}}

% Essential manuscript parts
  \title{Word Norms 2 This Paper Will Get Finished}

  \shorttitle{Title}


  \author{First Author\textsuperscript{1}~\& Ernst-August Doelle\textsuperscript{1,2}}

  % \def\affdep{{"", ""}}%
  % \def\affcity{{"", ""}}%

  \affiliation{
    \vspace{0.5cm}
          \textsuperscript{1} Wilhelm-Wundt-University\\
          \textsuperscript{2} Konstanz Business School  }

  \authornote{
    Add complete departmental affiliations for each author here. Each new
    line herein must be indented, like this line.
    
    Enter author note here.
    
    Correspondence concerning this article should be addressed to First
    Author, Postal address. E-mail:
    \href{mailto:my@email.com}{\nolinkurl{my@email.com}}
  }


  \abstract{Enter abstract here. Each new line herein must be indented, like this
line.}
  \keywords{keywords \\

    \indent Word count: X
  }





\usepackage{amsthm}
\newtheorem{theorem}{Theorem}[section]
\newtheorem{lemma}{Lemma}[section]
\theoremstyle{definition}
\newtheorem{definition}{Definition}[section]
\newtheorem{corollary}{Corollary}[section]
\newtheorem{proposition}{Proposition}[section]
\theoremstyle{definition}
\newtheorem{example}{Example}[section]
\theoremstyle{definition}
\newtheorem{exercise}{Exercise}[section]
\theoremstyle{remark}
\newtheorem*{remark}{Remark}
\newtheorem*{solution}{Solution}
\begin{document}

\maketitle

\setcounter{secnumdepth}{0}



\section{Methods}\label{methods}

We report how we determined our sample size, all data exclusions (if
any), all manipulations, and all measures in the study.

\subsection{Participants}\label{participants}

\begin{table}[tbp]
\begin{center}
\begin{threeparttable}
\caption{\label{tab:unnamed-chunk-1}}
\begin{tabular}{lccc}
\toprule
Institution & Total Participants & Concepts & Mean $N$\\
\midrule
University of Mississippi & 749 & 658 & 67.8\\
Missouri State University & 1420 & 720 & 71.4\\
Montana State University & 127 & 120 & 63.5\\
Mechanical Turk 1 & 571 & 310 & 60\\
Mechanical Turk 2 & 198 & 1914 & 30\\
\bottomrule
\addlinespace
\end{tabular}
\begin{tablenotes}[para]
\textit{Note.} 
\end{tablenotes}
\end{threeparttable}
\end{center}
\end{table}

\subsection{Material}\label{material}

\begin{table}[tbp]
\begin{center}
\begin{threeparttable}
\caption{\label{tab:unnamed-chunk-3}}
\begin{tabular}{lccccc}
\toprule
Institution & Adjective & Noun & Verb & Other & Total\\
\midrule
University of Mississippi & 10.97(12.48) & 10.40(13.08) & 5.28(6.71) & 10.01(10.16) & 9.45(12.02)\\
Missouri State University & 10.41(11.57) & 10.71(13.48) & 5.87(6.46) & 10.11(11.74) & 9.52(11.94)\\
Montana State University & 10.95(11.58) & 10.66(12.13) & 5.89(7.86) & 11.27(10.64) & 9.67(11.34)\\
Mechanical Turk 1 & 9.00(8.92) & 9.73(11.35) & 5.17(6.64) & 11.30(10.41) & 8.65(10.17)\\
Mechanical Turk 2 & 5.23(4.85) & 5.71(5.15) & 3.36(3.50) & 5.87(4.84) & 5.02(4.81)\\
Total & 8.32(9.62) & 8.19(10.07) & 4.55(5.91) & 8.31(8.52) & 7.42(9.30)\\
\bottomrule
\addlinespace
\end{tabular}
\begin{tablenotes}[para]
\textit{Note.} 
\end{tablenotes}
\end{threeparttable}
\end{center}
\end{table}

\subsection{Procedure}\label{procedure}

\subsection{Data analysis}\label{data-analysis}

We used R (Version 3.4.3; R Core Team, 2017) and the R-packages
\emph{lattice} (Version 0.20.35; Sarkar, 2008), \emph{MASS} (Version
7.3.49; Venables \& Ripley, 2002), \emph{memisc} (Version 0.99.14.9;
Elff, 2017), \emph{papaja} (Version 0.1.0.9709; Aust \& Barth, 2018),
and \emph{reshape} (Version 0.8.6; Wickham \& Hadley, 2007) for all our
analyses.

\section{Results}\label{results}

\begin{table}[tbp]
\begin{center}
\begin{threeparttable}
\caption{\label{tab:unnamed-chunk-5}}
\begin{tabular}{llcccc}
\toprule
Cue Type & Feature Type & Raw & Root & $M$ Freq. Raw & $M$ Freq. Root\\
\midrule
Adjective & Adjective & 38.09 & 29.74 & 17.84(16.47) & 30.02(18.83)\\
 & Noun & 40.02 & 46.74 & 13.14(14.96) & 29.71(19.94)\\
 & Verb & 17.69 & 20.72 & 8.51(9.78) & 26.88(17.27)\\
 & Other & 4.2 & 2.8 & 15.17(15.64) & 28.04(15.54)\\
Noun & Adjective & 16.56 & 12.07 & 15.55(15.17 & 31.20(18.17)\\
 & Noun & 60.85 & 62.67 & 17.21(17.01) & 33.26(20.05)\\
 & Verb & 20.8 & 23.68 & 8.88(9.73) & 31.01(17.87)\\
 & Other & 1.79 & 1.58 & 17.06(15.29) & 28.87(17.14)\\
Verb & Adjective & 15.16 & 12.27 & 13.95(13.98) & 30.03(18.28\\
 & Noun & 42.92 & 44.35 & 14.59(14.92) & 29.59(18.90\\
 & Verb & 36.92 & 39.72 & 12.75(14.85) & 30.43(19.54)\\
 & Other & 5 & 3.66 & 19.16(15.95) & 25.59(19.54)\\
Other & Adjective & 20.8 & 20.32 & 16.61(17.37) & 31.66(19.51)\\
 & Noun & 42.74 & 39.03 & 16.77(19.41) & 37.28(25.94\\
 & Verb & 19.66 & 23.93 & 7.18(7.57) & 26.14(19.38\\
 & Other & 16.81 & 16.71 & 22.72(16.69) & 30.70(18.48)\\
Total & Adjective & 19.74 & 14.93 & 16.12(15.57) & 30.75(18.37)\\
 & Noun & 55.41 & 57.81 & 16.55(16.74) & 32.58(20.09)\\
 & Verb & 22.02 & 24.95 & 9.50(10.91) & 30.29(18.24)\\
 & Other & 2.82 & 2.31 & 17.76(15.83) & 28.45(16.83)\\
\bottomrule
\addlinespace
\end{tabular}
\begin{tablenotes}[para]
\textit{Note.} 
\end{tablenotes}
\end{threeparttable}
\end{center}
\end{table}

\begin{table}[tbp]
\begin{center}
\begin{threeparttable}
\caption{\label{tab:unnamed-chunk-7}}
\begin{tabular}{llc}
\toprule
Affix Tag & Example & Percent\\
\midrule
Actions/Processes & ion, ment, ble, ate, ize & 7.38\\
Characteristic & y, ous, nt, ful, ive, wise & 23.55\\
Location & under, sub, mid, inter & 0.47\\
Magnitude & er, est, over, super, extra & 1.54\\
Not & less, dis, un, non, in , im, ab & 2.76\\
Number & s, uni, bi, tri, semi & 26.05\\
Opposites/Wrong & mis, anti, de & 0.15\\
Past Tense & ed & 9.11\\
Person/Object & er, or, men, person, ess, ist & 8.18\\
Present Participle & ing & 14.33\\
Slang & bros, bike, bbq, diff, h2o & 0.12\\
Third Person & s & 5.85\\
Time & fore, pre, post, re & 0.51\\
\bottomrule
\addlinespace
\end{tabular}
\begin{tablenotes}[para]
\textit{Note.} 
\end{tablenotes}
\end{threeparttable}
\end{center}
\end{table}

\begin{table}[tbp]
\begin{center}
\begin{threeparttable}
\caption{\label{tab:unnamed-chunk-11}}
\begin{tabular}{lcccccccc}
\toprule
  & Root & Raw & Affix & Previous COS & JCN & LSA & FSG & BSG\\
\midrule
Root & 1 &  &  &  &  &  &  & \\
Raw & 0.93 & 1 &  &  &  &  &  & \\
Affix & 0.5 & 0.53 & 1 &  &  &  &  & \\
Previous COS & 0.94 & 0.91 & 0.49 & 1 &  &  &  & \\
JCN & -0.18 & -0.22 & -0.17 & -0.22 & 1 &  &  & \\
LSA & 0.18 & 0.15 & 0.1 & 0.21 & -0.06 & 1 &  & \\
FSG & 0.06 & 0.04 & 0.08 & 0.1 & -0.15 & 0.24 & 1 & \\
BSG & 0.14 & 0.15 & 0.17 & 0.18 & -0.18 & 0.26 & 0.31 & 1\\
\bottomrule
\addlinespace
\end{tabular}
\begin{tablenotes}[para]
\textit{Note.} 
\end{tablenotes}
\end{threeparttable}
\end{center}
\end{table}

\section{Discussion}\label{discussion}

\newpage

\section{References}\label{references}

\setlength{\parindent}{-0.5in} \setlength{\leftskip}{0.5in}

\hypertarget{refs}{}
\hypertarget{ref-R-papaja}{}
Aust, F., \& Barth, M. (2018). \emph{papaja: Create APA manuscripts with
R Markdown}. Retrieved from \url{https://github.com/crsh/papaja}

\hypertarget{ref-R-memisc}{}
Elff, M. (2017). \emph{Memisc: Management of survey data and
presentation of analysis results}. Retrieved from
\url{https://CRAN.R-project.org/package=memisc}

\hypertarget{ref-R-base}{}
R Core Team. (2017). \emph{R: A language and environment for statistical
computing}. Vienna, Austria: R Foundation for Statistical Computing.
Retrieved from \url{https://www.R-project.org/}

\hypertarget{ref-R-lattice}{}
Sarkar, D. (2008). \emph{Lattice: Multivariate data visualization with
r}. New York: Springer. Retrieved from
\url{http://lmdvr.r-forge.r-project.org}

\hypertarget{ref-R-MASS}{}
Venables, W. N., \& Ripley, B. D. (2002). \emph{Modern applied
statistics with s} (Fourth.). New York: Springer. Retrieved from
\url{http://www.stats.ox.ac.uk/pub/MASS4}

\hypertarget{ref-R-reshape}{}
Wickham, \& Hadley. (2007). Reshaping data with the reshape package.
\emph{Journal of Statistical Software}, \emph{21}(12). Retrieved from
\url{http://www.jstatsoft.org/v21/i12/paper}






\end{document}
