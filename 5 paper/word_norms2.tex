\documentclass[english,man]{apa6}

\usepackage{amssymb,amsmath}
\usepackage{ifxetex,ifluatex}
\usepackage{fixltx2e} % provides \textsubscript
\ifnum 0\ifxetex 1\fi\ifluatex 1\fi=0 % if pdftex
  \usepackage[T1]{fontenc}
  \usepackage[utf8]{inputenc}
\else % if luatex or xelatex
  \ifxetex
    \usepackage{mathspec}
    \usepackage{xltxtra,xunicode}
  \else
    \usepackage{fontspec}
  \fi
  \defaultfontfeatures{Mapping=tex-text,Scale=MatchLowercase}
  \newcommand{\euro}{€}
\fi
% use upquote if available, for straight quotes in verbatim environments
\IfFileExists{upquote.sty}{\usepackage{upquote}}{}
% use microtype if available
\IfFileExists{microtype.sty}{\usepackage{microtype}}{}

% Table formatting
\usepackage{longtable, booktabs}
\usepackage{lscape}
% \usepackage[counterclockwise]{rotating}   % Landscape page setup for large tables
\usepackage{multirow}		% Table styling
\usepackage{tabularx}		% Control Column width
\usepackage[flushleft]{threeparttable}	% Allows for three part tables with a specified notes section
\usepackage{threeparttablex}            % Lets threeparttable work with longtable

% Create new environments so endfloat can handle them
% \newenvironment{ltable}
%   {\begin{landscape}\begin{center}\begin{threeparttable}}
%   {\end{threeparttable}\end{center}\end{landscape}}

\newenvironment{lltable}
  {\begin{landscape}\begin{center}\begin{ThreePartTable}}
  {\end{ThreePartTable}\end{center}\end{landscape}}

  \usepackage{ifthen} % Only add declarations when endfloat package is loaded
  \ifthenelse{\equal{\string man}{\string man}}{%
   \DeclareDelayedFloatFlavor{ThreePartTable}{table} % Make endfloat play with longtable
   % \DeclareDelayedFloatFlavor{ltable}{table} % Make endfloat play with lscape
   \DeclareDelayedFloatFlavor{lltable}{table} % Make endfloat play with lscape & longtable
  }{}%



% The following enables adjusting longtable caption width to table width
% Solution found at http://golatex.de/longtable-mit-caption-so-breit-wie-die-tabelle-t15767.html
\makeatletter
\newcommand\LastLTentrywidth{1em}
\newlength\longtablewidth
\setlength{\longtablewidth}{1in}
\newcommand\getlongtablewidth{%
 \begingroup
  \ifcsname LT@\roman{LT@tables}\endcsname
  \global\longtablewidth=0pt
  \renewcommand\LT@entry[2]{\global\advance\longtablewidth by ##2\relax\gdef\LastLTentrywidth{##2}}%
  \@nameuse{LT@\roman{LT@tables}}%
  \fi
\endgroup}


\ifxetex
  \usepackage[setpagesize=false, % page size defined by xetex
              unicode=false, % unicode breaks when used with xetex
              xetex]{hyperref}
\else
  \usepackage[unicode=true]{hyperref}
\fi
\hypersetup{breaklinks=true,
            pdfauthor={},
            pdftitle={English Semantic Feature Production Norms: An Extended Database of 4,437 Concepts},
            colorlinks=true,
            citecolor=blue,
            urlcolor=blue,
            linkcolor=black,
            pdfborder={0 0 0}}
\urlstyle{same}  % don't use monospace font for urls

\setlength{\parindent}{0pt}
%\setlength{\parskip}{0pt plus 0pt minus 0pt}

\setlength{\emergencystretch}{3em}  % prevent overfull lines

\ifxetex
  \usepackage{polyglossia}
  \setmainlanguage{}
\else
  \usepackage[english]{babel}
\fi

% Manuscript styling
\captionsetup{font=singlespacing,justification=justified}
\usepackage{csquotes}
\usepackage{upgreek}

 % Line numbering
  \usepackage{lineno}
  \linenumbers


\usepackage{tikz} % Variable definition to generate author note

% fix for \tightlist problem in pandoc 1.14
\providecommand{\tightlist}{%
  \setlength{\itemsep}{0pt}\setlength{\parskip}{0pt}}

% Essential manuscript parts
  \title{English Semantic Feature Production Norms: An Extended Database of 4,437
Concepts}

  \shorttitle{Semantic Norms}


  \author{Erin M. Buchanan\textsuperscript{1}, K. D. Valentine\textsuperscript{2}, \& Nicholas P. Maxwell\textsuperscript{1}}

  % \def\affdep{{"", "", ""}}%
  % \def\affcity{{"", "", ""}}%

  \affiliation{
    \vspace{0.5cm}
          \textsuperscript{1} Missouri State University\\
          \textsuperscript{2} University of Missouri  }

  \authornote{
    Erin M. Buchanan is an Associate Professor of Quantitative Psychology at
    Missouri State University. K. D. Valentine is a Ph.D.~candidate at the
    University of Missouri. Nicholas P. Maxwell is a Masters' candidate at
    Missouri State University.
    
    We would like to thank Keith Hutchison and David Balota for their
    contributions to this project, including the funds to secure Mechanical
    Turk participants.
    
    Correspondence concerning this article should be addressed to Erin M.
    Buchanan, 901 S. National Ave, Springfield, MO 65897. E-mail:
    \href{mailto:erinbuchanan@missouristate.edu}{\nolinkurl{erinbuchanan@missouristate.edu}}
  }


  \abstract{The largest limiting factor in understanding memory and language
networks is often the availability of normed stimuli to use and explore
in experimental studies. In this study, we expand on three previous
semantic feature overlap norms to over 4,000 cue stimuli ranging from
nouns, verbs, adjectives, and other parts of speech. Participants in the
norming study were asked to provide feature components of each cue
stimuli, which were combined with the previous research using semantic
feature production procedures. In addition to expanding previous
research, this project explores different semantic overlap measurements
by coding each word feature listed by root and affixes to determine
different strengths of feature overlap. All information is provided in a
searchable database for easy access and utilization for future
researchers when designing experiments. The final database of cue-target
pairs was paired with the Semantic Priming Project to examine the
relation of feature overlap statistics on semantic priming in tandem
with other psycholinguistic variables, such as association and
thematics.}
  \keywords{semantics, word norms, database, psycholinguistics \\

    
  }





\usepackage{amsthm}
\newtheorem{theorem}{Theorem}
\newtheorem{lemma}{Lemma}
\theoremstyle{definition}
\newtheorem{definition}{Definition}
\newtheorem{corollary}{Corollary}
\newtheorem{proposition}{Proposition}
\theoremstyle{definition}
\newtheorem{example}{Example}
\theoremstyle{definition}
\newtheorem{exercise}{Exercise}
\theoremstyle{remark}
\newtheorem*{remark}{Remark}
\newtheorem*{solution}{Solution}
\begin{document}

\maketitle

\setcounter{secnumdepth}{0}



Semantic representations are the focus of a large area of research which
tries to delineate the essential features of a concept. These features
are key to many models of semantic memory (Collins \& Loftus, 1975;
Collins \& Quillian, 1969), and they have been used to create both
feature based (Cree \& McRae, 2003; Smith, Shoben, \& Rips, 1974;
Vigliocco, Vinson, Lewis, \& Garrett, 2004) and distributional based
models (Griffiths, Steyvers, \& Tenenbaum, 2007; Jones \& Mewhort, 2007;
Riordan \& Jones, 2011). Category set creation was a seminal task with
corresponding norms that have been prevelant in the literature
(Ashcraft, 1978; Rosch, Mervis, Rosch, \& Mervis, 1975; Toglia, 2009;
Toglia \& Battig, 1978). Feature production norms are created by
soliciting participants to list properties or features of a target
concept. These features are then compiled into feature sets that are
thought to represent, at least somewhat, the memory representation of a
particular concept. Previous work on semantic feature production norms
in English includes databases by Buchanan, Holmes, Teasley, and
Hutchison (2013), McRae, Cree, Seidenberg, and McNorgan (2005), and
Vinson and Vigliocco (2008).

For example, when queried on what defines a \emph{cat}, participants may
list \emph{tail}, \emph{animal}, and \emph{pet}. These features capture
the most common types of descriptons: \enquote{is a} and \enquote{has
a}. Additionally, feature descriptions may include uses, locations,
behavior, and gender (i.e., \emph{actor} denotes both a person and
gender). The goal of these norms is often to create a set of
high-probability features, as there can and will be many idiosyncratic
features listed in this task, corresponding to the theory of fuzzy logic
for category representation (Medin, 1989). These norms have now been
published in Italian (Montefinese, Ambrosini, Fairfield, \& Mammarella,
2013; Reverberi, Capitani, \& Laiacona, 2004), German (and Italian,
Kremer \& Baroni, 2011), Portuguese (Stein \& de Azevedo Gomes, 2009),
Spanish (Vivas, Vivas, Comesaña, \& Coni, 2017), and Dutch (Ruts et al.,
2004), as well as for the blind (Lenci, Baroni, \& Cazzolli, 2013).

The data from these studies has been used to explain many semantic based
phenomena in several ways. First, the feature production norms can be
used as the underlying data to create models of semantic priming and
cognition (Cree, McRae, \& McNorgan, 1999; Rogers \& McClelland, 2004;
Vigliocco et al., 2004). Moss, Tyler, and Devlin (2002) explored how
deficits in cateories may arise with production norms, and since these
studies focus on the likelihood of cue-feature combinations, features
can be used to examine the probablistic nature of language (Cree \&
McRae, 2003; McRae, Sa, \& Seidenberg, 1997; Pexman, Holyk, \& Monfils,
2003). When using database norms to select for stimuli, others have
studied semantic word-picture interference (Vieth, Mcmahon, \&
Zubicaray, 2014), recognition memory (Montefinese, Daniele, \&
Ambrosini, 2015), and semantic richness, which is a measure of shared
defining features (Grondin, Lupker, \& McRae, 2009; Kounios et al.,
2009; Yap \& Pexman, 2016; Yap, Lim, \& Pexman, 2015). The Vinson and
Vigliocco labs have shown the power of turning in-house data projects
into a larger norming set (Vinson \& Vigliocco, 2008), as they published
papers on aphasia (Vinson \& Vigliocco, 2002; Vinson, Vigliocco, Cappa,
\& Siri, 2003), meaning-syntactic differences (Vigliocco, Vinson, \&
Siri, 2005; Vigliocco, Vinson, Damian, \& Levelt, 2002), and
representational models (Vigliocco et al., 2004).

However, it would be unwise to consider these norms as an exact
representation of a concept in memory (McRae et al., 2005). These norms
represent salient features that participants can recall, likely because
saliency is considered special to our understanding of concepts (Cree \&
McRae, 2003). Additionally, Barsalou (2003) suggested that participants
are likely creating a mental model of the concept based on experience
and using that model to create a feature property list. This model may
represent a specific instance of a category (i.e., their pet dog), and
feature lists will represent that particular memory.

The advantage of big data in psycholinguistics cannot be understated.
Computational modeling of memory requires sufficiently large datasets to
accurately portray semantic memory. There are many large corpora that
could be used for exploring langauge through frequency (see the SUBTLEX
projects Brysbaert \& New, 2009; New, Brysbaert, Veronis, \& Pallier,
2007). Additionally, there are large lexicon projects that explore how
the basic features of words, such as orthographic neighborhood, length,
and part of speech, affect semantic priming (Balota et al., 2007;
Keuleers, Lacey, Rastle, \& Brysbaert, 2012). Large databases of age of
acquisition (Kuperman, Stadthagen-Gonzalez, \& Brysbaert, 2012),
concreteness (Brysbaert, Warriner, \& Kuperman, 2014), and valence
(Warriner, Kuperman, \& Brysbaert, 2013) provide further avenues for
understanding the impact these rated properties have on semantic memory.
For example, age of acquisition and concreteness ratings have been shown
to predict performance on recall tasks (Brysbaert et al., 2014;
Dewhurst, Hitch, \& Barry, 1998), while valence ratings are useful for
gauging the effects of emotion on meaning (Warriner et al., 2013).
However, even as we detail here a small subset of the larger normed
stimuli avaliable (Buchanan, Valentine, \& Maxwell, 2018), research can
be limited by the overlap between these datasets. If a researcher wishes
to control for lexical and relational variables, the inclusion of each
new variable to the study will further restrict the item pool for study.
Large, overlapping datasets are crucial for exploring the entire range
of an effect, and to ensure that the stimuli set is not the only
contributing factor to the results of a study.

Therefore, the purpose of this study is to further expand the stimuli
and variable options avaliable to the field, as well as promote the use
of these norms for stimuli creation. To accomplish these goals, we have
expanded our original semantic feature production norms (Buchanan et
al., 2013) to include all cues and targets from The Semantic Priming
Project (Hutchison et al., 2013). The existing norms were reprocessed
along with these new norms to explore the impact of feature coding and
affixes on variable creation and prediction. The entire dataset is
avaliable on our website (\url{http://www.wordnorms.com}) which has been
revamped with a new interface and web applications to easily find and
select stimuli for future experiments. The data collection,
(re)processing, website, and finalized dataset are detailed below.

\section{Method}\label{method}

\subsection{Participants}\label{participants}

Participants in the newly collected stimuli set were gathered from
Amazon's Mechanical Turk, which is a large, diverse participant pool
wherein users can complete surveys for small sums of money (Buhrmester,
Kwang, \& Gosling, 2011). Answers can be screened for errors, and
incorrect or incomplete surveys can be rejected or discarded without
payment. Each participant was paid five cents for a survey, and they
could complete multiple Human Intelligence Tasks or HITS. Each HIT
included five concepts, and HITS would remain active until \emph{n} = 30
valid survey answers were collected. HITS were usually rejected if they
included copied definitions from Wikipedia, \enquote{I don't know}, or
writing a paragraph about the concept. These answers were discarded, as
described below. Table \ref{tab:part-table} includes the sample sizes
from the new study (Mechanical Turk 2), as well as the sample sizes from
the previous study, as described in Buchanan et al. (2013).

\begin{table}[tbp]
\begin{center}
\begin{threeparttable}
\caption{\label{tab:part-table}Sample Size and Concept Norming Size for Each Data Collection Location/Time Point}
\begin{tabular}{lccc}
\toprule
Institution & Total Participants & Concepts & Mean $N$\\
\midrule
University of Mississippi & 749 & 658 & 67.8\\
Missouri State University & 1420 & 720 & 71.4\\
Montana State University & 127 & 120 & 63.5\\
Mechanical Turk 1 & 571 & 310 & 60\\
Mechanical Turk 2 & 198 & 1914 & 30\\
\bottomrule
\end{tabular}
\end{threeparttable}
\end{center}
\end{table}

\subsection{Materials}\label{materials}

The purpose of this second norming set was to expand the Buchanan et al.
(2013) norms to include all concepts from the Semantic Priming Project
(Hutchison et al., 2013). The original concept set was selected
primarily from the Nelson, McEvoy, and Schreiber (2004) database, with
small overlaps in the McRae et al. (2005) and Vinson and Vigliocco
(2008) database sets for convergent validity. In the Semantic Priming
Project, cue-target pairs were shown to participants to examine naming
and lexical decision time priming across related and unrelated pairs.
The related pairs included first associate (most common response to a
cue) and other associates (second or greater common responses to cues)
as their target words. The Buchanan et al. (2013) publication of
concepts included the cue words from the Semantic Priming Project, while
this project expanded to include unnormed cue words and all target words
for all first and other associate pairs. The addition of these concepts
allowed for complete overlap between the Semantic Priming Project and
feature production norms.

Concepts were labeled by part of speech using the English Lexicon
Project (Balota et al., 2007), the free association norms, and Google's
define search when necessary. When labelling these words, we used the
most common part of speech to categorize concepts. This choice was
predominately for simplicity of categorization, however, the
participants were shown concepts without the suggestion of which sense
to use for the word. Therefore, multiple senses (i.e., \emph{bat} is
noun and a verb) are embedded into the feature production norms, while
the database is labeled with single parts of speech. The other parts of
speech can be found in the English Lexicon Project or multiple other
databases. This dataset was combined with McRae et al. (2005) and Vinson
and Vigliocco (2008) feature production norms, which resulted in a
combined total of 4437 concepts. 70.4\% of concepts were nouns, 14.9\%
adjectives, 12.4\% verbs, and 2.3\% were other forms of speech, such as
adverbs and conjunctions.

\subsection{Procedure}\label{procedure}

Each HIT was kept to five concepts, and usual survey response times were
between five to seven minutes. Each HIT was open until thirty
participants had successfully completed the HIT and were paid the five
cents for the HIT. The survey instructions were copied from McRae et al.
(2005)'s Appendix B, which were also used in the previous publication of
these norms. Because the McRae et al. (2005) data was collected on
paper, we modified these instructions slightly. The original lines to
write in responses were changed to an online textbox response window.
The detailed instructions additionally no longer contained information
about how a participant should only consider the noun of the target
concept, as the words in our study included multiple forms of speech and
senses. Participants were encouraged to list the properties or features
of each concept in the following areas: physical (looks, sounds, and
feels), functional (uses), and categorical (belongings). The same
examples used previously in McRae et al. (2005) and Buchanan et al.
(2013) (\emph{duck, cucumber, stove}) were included to aid in task
understanding and completion. Participants signed up for the HITS
through Amazon's Mechanical Turk website and completed the study within
the Mechanical Turk framework. Approved HITs were compensated through
the Mechanical Turk system. All answers were then combined into a larger
dataset.

\subsection{Data Processing}\label{data-processing}

The entire dataset, at each processing stage described here, can be
found at: \url{https://osf.io/cjyzw/}. On our OSF page, we have included
a detailed processing guide on how concepts were (re)examined for this
publication. This paper was written with \emph{R} markdown (R Core Team,
2017) and \emph{papaja} (Aust \& Barth, 2018). The markdown document
allows an interested reader to view the scripts that created the article
in line with the written text. However, the processing of the text
documents was performed on the raw files, and therefore, we have
included the processing guide for transparency of each stage.

First, each concept was separated into an individual text file that is
included as the \enquote{raw} data online. Each of these files was then
spell checked and corrected when the participant answer was obviously a
typo. As noted earlier, participants often tried to cut and paste
Wikipedia or other online dictionary sources into the their answers to
complete surveys quickly with minimal effort. These entries were easily
found because the formatting of the webpage was included in their
answer. These answers were then discarded from the individual concept's
text file. Next, each concept was processed for feature frequency. In
this stage, the raw frequency counts of each cue-feature combination
were calculated and put together into one large file. Cue-cue
combinations were discarded, as participants might write \enquote{a
zebra is a horse} when asked to define \emph{zebra}. English stop words
such as \emph{the, an, of} were then discarded, as well as terms that
were often used as part of a definition (\emph{like, means, describes}).

To create the final root cosine values, we then created a
\enquote{translated} column for each feature listed. This column
indicated the root word for each feature, and additional columns were
added with the affixes that were used in the original feature. For
example, the original feature \emph{cats} would be translated to
\emph{cat} and \emph{s}, wherein \emph{cat} would be the translated
feature and the \emph{s} would be the affix code. Multiple affix codes
were often needed for features, as \emph{beautifully} would have been
translated to \emph{beauty}, \emph{ful}, and \emph{ly}. Often, the noun
version of the feature would be used for the translation or the most
common part of speech for each feature would be recorded. The sample
size for the cue was added to this dataset, as the sample sizes varied
across experiment time, as shown in Table \ref{tab:part-table}.
Therefore, instead of using raw feature frequency, we normalized each
count into the percent of participants that included that feature with
each cue.

At this stage, the data was reduced to cue-feature combinations that
were listed by at least 16\% of participants (matching McRae et al.
(2005)'s procedure) or were in the top five features listed for that
cue. This calculation was performed on the translated normalized feature
percent. For example, \emph{beauty} may have been listed as
\emph{beauty, beautiful, beautifully, beautifulness}, and this feature
would have been listed four times in the dataset for the original cue.
The \emph{frequency\_feature} column indicates the frequency of the
original, unedited feature, while the \emph{frequency\_translated}
includes all combinations of \emph{beauty} into one overall feature.
Because non-nouns can be more difficult to create a feature list for, we
included the top five descriptors in addition to the 16\% listed
criteria, to ensure that each concept included at least five features.
Table \ref{tab:feature-table} indicates the average number of
cue-feature pairs found for each data collection site/time point and
part of speech for the cue word.

The parts of speech for the cue, original feature, and translated
feature were merged with this file as described above. Table
\ref{tab:percent-table} depicts the pattern of feature responses for
cue-feature part of speech combinations. This table includes the percent
of features listed for each cue-feature part of speech combination
(i.e., what is the percent of time that both the cue and feature were
both adjectives) for the original feature (raw) and translated feature
(root). Next, the normalized frequency percent average was calculated
along with their standard deviations. These columns indicate the
frequency percent that a cue-feature part of speech combination was
listed across participants (i.e., what is the average percent of
participants that listed an adjective feature for an adjective cue).
These two types of calculation describe the likelihood of seeing part of
speech combinations across the concepts, along with the likelihood of
those cue-feature part of speech combinations across participants.

\begin{table}[tbp]
\begin{center}
\begin{threeparttable}
\caption{\label{tab:feature-table}Average (SD) Cue-Feature Pairs by Location/Time Point}
\begin{tabular}{lccccc}
\toprule
Institution & Adjective & Noun & Verb & Other & Total\\
\midrule
University of Mississippi & 5.57 (1.53) & 7.35 (4.05) & 5.33 (.87) & 6.01 (2.11) & 6.71 (3.44)\\
Missouri State University & 5.74 (1.56) & 6.85 (2.82) & 6.67 (2.08) & 7.45 (5.35) & 6.65 (2.92)\\
Montana State University & 5.81 (1.74) & 7.25 (3.35) & 5.59 (1.13) & 5.76 (1.74) & 6.69 (2.93)\\
Mechanical Turk 1 & 6.27 (2.28) & 7.74 (4.34) & 5.77 (1.17) & 5.57 (1.40) & 7.14 (3.79)\\
Mechanical Turk 2 & 5.76 (1.36) & 6.62 (1.85) & 5.92 (1.38) & 5.78 (1.17) & 6.38 (1.75)\\
Total & 5.78 (1.61) & 6.94 (2.88) & 5.67 (1.18) & 5.84 (1.71) & 6.57 (2.60)\\
\bottomrule
\end{tabular}
\end{threeparttable}
\end{center}
\end{table}

\begin{table}[tbp]
\begin{center}
\begin{threeparttable}
\caption{\label{tab:percent-table}Percents and Average Normalized Percent Frequency for Cue-Feature Part of Speech Combinations}
\begin{tabular}{llcccc}
\toprule
Cue Type & Feature Type & \% Raw & \% Root & $M$ Freq. Raw & $M$ Freq. Root\\
\midrule
Adjective & Adjective & 38.09 & 29.74 & 17.84 (16.47) & 30.02 (18.83)\\
 & Noun & 40.02 & 46.74 & 13.14 (14.96) & 29.71 (19.94)\\
 & Verb & 17.69 & 20.72 & 8.51 (9.78) & 26.88 (17.27)\\
 & Other & 4.20 & 2.80 & 15.17 (15.64) & 28.04 (15.54)\\
Noun & Adjective & 16.56 & 12.07 & 15.55 (15.17) & 31.20 (18.17)\\
 & Noun & 60.85 & 62.67 & 17.21 (17.01) & 33.26 (20.05)\\
 & Verb & 20.80 & 23.68 & 8.88 (9.73) & 31.01 (17.87)\\
 & Other & 1.79 & 1.58 & 17.06 (15.29) & 28.87 (17.14)\\
Verb & Adjective & 15.16 & 12.27 & 13.95 (13.98) & 30.03 (18.28)\\
 & Noun & 42.92 & 44.35 & 14.59 (14.92) & 29.59 (18.90)\\
 & Verb & 36.92 & 39.72 & 12.75 (14.85) & 30.43 (19.54)\\
 & Other & 5.00 & 3.66 & 19.16 (15.95) & 25.59 (19.54)\\
Other & Adjective & 20.80 & 20.32 & 16.61 (17.37) & 31.66 (19.51)\\
 & Noun & 42.74 & 39.03 & 16.77 (19.41) & 37.28 (25.94)\\
 & Verb & 19.66 & 23.93 & 7.18 (7.57) & 26.14 (19.38)\\
 & Other & 16.81 & 16.71 & 22.72 (16.69) & 30.70 (18.48)\\
Total & Adjective & 19.74 & 14.93 & 16.12 (15.57) & 30.75 (18.37)\\
 & Noun & 55.41 & 57.81 & 16.55 (16.74) & 32.58 (20.09)\\
 & Verb & 22.02 & 24.95 & 9.50 (10.91) & 30.29 (18.24)\\
 & Other & 2.82 & 2.31 & 17.76 (15.83) & 28.45 (16.83)\\
\bottomrule
\addlinespace
\end{tabular}
\begin{tablenotes}[para]
\textit{Note.} Raw words indicate original feature listed, while root words indicated translated feature. 
\end{tablenotes}
\end{threeparttable}
\end{center}
\end{table}

The top cue-feature combinations the reprocessed and new data collection
were then combined with the cue-feature combinations from McRae et al.
(2005) and Vinson and Vigliocco (2008). We included all of their
cue-feature combinations with the cue-feature listed in their
supplemental files with the feature in the raw feature column. If
features could be translated into root words with affixes, the same
procedure as described above was applied. The final file then included
the original dataset, cue, feature, translated feature, frequency of the
original feature, frequency of the translated feature, sample size, and
normalized frequencies for the original and translated feature. This
file includes 69284 cue-original feature combinations, with 48925 from
our dataset, and 24449 of which are cue-translated feature combinations.
Statisics in Tables \ref{tab:feature-table} and \ref{tab:percent-table}
only include information from the reprocessed Buchanan et al. (2013)
norms and the new cues collected for this project.

The final data processing step was to code affixes found on the original
features. A complete affix list translation file can be found online in
our OSF files. Table \ref{tab:affix-table} displays the list of affix
tags, common examples for each type of affix, and the percent of affixes
that fell into each cateory. The percent values are calculated on the
overall affix list, as feature words could have up to three different
affixes. Generally, affixes were tagged in a one-to-one match, however,
special care was taken with numbers and verb tenses. Features like
cat\emph{s} would be coded as a number affix, while features like
walk\emph{s} would be coded as a third person verb. In the final words
file found online, we additionally added forward strength (FSG) and
backward strength (BSG) for investigation into association overlap
(Nelson et al., 2004). The last few columns indicate the word list a
concept was originally normed in to allow for matching to the original
raw files on the OSF page, along with the code for each school and time
point of collection.

\begin{table}[tbp]
\begin{center}
\begin{threeparttable}
\caption{\label{tab:affix-table}Example of Affix Coding and Percent of Affixes Found}
\begin{tabular}{llc}
\toprule
Affix Tag & Example & Percent\\
\midrule
Actions/Processes & ion, ment, ble, ate, ize & 8.21\\
Characteristic & y, ous, nt, ful, ive, wise & 22.72\\
Location & under, sub, mid, inter & 0.44\\
Magnitude & er, est, over, super, extra & 1.31\\
Not & less, dis, un, non, in , im, ab & 2.76\\
Number & s, uni, bi, tri, semi & 28.31\\
Opposites/Wrong & mis, anti, de & 0.13\\
Past Tense & ed & 8.03\\
Person/Object & er, or, men, person, ess, ist & 7.23\\
Present Participle & ing & 14.03\\
Slang & bros, bike, bbq, diff, h2o & 0.12\\
Third Person & s & 6.16\\
Time & fore, pre, post, re & 0.54\\
\bottomrule
\end{tabular}
\end{threeparttable}
\end{center}
\end{table}

This affix processing procedure is a slight departure from our previous
work, as we previously argued to keep some morphologically similar
features separate if they denoted different concepts. For example,
\emph{act} and \emph{actor} were separated because each feature
explained a separate component of the cue word (i.e., noun and gender).
The original processing in Buchanan et al. (2013) combined features that
overlapped in cue sets by 80\%. In this reprocessing and update, we
translated all words to a root form, and coded these translations, thus,
allowing for the exploration of the affect of affixes on semantic
feature overlap. Both forms of the feature are provided for flexibility
in calculating overlap by using the original feature (raw), the
translated feature (root), and the affix overlap by code (affix). Cosine
values were calculated for each of these feature sets by using the
following formula:

\[
\frac{\sum_{i=1}^{n} A_i \times B_i} {\sqrt{\sum_{i=1}^{n} A_i^2} \times \sqrt{\sum_{i=1}^{n} B_i^2}}
\]

This formula is similar to a dot-product correlation, where \(A_i\) and
\(B_i\) indicate the overlapping feature normalized frequency between
cue A and cue B. The \emph{i} subscript denotes the current cue, and
when features match, the frequencies are multiplied together and summed
across all matches (\(\Sigma\)). For the denominator, the feature
normalized frequency is first squared and summed from \emph{i} to
\emph{n} features for cue A and B. The square root of these summation
values is then multiplied together. In essence, the numerator calculates
the overlap of feature frequency for matching features, while the
denominator accounts for the entire feature frequency set for each cue.
Cosine values range from 0 (no overlapping features) to 1 (complete
overlapping features). With nearly five thousand cue words, just under
twenty-five million cue-cue cosine combinations can be calculated. In
the datasets presented online, we only included cue-cue combinations
with a feature overlap of at least two features, in order to reduce the
large quantity of zero and very low cosine values. This procedure
additionally allowed for online presentation of the data, as millions of
cosines were not feasible for our server. The complete feature list,
along with our code to calculate cosine, can be used to obtain values
not presented in our data if desired.

\subsection{Website}\label{website}

In addition to our OSF page, we present a revamped website for this data
at \url{http://www.wordnorms.com/}. The single words page includes
information about each of the cue words including cue set size,
concreteness, word frequency from multiple sources, length, full part of
speech, orthographic/phonographic neighborhood, and number of phonemes,
syllables, and morphemes. These values were taken from Nelson et al.
(2004), Balota et al. (2007), and Brysbaert and New (2009). A definition
of each of these variables is provided along with the minimum, maximum,
mean, and standard deviation of numeric values. The table is programmed
using Shiny apps (Chang, Cheng, Allaire, Xie, \& McPherson, 2017). Shiny
is an \emph{R} package that allows the creation of dynamic graphical
user interfaces for interactive web applications. The advantage to using
Shiny applications is data manipulation and visualization with the
additional bonus of up to date statistics for provided data (i.e., as
typos are fixed or data is updated, the web app will display the most
recent caculations). In addition to the variable table, users can search
and save filtered output using our Shiny search app. With this app, you
can filter for specific variable ranges and save the output in a csv or
Excel file. The complete data is also provided for download.

On the word pairs page, all information about word-pair statistics can
be found. A second variable table is provided with semantic and
associative statistics. This dataset includes the cue and target words
from this project (cue-cue combinations), the root, raw, and affix
cosines described above, as well as the original Buchanan et al. (2013)
cosines. Additional semantic information includes Latent Semantic
Analysis (LSA; Landauer \& Dumais, 1997) and JCN (Jiang \& Conrath,
1997) values provided in the Maki, McKinley, and Thompson (2004) norms,
along with FSG and BSG from the Nelson et al. (2004) norms for
association. The descriptions, minimum, maximum, mean, and standard
deviations of these values are provided in the app. Again, the search
app includes all of these stimuli for cue-cue combinations with two or
more features in common, where you can filter this data for experimental
stimuli creation. The separation of single and word-pair data (as well
as cosine calculation reduction to cues with two features in common) was
practical, as the applications run slowly as a factor of the number of
rows and columns of data. On each page, we link the data, applications,
and source code so that others may use and manipulate our work depending
on their data creation or visualization goals.

\section{Results}\label{results}

An examination of the results of the cue-feature lists indicated that
the new data collected was similar to the previous semantic feature
production norms. As shown in Table \ref{tab:feature-table}, the new
Mechanical Turk data showed roughly the same number of listed features
for each cue concept, usually between five to seven features. Table
\ref{tab:percent-table} portrayed that adjective cues generally included
other adjectives or nouns as features, while noun cues were
predominately described by other nouns. Verb cues included a large
feature list of nouns, but then was equally split between adjectives,
other verbs, and other categories. Lastly, the other cue types generally
elicited nouns and verbs. Normalized percent frequencies were generally
between seven and twenty percent when examining the raw words. These
words included multiple forms, as the percent increased to around thirty
percent when features were translated into their root words. Indeed,
nearly half of the 48925 cue-feature pairs were repeated, as 24449
cue-feature pairs were unique when examining translated features.

36030 affix values were found, which arose from 4407 of the 4437 cue
concepts. 33052 first affixes were found, with 2832 second place
affixes, and 146 third place affixes. Table \ref{tab:affix-table} shows
the distribution of these affix values. Generally, numbers were the
largest category of affixes demonstrating that participants often
indicated the quantity of the feature when describing the cue word. The
second largest affix category was characteristics which often indicated
the switch to or from a noun form of the feature word (i.e.,
\emph{angry} to \emph{anger}). Verb tenses (past tense, present
participle, and third person) comprised a large set of affixes
indicating the type of concept or when a concept might be doing an
action for a cue. Persons and objects were also indicated about 7\% of
the time, while actions and processes of the cue were mentioned about
8\% of the time.

\subsection{Divergent Validity}\label{divergent-validity}

\begin{table}[tbp]
\begin{center}
\begin{threeparttable}
\caption{\label{tab:divergent-table}Percent and Mean Overlap to the Free Association Norms}
\begin{tabular}{lccccc}
\toprule
  & \% Overlap & $M$ FSG & $SD$ FSG & Min & Max\\
\midrule
Adjective & 51.86 & .12 & .15 & .01 & .94\\
Noun & 36.48 & .11 & .14 & .01 & .91\\
Verb & 32.15 & .11 & .13 & .01 & .94\\
Other & 44.44 & .13 & .18 & .01 & .88\\
Total & 37.47 & .11 & .14 & .01 & .94\\
\bottomrule
\end{tabular}
\end{threeparttable}
\end{center}
\end{table}

When collecting semantic feature production norms, there can be a
concern that the information produced will simply mimic the free
association norms, and thus, be a representation of association
(context) rather than semanticity (meaning). Table
\ref{tab:divergent-table} portrays the overlap with the Nelson et al.
(2004) norms. The percent of time a cue-feature combination was present
in the free association norms was calculated, along with the average FSG
for those overlapping pairs. These values were calculated on the
complete dataset with the McRae et al. (2005) and Vinson and Vigliocco
(2008) norms, as we are presenting them as a combined dataset, on the
translated cue-feature set only. The overall overlap between the
database cue-feature sets and the free association cue-target sets was
approximately 37\%, ranging from 32\% for verbs and nearly 52\% for
adjectives. Similiar to our previous results, the range of the FSG was
large (.01 - .94), however, the average FSG was low for overlapping
pairs, \emph{M} = .11 (\emph{SD} = .14). These results indicated that
while it will always be difficult to separate association and meaning,
the dataset presented here represents a low association when examining
overlapping values, and more than 60\% of the data is completely
separate from the free association norms. The limitation to this finding
is the removal of idiosyncratic responses from the Nelson et al. (2004)
norms, but even if these were to be included in some form, the average
FSG would still be quite low when comparing cue-feature lists to
cue-target lists.

\subsection{Convergent Validity}\label{convergent-validity}

To examine the validity of cosine values, we calculated the average
cosine score between the new processing of the data for each of the
three feature production norms used in this project. Overlapping cues in
each of three database sets were found (\emph{n} = 188), and the average
cosine between their feature sets was examined. Buchanan et al. (2013)
and the new dataset are listed the subscript B, while McRae et al.
(2005) is M and V for Vinson and Vigliocco (2008). For root cosine
values, we found high overlap between all three datasets: \(M_{BM}\) =
.67 (\emph{SD} = .14), \(M_{BV}\) = .66 (\emph{SD} = .18), and
\(M_{MV}\) = .72 (\emph{SD} = .11). The raw cosine values also
overlapped well, even though the McRae et al. (2005) and Vinson and
Vigliocco (2008) datasets were already mostly preprocessed for word
stems: \(M_{BM}\) = .55 (\emph{SD} = .15), \(M_{BV}\) = .54 (\emph{SD} =
.20), and \(M_{MV}\) = .45 (\emph{SD} = .19). Last, the affix cosines
overlapped similiarly between BM datasets, \(M_{BM}\) = .43 (\emph{SD} =
.29), but did not overlap with the V datasets: \(M_{BV}\) = .04
(\emph{SD} = .14), and \(M_{MV}\) = .09 (\emph{SD} = .19), likely due to
V dataset preprocessing.

The correlation between root, raw, affix, old cosine, LSA, and JCN were
calculated to examine convergent validity. As shown in Table
\ref{tab:correlation-table}, the intercorrelations between the cosine
measures are high, especially between our previous work and this
dataset. JCN is backwards coded, as zero values indicate close semantic
neighbors (low dictionary distance) and high values indicate low
semantic relation. The small negative correlations replicated previous
findings (Buchanan et al., 2013). LSA values showed small positive
correlations with cosine values, indicating some overlap with thematic
information and semantic feature overlap (Maki \& Buchanan, 2008). These
correlations were slightly different than our previous publication,
likely because we restricted this cosine set to values with at least two
features in common. LSA and JCN correlations were lower than LSA-COS and
JCN-COS, but these values indicated that themes and dictionary distance
were similarly related to feature overlap.

\begin{table}[tbp]
\begin{center}
\begin{threeparttable}
\caption{\label{tab:correlation-table}Correlations between Semantic, Associative, and Thematic Variables}
\begin{tabular}{lcccccccc}
\toprule
  & Root & Raw & Affix & Previous COS & JCN & LSA & FSG & BSG\\
\midrule
Root & 1 &  &  &  &  &  &  & \\
Raw & .93 & 1 &  &  &  &  &  & \\
Affix & .50 & .53 & 1 &  &  &  &  & \\
Previous COS & .94 & .91 & .49 & 1 &  &  &  & \\
JCN & -.18 & -.22 & -.17 & -.22 & 1 &  &  & \\
LSA & .18 & .15 & .10 & .21 & -.06 & 1 &  & \\
FSG & .06 & .04 & .08 & .10 & -.15 & .24 & 1 & \\
BSG & .14 & .15 & .17 & .18 & -.18 & .26 & .31 & 1\\
\bottomrule
\end{tabular}
\end{threeparttable}
\end{center}
\end{table}

\subsection{Relation to Semantic
Priming}\label{relation-to-semantic-priming}

\begin{table}[tbp]
\begin{center}
\begin{threeparttable}
\caption{\label{tab:ldt-table}LDT Response Latencies Correlation with Semantic and Associative Variables}
\begin{tabular}{lcccc}
\toprule
Variable & \multicolumn{1}{c}{FA-LDT 200} & \multicolumn{1}{c}{FA-LDT 1200} & \multicolumn{1}{c}{OA-LDT 200} & \multicolumn{1}{c}{OA-LDT 1200}\\
\midrule
Root COS & .12 & .07 & .09 & .08\\
Raw COS & .12 & .06 & .09 & .06\\
Affix COS & .09 & .07 & .06 & .04\\
Target Root FSS & .00 & -.01 & -.02 & -.02\\
Target Raw FSS & -.00 & -.02 & -.02 & -.02\\
Target CSS & .01 & .01 & -.03 & .02\\
Cue Root FSS & .04 & -.01 & .04 & .02\\
Cue Raw FSS & .06 & -.01 & .03 & .02\\
Cue CSS & .05 & .02 & .07 & .02\\
FSG & .01 & .10 & .05 & .06\\
BSG & .14 & .09 & .09 & .06\\
LSA & .08 & .08 & .13 & .08\\
JCN & -.01 & .01 & -.06 & .01\\
\bottomrule
\addlinespace
\end{tabular}
\begin{tablenotes}[para]
\textit{Note.} Missing values excluded pairwise for JCN. FA: first associate, OA: other associate, CSS: cue set size, and FSS: feature set size.
\end{tablenotes}
\end{threeparttable}
\end{center}
\end{table}

\begin{table}[tbp]
\begin{center}
\begin{threeparttable}
\caption{\label{tab:name-table}Naming Response Latencies Correlation with Semantic and Associative Variables}
\begin{tabular}{lcccc}
\toprule
Variable & \multicolumn{1}{c}{FA-Name 200} & \multicolumn{1}{c}{FA-Name 1200} & \multicolumn{1}{c}{OA-Name 200} & \multicolumn{1}{c}{OA-Name 1200}\\
\midrule
Root COS & -.01 & .09 & .00 & .05\\
Raw COS & -.01 & .10 & .00 & .04\\
Affix COS & -.01 & .06 & .02 & .01\\
Target Root FSS & -.03 & -.05 & .01 & .03\\
Target Raw FSS & -.03 & -.03 & -.00 & .03\\
Target CSS & -.04 & -.04 & .01 & .00\\
Cue Root FSS & -.02 & -.00 & .02 & -.00\\
Cue Raw FSS & .00 & -.01 & .02 & .00\\
Cue CSS & .00 & -.02 & .00 & .02\\
FSG & -.03 & .06 & .04 & .03\\
BSG & .11 & .10 & .11 & .04\\
LSA & .07 & .04 & .06 & .05\\
JCN & -.04 & .00 & -.08 & -.01\\
\bottomrule
\addlinespace
\end{tabular}
\begin{tablenotes}[para]
\textit{Note.} Missing values excluded pairwise for JCN. FA: first associate, OA: other associate, CSS: cue set size, and FSS: feature set size.
\end{tablenotes}
\end{threeparttable}
\end{center}
\end{table}

As a second examination of convergent validity, the correlation between
values calculated from these norms and the \emph{Z} priming values from
the Semantic Priming Project were examined. The Semantic Priming Project
includes lexical decision and naming response latencies for priming at
200 and 1200 ms stimulus onset asychronies (SOA). In these experiments,
participants were shown cue-target words that were either the first
associate of a concept or an other associate (second response or higher
in the Nelson et al. (2004) norms). The response latency of the target
word was subtracted from the non-primed lexical decision or naming time
using the English Lexicon Project as the baseline expected response
latencies for concepts. Therefore, each target item received four (two
SOAs by two tasks lexical decision or naming) priming times, and we
selected the \emph{Z}-scored priming from the dataset to correlate with
our data. In addition to root, raw, and affix cosine, we additionally
calculated feature set size for the cue and target of the primed pairs.
Feature set size is the number of features listed by participants when
creating the norms for that concept. Because of the nature of our norms,
we calculated both feature set size for the raw, untranslated features,
as well as the translated features. The average feature set sizes for
our dataset can be found in Table \ref{tab:feature-table}. The last
variable included was cosine set size which was defined as the number of
other concepts each cue or target was nonzero paired with in the cosine
values. Feature set size indicates the number of features listed for
each cue or target, while cosine set size indicates the number of other
semantically related concepts for each cue or target.

Tables \ref{tab:ldt-table} and \ref{tab:name-table} display the
correlations between the new semantic variables described above, as well
as FSG, BSG, LSA, and JCN for reference. For lexical decision priming,
we found small correlations between the root and raw cosine values and
priming, with the largest for first associates in the 200 ms condition.
The correlations decreased for the 1200 ms condition and the other
associate SOAs. These two variables are highly correlated, therefore, it
is not surprising that they have similiar correlations with priming.
Affix cosine also was slighty related to priming, especially for first
associates in the 200 ms condition. Most of the cue and feature set
sizes were not related to priming, showing correlations close to zero in
most instances. Cue set size for the cue word was somewhat related to
200 ms priming, along with raw cue feature set size. These correlations
are small, but they are comparable or greater than the correlations for
association and other measures of semantic or thematic relatedness. For
naming, the results are less consistent. Cosine values are related to
1200 ms naming in first and other associates, and none of the feature or
cue set sizes showed any relationship with priming. Again, we see that
many of the other associative and semantic variables correspondingly do
not correlate with priming. In both naming and lexical decision priming,
BSG has a small but consistent relationship with priming, which may
indicate the processing of the target back to the cue. LSA was also a
small predictor of priming across conditions.

\section{Discussion}\label{discussion}

This research project focused on expanding the avaliability of English
semantic feature overlap norms, in an effort to provide more coverage of
concepts that occur in other large database projects like the Semantic
Priming and English Lexicon Projects. The number and breadth of
linguistic variables and normed databases has increased over the years,
however, researchers can still be limited by the concept overlap between
them. Projects like the Small World of Words provide newly expanded
datasets for association norms, and our work helps fill the voids for
corresponding semantic norms. To provide the largest dataset of similiar
data, we combined the newly collected data with previous work by using
Buchanan et al. (2013), McRae et al. (2005), and Vinson and Vigliocco
(2008) together. These norms were reprocessed from previous work to
explore the impact of feature coding for feature overlap. As shown in
the correlation between root and raw cosines, the parsing of words to
root form created very similiar results across other variables. This
finding does not imply that these cosine values are the same, as root
cosines were larger than their corresponding raw cosine. It does,
however, imply that the cue-feature coding can produce similiar results
in raw or translated format.

Of particular interest was the information that is often lost when
translating raw features back to a root word. One surprising result in
this study was the sheer number of affixes present on each cue word.
With these values, we believe we have captured some of the nuance that
is often discarded in this type of research. Affix cosines were less
related to their feature root and raw counterparts, but also showed
small correlations with semantic priming. Potentially, affix overlap can
be used to add small, but meaningful predictive value to related
semantic phenomena. Further investigation into the compound prediction
of these variables is warrented to fully explore how these, and other
lexical variables, may be used to understand semantic priming. An
examination of the cosine values from the Semantic Priming Project
cue-target set indicates that these values were low, with many zeros.
This restriction of range could explain the small correlations with
priming, along with the understanding that semantic priming itself can
be exceedingly variable and small across items.

We encourage readers to use the corresponding website associated with
these norms to download the data, explore the Shiny apps, and use the
options provided for controlled experimental stimuli creation. We
previously documented the limitations of feature production norms that
rely on on single word instances as their features (i.e., \emph{four}
and \emph{legs}), rather than combined phrase sets. One limitation,
potentially, is the inability to create fine distinctions between cues;
however, the small feature set sizes imply that the granulation of
features is large, since many distinguishing features are often never
listed in these tasks. For instance, \emph{dogs} are living creatures,
but \emph{has lungs} or \emph{has skin} would usually not be listed
during a feature production task, and thus, feature sets should not be
considered a complete snapshot of mental represenation (Rogers \&
McClelland, 2004). The previous data and other norms were purposely
combined in the recoded format, so that researchers could use the entire
set of avaliable norms which increases comparability across datasets.
Given the strong correlation between databases, we suspect that using
single word features does not reduce their reliability and validity.

One other important limitation of the instructions in this study is that
multiple senses of concepts were not distinguished. We did not wish to
prime participants for specific senses to capture the features for
multiple senses of a concept, however, this procedure could lead to
lower cosine values for concepts that might intuitively seem very
related. The feature production lists could be used to sort senses and
recalculate overlap values, but it is likely that feature information is
correspondingly mixed or sorted into small sublists in memory as well.
The addition of the coded affix information may help capture some of
those sense differences, as well as the some of the spatial and
relational features that are not traditionally captured by simple
feature production. For example, by understanding the numbers or actors
affixes, we may gain more information about semanticity that is often
regarded as something to disregard in data processing.

\newpage

\section{References}\label{references}

\setlength{\parindent}{-0.5in} \setlength{\leftskip}{0.5in}

\hypertarget{refs}{}
\hypertarget{ref-Ashcraft1978a}{}
Ashcraft, M. H. (1978). Property norms for typical and atypical items
from 17 categories: A description and discussion. \emph{Memory \&
Cognition}, \emph{6}(3), 227--232.
doi:\href{https://doi.org/10.3758/BF03197450}{10.3758/BF03197450}

\hypertarget{ref-R-papaja}{}
Aust, F., \& Barth, M. (2018). \emph{papaja: Create APA manuscripts with
R Markdown}. Retrieved from \url{https://github.com/crsh/papaja}

\hypertarget{ref-Balota2007}{}
Balota, D. A., Yap, M. J. M. J., Hutchison, K. A., Cortese, M. J.,
Kessler, B., Loftis, B., \ldots{} Treiman, R. (2007). The English
lexicon project. \emph{Behavior Research Methods}, \emph{39}(3),
445--459.
doi:\href{https://doi.org/10.3758/BF03193014}{10.3758/BF03193014}

\hypertarget{ref-Barsalou2003}{}
Barsalou, L. W. (2003). Abstraction in perceptual symbol systems, (May),
1177--1187.
doi:\href{https://doi.org/10.1098/rstb.2003.1319}{10.1098/rstb.2003.1319}

\hypertarget{ref-Brysbaert2009}{}
Brysbaert, M., \& New, B. (2009). Moving beyond Kučera and Francis: A
critical evaluation of current word frequency norms and the introduction
of a new and improved word frequency measure for American English.
\emph{Behavior Research Methods}, \emph{41}(4), 977--990.
doi:\href{https://doi.org/10.3758/BRM.41.4.977}{10.3758/BRM.41.4.977}

\hypertarget{ref-Brysbaert2013}{}
Brysbaert, M., Warriner, A. B., \& Kuperman, V. (2014). Concreteness
ratings for 40 thousand generally known English word lemmas.
\emph{Behavior Research Methods}, \emph{46}(3), 904--911.
doi:\href{https://doi.org/10.3758/s13428-013-0403-5}{10.3758/s13428-013-0403-5}

\hypertarget{ref-Buchanan2013}{}
Buchanan, E. M., Holmes, J. L., Teasley, M. L., \& Hutchison, K. A.
(2013). English semantic word-pair norms and a searchable Web portal for
experimental stimulus creation. \emph{Behavior Research Methods},
\emph{45}(3), 746--757.
doi:\href{https://doi.org/10.3758/s13428-012-0284-z}{10.3758/s13428-012-0284-z}

\hypertarget{ref-Buchanan2018}{}
Buchanan, E. M., Valentine, K. D., \& Maxwell, N. P. (2018). LAB:
Linguistic Annotated Bibliograpy - A searchable portal for normed
database information. Retrieved from \url{https://osf.io/r6y3n}

\hypertarget{ref-Buhrmester2011}{}
Buhrmester, M., Kwang, T., \& Gosling, S. D. (2011). Amazon's Mechanical
Turk. \emph{Perspectives on Psychological Science}, \emph{6}(1), 3--5.
doi:\href{https://doi.org/10.1177/1745691610393980}{10.1177/1745691610393980}

\hypertarget{ref-R-shiny}{}
Chang, W., Cheng, J., Allaire, J., Xie, Y., \& McPherson, J. (2017).
\emph{Shiny: Web application framework for r}. Retrieved from
\url{https://CRAN.R-project.org/package=shiny}

\hypertarget{ref-Collins1975}{}
Collins, A. M., \& Loftus, E. F. (1975). A spreading-activation theory
of semantic processing. \emph{Psychological Review}, \emph{82}(6),
407--428.
doi:\href{https://doi.org/10.1037/0033-295X.82.6.407}{10.1037/0033-295X.82.6.407}

\hypertarget{ref-Collins1969}{}
Collins, A. M., \& Quillian, M. R. (1969). Retrieval time from semantic
memory. \emph{Journal of Verbal Learning and Verbal Behavior},
\emph{8}(2), 240--247.
doi:\href{https://doi.org/10.1016/S0022-5371(69)80069-1}{10.1016/S0022-5371(69)80069-1}

\hypertarget{ref-Cree2003}{}
Cree, G. S., \& McRae, K. (2003). Analyzing the factors underlying the
structure and computation of the meaning of chipmunk, cherry, chisel,
cheese, and cello (and many other such concrete nouns). \emph{Journal of
Experimental Psychology: General}, \emph{132}(2), 163--201.
doi:\href{https://doi.org/10.1037/0096-3445.132.2.163}{10.1037/0096-3445.132.2.163}

\hypertarget{ref-Cree1999}{}
Cree, G. S., McRae, K., \& McNorgan, C. (1999). An attractor model of
lexical conceptual processing: Simulating semantic priming.
\emph{Cognitive Science}, \emph{23}, 371--414.
doi:\href{https://doi.org/10.1016/S0364-0213(99)00005-1}{10.1016/S0364-0213(99)00005-1}

\hypertarget{ref-Dewhurst1998}{}
Dewhurst, S. a., Hitch, G. J., \& Barry, C. (1998). Separate effects of
word frequency and age of acquisition in recognition and recall.
\emph{Journal of Experimental Psychology: Learning, Memory, and
Cognition}, \emph{24}(2), 284--298.
doi:\href{https://doi.org/10.1037/0278-7393.24.2.284}{10.1037/0278-7393.24.2.284}

\hypertarget{ref-Griffiths2007}{}
Griffiths, T. L., Steyvers, M., \& Tenenbaum, J. B. (2007). Topics in
semantic representation. \emph{Psychological Review}, \emph{114}(2),
211--244.
doi:\href{https://doi.org/10.1037/0033-295X.114.2.211}{10.1037/0033-295X.114.2.211}

\hypertarget{ref-Grondin2009}{}
Grondin, R., Lupker, S. J., \& McRae, K. (2009). Shared features
dominate semantic richness effects for concrete concepts. \emph{Journal
of Memory and Language}, \emph{60}(1), 1--19.
doi:\href{https://doi.org/10.1016/j.jml.2008.09.001}{10.1016/j.jml.2008.09.001}

\hypertarget{ref-Hutchison2013}{}
Hutchison, K. A., Balota, D. A., Neely, J. H., Cortese, M. J.,
Cohen-Shikora, E. R., Tse, C.-S., \ldots{} Buchanan, E. M. (2013). The
semantic priming project. \emph{Behavior Research Methods},
\emph{45}(4), 1099--1114.
doi:\href{https://doi.org/10.3758/s13428-012-0304-z}{10.3758/s13428-012-0304-z}

\hypertarget{ref-Jiang1997}{}
Jiang, J. J., \& Conrath, D. W. (1997). Semantic Similarity Based on
Corpus Statistics and Lexical Taxonomy. \emph{Proceedings of
International Conference Research on Computational Linguistics},
(Rocling X), 19--33.
doi:\href{https://doi.org/10.1.1.269.3598}{10.1.1.269.3598}

\hypertarget{ref-Jones2007}{}
Jones, M. N., \& Mewhort, D. J. K. (2007). Representing word meaning and
order information in a composite holographic lexicon.
\emph{Psychological Review}, \emph{114}(1), 1--37.
doi:\href{https://doi.org/10.1037/0033-295X.114.1.1}{10.1037/0033-295X.114.1.1}

\hypertarget{ref-Keuleers2012}{}
Keuleers, E., Lacey, P., Rastle, K., \& Brysbaert, M. (2012). The
British Lexicon Project: Lexical decision data for 28,730 monosyllabic
and disyllabic English words. \emph{Behavior Research Methods},
\emph{44}(1), 287--304.
doi:\href{https://doi.org/10.3758/s13428-011-0118-4}{10.3758/s13428-011-0118-4}

\hypertarget{ref-Kounios2009}{}
Kounios, J., Green, D. L., Payne, L., Fleck, J. I., Grondin, R., \&
McRae, K. (2009). Semantic richness and the activation of concepts in
semantic memory: evidence from event-related potentials. \emph{Brain
Research}, \emph{1282}, 95--102.
doi:\href{https://doi.org/10.1016/j.brainres.2009.05.092}{10.1016/j.brainres.2009.05.092}

\hypertarget{ref-Kremer2011a}{}
Kremer, G., \& Baroni, M. (2011). A set of semantic norms for German and
Italian. \emph{Behavior Research Methods}, \emph{43}(December 2010),
97--109.
doi:\href{https://doi.org/10.3758/s13428-010-0028-x}{10.3758/s13428-010-0028-x}

\hypertarget{ref-Kuperman2012}{}
Kuperman, V., Stadthagen-Gonzalez, H., \& Brysbaert, M. (2012).
Age-of-acquisition ratings for 30,000 English words. \emph{Behavior
Research Methods}, \emph{44}(4), 978--990.
doi:\href{https://doi.org/10.3758/s13428-012-0210-4}{10.3758/s13428-012-0210-4}

\hypertarget{ref-Landauer1997}{}
Landauer, T. K., \& Dumais, S. T. (1997). A solution to Plato's problem:
The latent semantic analysis theory of acquisition, induction, and
representation of knowledge. \emph{Psychological Review}, \emph{104}(2),
211--240.
doi:\href{https://doi.org/10.1037//0033-295X.104.2.211}{10.1037//0033-295X.104.2.211}

\hypertarget{ref-Lenci2013}{}
Lenci, A., Baroni, M., \& Cazzolli, G. (2013). BLIND : a set of semantic
feature norms from the congenitally blind, 1218--1233.
doi:\href{https://doi.org/10.3758/s13428-013-0323-4}{10.3758/s13428-013-0323-4}

\hypertarget{ref-Maki2008}{}
Maki, W. S., \& Buchanan, E. M. (2008). Latent structure in measures of
associative, semantic, and thematic knowledge. \emph{Psychonomic
Bulletin \& Review}, \emph{15}(3), 598--603.
doi:\href{https://doi.org/10.3758/PBR.15.3.598}{10.3758/PBR.15.3.598}

\hypertarget{ref-Maki2004}{}
Maki, W. S., McKinley, L. N., \& Thompson, A. G. (2004). Semantic
distance norms computed from an electronic dictionary (WordNet).
\emph{Behavior Research Methods, Instruments, \& Computers},
\emph{36}(3), 421--431.
doi:\href{https://doi.org/10.3758/BF03195590}{10.3758/BF03195590}

\hypertarget{ref-McRae2005}{}
McRae, K., Cree, G. S., Seidenberg, M. S., \& McNorgan, C. (2005).
Semantic feature production norms for a large set of living and
nonliving things. \emph{Behavior Research Methods}, \emph{37}(4),
547--559.
doi:\href{https://doi.org/10.3758/BF03192726}{10.3758/BF03192726}

\hypertarget{ref-McRae1997}{}
McRae, K., Sa, V. R. de, \& Seidenberg, M. S. (1997). On the nature and
scope of featural representations of word meaning. \emph{Journal of
Experimental Psychology: General}, \emph{126}(2), 99--130.
doi:\href{https://doi.org/10.1037/0096-3445.126.2.99}{10.1037/0096-3445.126.2.99}

\hypertarget{ref-Medin1989}{}
Medin, D. L. (1989). Concepts and conceptual structure. \emph{American
Psychologist1}, \emph{44}, 1469--1481.
doi:\href{https://doi.org/10.1037/0003-066X.44.12.1469}{10.1037/0003-066X.44.12.1469}

\hypertarget{ref-Montefinese2013}{}
Montefinese, M., Ambrosini, E., Fairfield, B., \& Mammarella, N. (2013).
Semantic memory : A feature-based analysis and new norms for Italian,
440--461.
doi:\href{https://doi.org/10.3758/s13428-012-0263-4}{10.3758/s13428-012-0263-4}

\hypertarget{ref-Montefinese2015}{}
Montefinese, M., Daniele, G., \& Ambrosini, E. (2015). Semantic
similarity between old and new items produces false alarms in
recognition memory. \emph{Psychological Research}, 785--794.
doi:\href{https://doi.org/10.1007/s00426-014-0615-z}{10.1007/s00426-014-0615-z}

\hypertarget{ref-Moss2002}{}
Moss, H. E., Tyler, L. K., \& Devlin, J. T. (2002). The emergence of
category-specific deficits in a distribuited semantic system. In E.
Forde \& G. Humphreys (Eds.), \emph{Category-specificity in mind and
brain} (pp. 115--145). CRC Press.

\hypertarget{ref-Nelson2004}{}
Nelson, D. L., McEvoy, C. L., \& Schreiber, T. A. (2004). The University
of South Florida free association, rhyme, and word fragment norms.
\emph{Behavior Research Methods, Instruments, \& Computers},
\emph{36}(3), 402--407.
doi:\href{https://doi.org/10.3758/BF03195588}{10.3758/BF03195588}

\hypertarget{ref-New2007}{}
New, B., Brysbaert, M., Veronis, J., \& Pallier, C. (2007). The use of
film subtitles to estimate word frequencies. \emph{Applied
Psycholinguistics}, \emph{28}(4), 661--677.
doi:\href{https://doi.org/10.1017/S014271640707035X}{10.1017/S014271640707035X}

\hypertarget{ref-Pexman2003}{}
Pexman, P. M., Holyk, G. G., \& Monfils, M.-H. (2003).
Number-of-features effects and semantic processing. \emph{Memory \&
Cognition}, \emph{31}(6), 842--855.
doi:\href{https://doi.org/10.3758/BF03196439}{10.3758/BF03196439}

\hypertarget{ref-R-base}{}
R Core Team. (2017). \emph{R: A language and environment for statistical
computing}. Vienna, Austria: R Foundation for Statistical Computing.
Retrieved from \url{https://www.R-project.org/}

\hypertarget{ref-Reverberi2004}{}
Reverberi, C., Capitani, E., \& Laiacona, E. (2004). Variabili semantico
lessicali relative a tutti gli elementi di una categoria semantica:
Indagine su soggetti normali italiani per la categoria ``frutta".
\emph{Giornale Italiano Di Psicologia}, \emph{31}, 497--522.

\hypertarget{ref-Riordan2011}{}
Riordan, B., \& Jones, M. N. (2011). Redundancy in perceptual and
linguistic experience: Comparing feature-based and distributional models
of semantic representation. \emph{Topics in Cognitive Science},
\emph{3}(2), 303--345.
doi:\href{https://doi.org/10.1111/j.1756-8765.2010.01111.x}{10.1111/j.1756-8765.2010.01111.x}

\hypertarget{ref-Rogers2004}{}
Rogers, T. T., \& McClelland, J. L. (2004). \emph{Semantic cognition: A
parallel distributed processing approach}. MIT Press.

\hypertarget{ref-Rosch1975}{}
Rosch, E., Mervis, C. B., Rosch, F., \& Mervis, C. B. (1975). Family
resemblances: Studies in the internal structure of categories.
\emph{Cognitive Psychology1}, \emph{7}(4), 573--605.
doi:\href{https://doi.org/10.1016/0010-0285(75)90024-9}{10.1016/0010-0285(75)90024-9}

\hypertarget{ref-Ruts2004}{}
Ruts, W. I. M., Deyne, S. D. E., Ameel, E. E. F., Vanpaemel, W.,
Verbeemen, T., \& Storms, G. (2004). Dutch norm data for 13 semantic
categories and 338 exemplars, \emph{36}(3), 506--515.

\hypertarget{ref-Smith1974}{}
Smith, E. E., Shoben, E. J., \& Rips, L. J. (1974). Structure and
process in semantic memory: A featural model for semantic decisions.
\emph{Psychological Review}, \emph{81}(3), 214.

\hypertarget{ref-Stein2009}{}
Stein, L., \& de Azevedo Gomes, C. (2009). Normas Brasileiras para
listas de palavras associadas: Associação semântica, concretude,
frequência e emocionalidade. \emph{Psicologia: Teoria E Pesquisa},
\emph{25}, 537--546.
doi:\href{https://doi.org/10.1590/S0102-37722009000400009}{10.1590/S0102-37722009000400009}

\hypertarget{ref-Toglia2009}{}
Toglia, M. P. (2009). Withstanding the test of time: The 1978 semantic
word norms. \emph{Behavior Research Methods}, \emph{41}(2), 531--533.
doi:\href{https://doi.org/10.3758/BRM.41.2.531}{10.3758/BRM.41.2.531}

\hypertarget{ref-Toglia1978}{}
Toglia, M. P., \& Battig, W. F. (1978). \emph{Handbook of semantic word
norms}. Hillside, NJ: Earlbaum.

\hypertarget{ref-Vieth2014}{}
Vieth, H. E., Mcmahon, K. L., \& Zubicaray, G. I. D. (2014). The roles
of shared vs . distinctive conceptual features in lexical access,
\emph{5}(September), 1--12.
doi:\href{https://doi.org/10.3389/fpsyg.2014.01014}{10.3389/fpsyg.2014.01014}

\hypertarget{ref-Vigliocco2005}{}
Vigliocco, G., Vinson, D. P., \& Siri, S. (2005). Semantic and
grammatical class effects in naming actions. \emph{Cognition},
\emph{94}, 91--100.
doi:\href{https://doi.org/10.1016/j.cognition.2004.06.004}{10.1016/j.cognition.2004.06.004}

\hypertarget{ref-Vigliocco2002}{}
Vigliocco, G., Vinson, D. P., Damian, M. M. F., \& Levelt, W. (2002).
Semantic distance effects on object and action naming. \emph{Cognition},
\emph{85}, 61--69.
doi:\href{https://doi.org/10.1016/S0010-0277(02)00107-5}{10.1016/S0010-0277(02)00107-5}

\hypertarget{ref-Vigliocco2004}{}
Vigliocco, G., Vinson, D. P., Lewis, W., \& Garrett, M. F. (2004).
Representing the meanings of object and action words: The featural and
unitary semantic space hypothesis. \emph{Cognitive Psychology},
\emph{48}(4), 422--488.
doi:\href{https://doi.org/10.1016/j.cogpsych.2003.09.001}{10.1016/j.cogpsych.2003.09.001}

\hypertarget{ref-Vinson2002}{}
Vinson, D. P., \& Vigliocco, G. (2002). A semantic analysis of noun-verb
dissociations in aphasia. \emph{Journal of Neurolinguistics}, \emph{15},
317--351.
doi:\href{https://doi.org/10.1016/S0911-6044(01)00037-9}{10.1016/S0911-6044(01)00037-9}

\hypertarget{ref-Vinson2008}{}
Vinson, D. P., \& Vigliocco, G. (2008). Semantic feature production
norms for a large set of objects and events. \emph{Behavior Research
Methods}, \emph{40}(1), 183--190.
doi:\href{https://doi.org/10.3758/BRM.40.1.183}{10.3758/BRM.40.1.183}

\hypertarget{ref-Vinson2003}{}
Vinson, D. P., Vigliocco, G., Cappa, S., \& Siri, S. (2003). The
breakdown of semantic knowledge: Insights from a statistical model of
meaning representation. \emph{Brain and Language}, \emph{86}(3),
347--365.
doi:\href{https://doi.org/10.1016/S0093-934X(03)00144-5}{10.1016/S0093-934X(03)00144-5}

\hypertarget{ref-Vivas2017}{}
Vivas, J., Vivas, L., Comesaña, A., \& Coni, A. G. (2017). Spanish
semantic feature production norms for 400 concrete concepts, 1095--1106.
doi:\href{https://doi.org/10.3758/s13428-016-0777-2}{10.3758/s13428-016-0777-2}

\hypertarget{ref-Warriner2013}{}
Warriner, A. B., Kuperman, V., \& Brysbaert, M. (2013). Norms of
valence, arousal, and dominance for 13,915 English lemmas.
\emph{Behavior Research Methods}, \emph{45}(4), 1191--1207.
doi:\href{https://doi.org/10.3758/s13428-012-0314-x}{10.3758/s13428-012-0314-x}

\hypertarget{ref-Yap2016}{}
Yap, M. J., \& Pexman, P. M. (2016). Semantic Richness Effects in
Syntactic Classification: The Role of Feedback. \emph{Frontiers in
Psychology}, \emph{7}, 1394.
doi:\href{https://doi.org/10.3389/fpsyg.2016.01394}{10.3389/fpsyg.2016.01394}

\hypertarget{ref-Yap2015}{}
Yap, M. J., Lim, G. Y., \& Pexman, P. M. (2015). Semantic richness
effects in lexical decision : The role of feedback, (July), 1148--1167.
doi:\href{https://doi.org/10.3758/s13421-015-0536-0}{10.3758/s13421-015-0536-0}






\end{document}
