\documentclass[english,,man]{apa6}
\usepackage{lmodern}
\usepackage{amssymb,amsmath}
\usepackage{ifxetex,ifluatex}
\usepackage{fixltx2e} % provides \textsubscript
\ifnum 0\ifxetex 1\fi\ifluatex 1\fi=0 % if pdftex
  \usepackage[T1]{fontenc}
  \usepackage[utf8]{inputenc}
\else % if luatex or xelatex
  \ifxetex
    \usepackage{mathspec}
  \else
    \usepackage{fontspec}
  \fi
  \defaultfontfeatures{Ligatures=TeX,Scale=MatchLowercase}
\fi
% use upquote if available, for straight quotes in verbatim environments
\IfFileExists{upquote.sty}{\usepackage{upquote}}{}
% use microtype if available
\IfFileExists{microtype.sty}{%
\usepackage{microtype}
\UseMicrotypeSet[protrusion]{basicmath} % disable protrusion for tt fonts
}{}
\usepackage{hyperref}
\hypersetup{unicode=true,
            pdftitle={English Semantic Feature Production Norms: An Extended Database of 4,436 Concepts},
            pdfauthor={Erin M. Buchanan, K. D. Valentine, \& Nicholas P. Maxwell},
            pdfkeywords={semantics, word norms, database, psycholinguistics},
            pdfborder={0 0 0},
            breaklinks=true}
\urlstyle{same}  % don't use monospace font for urls
\ifnum 0\ifxetex 1\fi\ifluatex 1\fi=0 % if pdftex
  \usepackage[shorthands=off,main=english]{babel}
\else
  \usepackage{polyglossia}
  \setmainlanguage[]{english}
\fi
\usepackage{graphicx,grffile}
\makeatletter
\def\maxwidth{\ifdim\Gin@nat@width>\linewidth\linewidth\else\Gin@nat@width\fi}
\def\maxheight{\ifdim\Gin@nat@height>\textheight\textheight\else\Gin@nat@height\fi}
\makeatother
% Scale images if necessary, so that they will not overflow the page
% margins by default, and it is still possible to overwrite the defaults
% using explicit options in \includegraphics[width, height, ...]{}
\setkeys{Gin}{width=\maxwidth,height=\maxheight,keepaspectratio}
\IfFileExists{parskip.sty}{%
\usepackage{parskip}
}{% else
\setlength{\parindent}{0pt}
\setlength{\parskip}{6pt plus 2pt minus 1pt}
}
\setlength{\emergencystretch}{3em}  % prevent overfull lines
\providecommand{\tightlist}{%
  \setlength{\itemsep}{0pt}\setlength{\parskip}{0pt}}
\setcounter{secnumdepth}{0}
% Redefines (sub)paragraphs to behave more like sections
\ifx\paragraph\undefined\else
\let\oldparagraph\paragraph
\renewcommand{\paragraph}[1]{\oldparagraph{#1}\mbox{}}
\fi
\ifx\subparagraph\undefined\else
\let\oldsubparagraph\subparagraph
\renewcommand{\subparagraph}[1]{\oldsubparagraph{#1}\mbox{}}
\fi

%%% Use protect on footnotes to avoid problems with footnotes in titles
\let\rmarkdownfootnote\footnote%
\def\footnote{\protect\rmarkdownfootnote}


  \title{English Semantic Feature Production Norms: An Extended Database of 4,436
Concepts}
    \author{Erin M. Buchanan\textsuperscript{1}, K. D. Valentine\textsuperscript{2},
\& Nicholas P. Maxwell\textsuperscript{1}}
    \date{}
  
\shorttitle{Semantic Norms}
\affiliation{
\vspace{0.5cm}
\textsuperscript{1} Missouri State University\\\textsuperscript{2} University of Missouri}
\keywords{semantics, word norms, database, psycholinguistics}
\usepackage{csquotes}
\usepackage{upgreek}
\captionsetup{font=singlespacing,justification=justified}

\usepackage{longtable}
\usepackage{lscape}
\usepackage{multirow}
\usepackage{tabularx}
\usepackage[flushleft]{threeparttable}
\usepackage{threeparttablex}

\newenvironment{lltable}{\begin{landscape}\begin{center}\begin{ThreePartTable}}{\end{ThreePartTable}\end{center}\end{landscape}}

\makeatletter
\newcommand\LastLTentrywidth{1em}
\newlength\longtablewidth
\setlength{\longtablewidth}{1in}
\newcommand{\getlongtablewidth}{\begingroup \ifcsname LT@\roman{LT@tables}\endcsname \global\longtablewidth=0pt \renewcommand{\LT@entry}[2]{\global\advance\longtablewidth by ##2\relax\gdef\LastLTentrywidth{##2}}\@nameuse{LT@\roman{LT@tables}} \fi \endgroup}


\DeclareDelayedFloatFlavor{ThreePartTable}{table}
\DeclareDelayedFloatFlavor{lltable}{table}
\DeclareDelayedFloatFlavor*{longtable}{table}
\makeatletter
\renewcommand{\efloat@iwrite}[1]{\immediate\expandafter\protected@write\csname efloat@post#1\endcsname{}}
\makeatother
\usepackage{lineno}

\linenumbers

\authornote{Erin M. Buchanan is an Associate Professor of
Quantitative Psychology at Missouri State University. K. D. Valentine is
a Ph.D.~candidate at the University of Missouri. Nicholas P. Maxwell is
a Masters' candidate at Missouri State University.

We would like to thank Keith Hutchison and David Balota for their
contributions to this project, including the funds to secure Mechanical
Turk participants. Additionally, we thank Simon De Deyne and an
anonymous reviewer for their comments on this manuscript.

Correspondence concerning this article should be addressed to Erin M.
Buchanan, 901 S. National Ave, Springfield, MO 65897. E-mail:
\href{mailto:erinbuchanan@missouristate.edu}{\nolinkurl{erinbuchanan@missouristate.edu}}}

\abstract{
A limiting factor in understanding memory and language is often the
availability of large numbers of stimuli to use and explore in
experimental studies. In this study, we expand on three previous
databases of concepts to over 4,000 words including nouns, verbs,
adjectives, and other parts of speech. Participants in the study were
asked to provide lists of features for each concept presented (a
semantic feature production task), which were combined with previous
research in this area. These feature lists for each concept were then
coded into their root word form and affixes (i.e., \emph{cat} and
\emph{s} for \emph{cats}) to explore the impact of word form on semantic
similarity measures, which are often calculated by comparing concept
feature lists (feature overlap). All concept features, coding, and
calculated similarity information is provided in a searchable database
for easy access and utilization for future researchers when designing
experiments that use word stimuli. The final database of word pairs was
combined with the Semantic Priming Project to examine the relation of
semantic similarity statistics on semantic priming in tandem with other
psycholinguistic variables.


}

\usepackage{amsthm}
\newtheorem{theorem}{Theorem}[section]
\newtheorem{lemma}{Lemma}[section]
\theoremstyle{definition}
\newtheorem{definition}{Definition}[section]
\newtheorem{corollary}{Corollary}[section]
\newtheorem{proposition}{Proposition}[section]
\theoremstyle{definition}
\newtheorem{example}{Example}[section]
\theoremstyle{definition}
\newtheorem{exercise}{Exercise}[section]
\theoremstyle{remark}
\newtheorem*{remark}{Remark}
\newtheorem*{solution}{Solution}
\begin{document}
\maketitle

Semantic features are the focus of a large area of research which tries
to delineate the semantic representation of a concept. These features
are key to models of semantic memory (Collins \& Loftus, 1975; i.e.,
memory for facts; Collins \& Quillian, 1969), and they have been used to
create both feature based (Cree \& McRae, 2003; Smith, Shoben, \& Rips,
1974; Vigliocco, Vinson, Lewis, \& Garrett, 2004) and distributional
based models (Griffiths, Steyvers, \& Tenenbaum, 2007; Jones \& Mewhort,
2007; Riordan \& Jones, 2011). Feature based models indicate that the
degree of similarity between concepts is defined by their overlapping
feature lists, while distributional based models posit that similarity
is defined by the overlap between linguistic network or context. To
create feature based similarity, participants were often asked to create
lists of properties for categories of words. This property listing was a
seminal task with corresponding norms that have been prevalent in the
literature (Ashcraft, 1978; Rosch \& Mervis, 1975; Toglia, 2009; Toglia
\& Battig, 1978). Feature production norms are created by soliciting
participants to list properties or features of a target concept without
focusing on category. These features are then compiled into feature sets
that are thought to represent the memory representation of a particular
concept, especially in early feature based models of memory (Collins \&
Loftus, 1975; Collins \& Quillian, 1969; Jones, Willits, \& Dennis,
2015; McRae \& Jones, 2013).

For example, when queried on what features define a \emph{cat},
participants may list \emph{tail}, \emph{animal}, and \emph{pet}. These
features capture the most common types of descriptions: \enquote{is a}
and \enquote{has a}. Additionally, feature descriptions may include
uses, locations, behavior, and gender (i.e., \emph{actor} denotes both a
person and gender). The goal of these norms is often to create a set of
high-probability features, as there can and will be many idiosyncratic
features listed in this task, to explore the nature of concept
structure. In the classic view of category structure, concepts have
defining features or properties, while the probabilistic view suggests
that categories are fuzzy with features that are typical of a concept
(Medin, 1989). These norms have now been published in Italian
(Montefinese, Ambrosini, Fairfield, \& Mammarella, 2013; Reverberi,
Capitani, \& Laiacona, 2004), German (and Italian, Kremer \& Baroni,
2011), Portuguese (Stein \& de Azevedo Gomes, 2009), Spanish (Vivas,
Vivas, Comesaña, Coni, \& Vorano, 2017), and Dutch (Ruts et al., 2004),
as well as for the blind (Lenci, Baroni, Cazzolli, \& Marotta, 2013).

Previous work on semantic feature production norms in English includes
databases by McRae, Cree, Seidenberg, and McNorgan (2005), Vinson and
Vigliocco (2008), Buchanan, Holmes, Teasley, and Hutchison (2013), and
Devereux, Tyler, Geertzen, and Randall (2014). McRae et al. (2005)'s
feature production norms focused on 541 nouns, specifically living and
nonliving objects. Vinson and Vigliocco (2008) expanded the stimuli set
by contributing norms for 456 concepts that included both nouns and
verbs. Buchanan et al. (2013) broadened to concepts other than nouns and
verbs with 1808 concepts normed. The Devereux et al. (2014) norms
included a replication of McRae et al. (2005)'s concepts with the
addition of several hundren more concrete concepts. The current paper
represents over two thousand new concepts added to these previous
projects and a reanalysis of the original data.

Creation of norms is vital to provide investigators with concepts that
can be used in future research. The concepts presented in the feature
production norming task are usually called \emph{cues}, and the
responses to the cue are called \emph{features}. In a semantic priming
task, the concept paired with a cue (first word) is denoted as a
\emph{target} (second word). In a lexical decision task, participants
are shown cue words before a related or unrelated target word. Their
task is to decide if the target word is a word or nonword as quickly as
possible. A similar task, naming, involves reading the second target
word aloud after viewing a related or unrelated cue word. Semantic
priming occurs when the target word is recognized (responded to or read
aloud) faster after the related cue word in comparison to the unrelated
cue word (Moss et al., 1995). The feature list data created from the
production task can be used to determine the strength of the relation
between cue and target word, often by calculating the feature overlap,
or number of shared features between concepts (McRae et al., 2005). Both
the cue-feature lists and the cue-cue combinations (i.e., the relation
between two cues in a feature production dataset, which becomes a
cue-target combination in the priming task) are useful and important
data for researchers in exploring various semantic based phenomena.

These feature category lists can provide insight into the probabilistic
nature of language and conceptual structure (Cree \& McRae, 2003; McRae,
Sa, \& Seidenberg, 1997; Moss, Tyler, \& Devlin, 2002; Pexman, Holyk, \&
Monfils, 2003). Additionally, the feature production norms can be used
as the underlying data to create models of semantic priming and
cognition focusing on cue-target relation (Cree, McRae, \& McNorgan,
1999; Rogers \& McClelland, 2004; Vigliocco et al., 2004). When using
database norms to select for stimuli, others have studied semantic
word-picture interference (i.e., slower naming times when distractor
words are related category concepts in a picture naming task; Vieth,
Mcmahon, \& Zubicaray, 2014), recognition memory (Montefinese, Zannino,
\& Ambrosini, 2015), and semantic richness, which is a measure of shared
defining features (Grondin, Lupker, \& McRae, 2009; Kounios et al.,
2009; Yap \& Pexman, 2016; Yap, Lim, \& Pexman, 2015). The Vinson and
Vigliocco labs have shown the power of turning in-house data projects
into a larger norming set (Vinson \& Vigliocco, 2008), as they published
papers on aphasia (i.e., the loss of understanding speech; Vinson \&
Vigliocco, 2002; Vinson, Vigliocco, Cappa, \& Siri, 2003),
meaning-syntactic differences (Vigliocco, Vinson, \& Siri, 2005; i.e.,
differences in naming times based on semantic or syntactic similarity;
Vigliocco, Vinson, Damian, \& Levelt, 2002), and representational models
(Vigliocco et al., 2004).

However, it would be unwise to consider these norms as an exact
representation of a concept in memory (McRae et al., 2005). These norms
represent salient features that participants can recall, likely because
saliency is considered special to our understanding of concepts (Cree \&
McRae, 2003). Additionally, Barsalou (2003) suggested that participants
are likely creating a mental model of the concept based on experience
and using that model to create a feature property list. This model may
represent a specific instance of a category (i.e., their pet dog), and
feature lists will represent that particular memory. One potential
solution to overcome saliency effects would be to solicit applicability
ratings for features across multiple exemplars (i.e., specific members)
of a category, as De Deyne et al. (2008) have shown that this procedure
provides reliable ratings across exemplars and provides more connections
than the sparse representations that can occur when producing features.

Computational modeling of memory requires sufficiently large datasets to
accurately portray semantic memory, therefore, the advantage of big data
in psycholinguistics cannot be understated. There are many large corpora
that could be used for exploring the structure of language and memory
through frequency (see the SUBTLEX projects Brysbaert \& New, 2009; New,
Brysbaert, Veronis, \& Pallier, 2007). Additionally, there are large
lexicon projects that explore how the basic features of words affect
semantic priming, such as orthographic neighborhood (words that are one
letter different from the concept), length, and part of speech (Balota
et al., 2007; Keuleers, Lacey, Rastle, \& Brysbaert, 2012). In contrast
to these basic linguistic features of words, other norming efforts have
involved subjective ratings of concepts. Large databases of age of
acquisition (i.e., rated age of learning the concept; Kuperman,
Stadthagen-Gonzalez, \& Brysbaert, 2012), concreteness (i.e., rating of
how perceptible a concept is; Brysbaert, Warriner, \& Kuperman, 2014),
and valence (i.e., rating of emotion in a concept; Warriner, Kuperman,
\& Brysbaert, 2013) provide further avenues for understanding the impact
these rated properties have on semantic memory. For example, age of
acquisition and concreteness ratings have been shown to predict
performance on recall tasks (Brysbaert et al., 2014; Dewhurst, Hitch, \&
Barry, 1998), while valence ratings are useful for gauging the effects
of emotion on meaning (Warriner et al., 2013). These projects represent
a small subset of the larger normed stimuli available (Buchanan,
Valentine, \& Maxwell, 2018), however, research is still limited by the
overlap between these datasets. If a researcher wishes to control for
lexical characteristics and subjective rating variables, the inclusion
of each new variable to the study will further restrict the item pool
for study. Large, overlapping datasets are crucial for exploring the
entire range of an effect, and to ensure that the stimuli set is not the
only contributing factor to the results of a study.

Therefore, the purpose of this study was to expand the number of cue and
feature word stimuli available, which additionally increases the
possible cue-target pairings for studies using word-pair stimuli (like
semantic priming tasks). To accomplish these goals, we have expanded our
original semantic feature production norms (Buchanan et al., 2013) to
include all cues and targets from The Semantic Priming Project
(Hutchison et al., 2013). The existing norms were reprocessed along with
these new norms to provide new feature coding and affixes (i.e., word
addition that modifies meaning, such as \emph{pre} or \emph{ing}) to
explore the impact of word form. Previously, Buchanan et al. (2013)
illustrated convergent validity with McRae et al. (2005) and Vinson and
Vigliocco (2008) with a difference in approach to processing feature
production data. In McRae et al. (2005) and Vinson and Vigliocco (2008),
features were coded with complexity, matching the \enquote{is a} and
\enquote{has a} format that was first found in Collins and Quillian
(1969) and Collins and Loftus (1975) models. Buchanan et al. (2013) took
a count based approach, wherein each feature is treated as a separate
concept (i.e. \emph{four legs} would be treated as two features, rather
than one complex feature). Both approaches allow for the computation of
similiarity by comparing feature lists for cue words, however, the count
based approach matches popular computational models, such as Latent
Semantic Analysis (Landauer \& Dumais, 1997) and Hyperspace Analogue to
Language (Lund \& Burgess, 1996). These models treat each word in a
document or text as a cue word and similarity is computed by assessing a
matrix of frequency counts between concepts and texts, which is similar
to comparing overlapping feature lists.

In the previous study, each feature was converted to common form if they
denoted the same concept (i.e., most features were translated to their
root form). This process often occurs to help capture the essential
features without increasing the sparsity of the matrix (i.e., the matrix
only contains one representation for \emph{beauty}, rather than several
for all word forms, thus lessening the number of empty cells in a
cue-feature matrix). However, we previously included a few exceptions to
this coding system, such as \emph{act} and \emph{actor} when the
differences in features denoted a change of action (noun/verb) or gender
or cue sets did not overlap (i.e., features like \emph{will} and
\emph{willing} did not have overlapping associated cues). These
exceptions were designed to capture how changes in morphology might be
important cues to word meaning, as hybrid models of word identification
have outlined that morpheme processing can be complex (Caramazza,
Laudanna, \& Romani, 1988; Marslen-Wilson, Tyler, Waksler, \& Older,
1994). Hybrid models include both a compositional view (i.e., words are
first broken down into their components \emph{cat} and \emph{s};
Jarvella \& Meijers, 1983; MacKay, 1978) and a full-listing view (i.e.,
each word form is represented completely separately, \emph{cat} and
\emph{cats} and processing occurs as a race between each type of
representation, Bradley, 1980; Butterworth, 1983). Given these models
and sparsity considerations, we created a coding system to capture the
feature word meaning, in addition to morphology, to provide different
levels of information about each cue-feature combination.

The entire dataset is available on our website
(\url{http://wordnorms.com/}) which has been revamped with a new
interface and web applications to easily find and select stimuli for
future experiments. The data collection, (re)processing, website, and
finalized dataset are detailed below. The basic properties of the
cue-feature data will be detailed, such as the average number of
features each cue elicited across parts of speech and datasets. The
cue-feature data will be explored for divergent validity from the free
association norms to show evidence that the new feature production norms
provide additional information not found in the Nelson, McEvoy, and
Schreiber (2004) dataset. We then provide details on how to calculate
semantic similarity and then use these values to portray convergent
validity by correlating multiple measures of semanticity. Additionally,
the similarity measures are compared to the priming times from the
Semantic Priming Project (Hutchison et al., 2013) to demonstrate the
relation between semantic similarity and priming.

\section{Method}\label{method}

\subsection{Participants}\label{participants}

Participants in the new stimuli set were recruited from Amazon's
Mechanical Turk, which is a large, diverse participant pool wherein
users can complete surveys for small sums of money (Buhrmester, Kwang,
\& Gosling, 2011). Answers can be screened for errors, and incorrect or
incomplete surveys can be rejected or discarded without payment. Each
participant was paid five cents for a survey, and they could complete
multiple Human Intelligence Tasks or HITS. Participants were required to
be located in the United States with a HIT approval rate of at least
80\%, and no other special qualifications were required. HITS would
remain active until \emph{n} = 30 valid survey answers were obtained.
Table \ref{tab:part-table} includes the sample sizes from the new study
(Mechanical Turk 2), as well as the sample sizes from the previous
study, as described in Buchanan et al. (2013).

\begin{table}[tbp]
\begin{center}
\begin{threeparttable}
\caption{\label{tab:part-table}Sample Size and Concept Norming Size for Each Data Collection Location/Time Point}
\begin{tabular}{lccc}
\toprule
Institution & Total Participants & Concepts & Mean $N$\\
\midrule
University of Mississippi & 749 & 658 & 67.8\\
Missouri State University & 1420 & 720 & 71.4\\
Montana State University & 127 & 120 & 63.5\\
Mechanical Turk 1 & 571 & 310 & 60\\
Mechanical Turk 2 & 198 & 1914 & 30\\
\bottomrule
\end{tabular}
\end{threeparttable}
\end{center}
\end{table}

\subsection{Materials}\label{materials}

A main purpose of this second norming set was to expand the Buchanan et
al. (2013) norms to include all concepts from the Semantic Priming
Project (Hutchison et al., 2013). The original concept set was selected
primarily from the Nelson et al. (2004) database, with small overlaps in
the McRae et al. (2005) and Vinson and Vigliocco (2008) database sets
for convergent validity. In the Semantic Priming Project, cue-target
pairs were shown to participants to examine naming (i.e., reading a
concept aloud) and lexical decision (i.e., responding if a presented
string is a word or nonword) response latency priming across related and
unrelated pairs. The related pairs included first associate (most common
response to a cue, \emph{sum}-\emph{add}) and other associates (second
or greater common responses to cues, \emph{safe}-\emph{protect}) as
their target words. The Buchanan et al. (2013) publication of concepts
included many of the cue words from the Semantic Priming Project, while
this project expanded to include unnormed cue words and all target words
for all first and other associate pairs. The addition of these concepts
allowed for complete overlap between the Semantic Priming Project and
feature production norms. As mentioned earlier, the McRae et al. (2005)
norms consist primarily of nouns, the Vinson and Vigliocco (2008)
dataset includes nouns and verbs, while the Buchanan et al. (2013)
included all word forms.

Concepts were labeled by part of speech using the English Lexicon
Project (Balota et al., 2007), the free association norms, and Google's
define search when necessary. When labeling these words, we used the
most common part of speech to categorize concepts. This choice was
predominately for simplicity of categorization, however, the
participants were shown concepts without the suggestion of which sense
to use for the word. Therefore, multiple senses (i.e., \emph{bat} is
noun and a verb) are embedded into the feature production norms, while
the database is labeled with single parts of speech. The other parts of
speech can be found in the English Lexicon Project or multiple other
databases. This dataset was combined with McRae et al. (2005) and Vinson
and Vigliocco (2008) feature production norms, which resulted in a
combined total of 4437 concepts. 70.4\% of concepts were nouns, 14.9\%
adjectives, 12.4\% verbs, and 2.3\% were other forms of speech, such as
adverbs and conjunctions. The new concepts from this norming set only
constituted: \emph{n} = 1916 concepts, 72.0 nouns, 14.9\% adjectives,
12.4\% verbs, and 2.3\% other parts of speech.

\subsection{Procedure}\label{procedure}

Each HIT was kept to five concepts, and usual survey response times were
between five to seven minutes. Each HIT was open until thirty
participants had successfully completed the HIT and were paid the five
cents for the HIT. HITS were usually rejected if they included copied
definitions from Wikipedia, \enquote{I don't know}, or the participant
wrote a paragraph about the concept. These answers were discarded, as
described below. The survey instructions were copied from McRae et al.
(2005)'s Appendix B, which were also used in the previous publication of
these norms. Because the McRae et al. (2005) data was collected on
paper, we modified these instructions slightly. The original lines to
write in responses were changed to an online text box response window.
The detailed instructions additionally no longer contained information
about how a participant should only consider the noun of the target
concept, as the words in our study included multiple forms of speech and
senses. Participants were encouraged to list the properties or features
of each concept in the following areas: physical (looks, sounds, and
feels), functional (uses), and categorical (belongings). The exact
instructions were as follows:

*``We want to know how people read words for meaning. Please fill in
features of the word that you can think of. Examples of different types
of features would be: how it looks, sounds, smells, feels, or tastes;
what it is made of; what it is used for; and where it comes from. Here
is an example:

duck: is a bird, is an animal, waddles, flies, migrates, lays eggs,
quacks, swims, has wings, has a beak, has webbed feet, has feathers,
lives in ponds, lives in water, hunted by people, is edible

Complete this questionnaire reasonably quickly, but try to list at least
a few properties for each word. Thank you very much for completing this
questionnaire.``*

Participants signed up for the HITS through Amazon's Mechanical Turk
website and completed the study within the Mechanical Turk framework.
Approved HITs were compensated through the Mechanical Turk system. All
answers were then combined into a larger dataset.

\subsection{Data Processing}\label{data-processing}

The entire dataset, at each processing stage described here, can be
found at: \url{https://osf.io/cjyzw/}. On our OSF page, we have included
a detailed processing guide on how concepts were (re)examined for this
publication. This paper was written with \emph{R} markdown (R Core Team,
2017) and \emph{papaja} (Aust \& Barth, 2018). The markdown document
allows an interested reader to view the scripts that created the article
in line with the written text. However, the processing of the text
documents was performed on the raw files, and therefore, we have
included the processing guide for transparency of each stage.

First, each concept was separated into an individual text file that is
included as the \enquote{raw} data online. Each of these files was then
spell checked and corrected when the participant answer was obviously a
typo. As noted earlier, participants often tried to cut and paste
Wikipedia or other online dictionary sources into the their answers to
complete surveys quickly with minimal effort. These entries were easily
found because the formatting of the webpage was included in their
answer. For example, the Wikipedia entry for \emph{zerba} includes the
phonetic spelling of the word, a set of paragraphs about zebras, a table
of contents, and then sectioned paragraphs matching that table of
contents. To find this data, lab members would open the raw text files
that were compiled for each cue, look for these large blocks of
formatted text, and delete that information. Approximately 113 HITS were
rejected because of poor data, and 4524 HITS were paid. Therefore, we
estimate approximately 2\% of the HITS included Wikipedia articles or
other ineligible entries. Next, each concept was processed for feature
frequency. In this stage, the raw frequency counts of each cue-feature
combination were calculated and put together into one large file.
Cue-cue combinations were discarded, as participants might write
\enquote{a zebra is a horse} when asked to define \emph{zebra}. English
stop words such as \emph{the, an, of} were then discarded, as well as
terms that were often used as part of a definition (\emph{like, means,
describes}).

We then created a \enquote{translated} column for each feature listed.
This column indicated the root word for each feature, and additional
columns were added with the affixes that were used in the original
feature. For example, the original feature \emph{cats} would be
translated to \emph{cat} and \emph{s}, wherein \emph{cat} would be the
translated feature and the \emph{s} would be the affix code. The
translation was first started by using a Snowball type stemmer (Porter,
2001), written in Python by a colleague of the first author. All
original features and their roots from this process were then put into
an Excel document, which was reviewed by the first author for
consistency and concepts with affixes that were not stemmed. Usually the
noun version of the feature would be used for the translation or the
most common part of speech for each feature.

At this stage, the data was reduced to cue-feature combinations that
were listed by at least 16\% of participants (matching McRae et al.
(2005)'s procedure) or were in the top five features listed for that
cue. This calculation was performed on the feature percent for the root
word (the \emph{translated} column). For example, \emph{beauty} may have
been listed as \emph{beauty, beautiful, beautifully, beautifulness}, and
this feature would have been listed four times in the dataset for the
original cue (original feature in the \emph{feature} column).

The sample size for the cue was added to this dataset, as the sample
sizes varied across experiment time, as shown in Table
\ref{tab:part-table}. Therefore, instead of using raw feature frequency,
we normalized each count into the percent of participants that included
that feature with each cue. The \emph{frequency\_feature} column
indicates the frequency of the original, unedited feature, while the
\emph{frequency\_translated} includes all combinations of \emph{beauty}
into one overall feature. Because non-nouns can be more difficult to
create a feature list for, we included the top five descriptors in
addition to the 16\% listed criteria, to ensure that each concept
included at least five features. Table \ref{tab:feature-table} indicates
the average number of cue-feature pairs found for each data collection
site/time point and part of speech for the cue word.

The parts of speech for the cue, original feature, and translated
feature were merged with this file as described above. Table
\ref{tab:percent-table} depicts the pattern of feature responses for
cue-feature part of speech combinations. This table includes the percent
of features listed for each cue-feature part of speech combination
(i.e., what is the percent of time that both the cue and feature were
both adjectives) for the original feature (raw) and translated feature
(root). Next, the average frequency percent was calculated along with
their standard deviations. These columns indicate the percent that a
cue-feature part of speech combination was listed across participants
(i.e., what is the average percent of participants that listed an
adjective feature for an adjective cue). These two types of calculation
describe the likelihood of seeing part of speech combinations across the
concepts, along with the likelihood of those cue-feature part of speech
combinations across participants. Statistics in Table
\ref{tab:percent-table} only include information from the reprocessed
Buchanan et al. (2013) norms and the new cues collected for this
project.

\begin{table}[tbp]
\begin{center}
\begin{threeparttable}
\caption{\label{tab:feature-table}Average (SD) Cue-Feature Pairs by Location/Time Point}
\begin{tabular}{lccccc}
\toprule
Institution & Adjective & Noun & Verb & Other & Total\\
\midrule
University of Mississippi & 5.57 (1.53) & 7.35 (4.05) & 5.33 (0.87) & 6.01 (2.11) & 6.71 (3.44)\\
Missouri State University & 5.74 (1.56) & 6.85 (2.82) & 6.67 (2.08) & 7.45 (5.35) & 6.65 (2.92)\\
Montana State University & 5.81 (1.74) & 7.25 (3.35) & 5.59 (1.13) & 5.76 (1.74) & 6.69 (2.93)\\
Mechanical Turk 1 & 6.27 (2.28) & 7.74 (4.34) & 5.77 (1.17) & 5.57 (1.40) & 7.14 (3.79)\\
Mechanical Turk 2 & 5.76 (1.36) & 6.62 (1.85) & 5.92 (1.38) & 5.78 (1.17) & 6.38 (1.75)\\
Total & 5.78 (1.61) & 6.94 (2.88) & 5.67 (1.18) & 5.84 (1.71) & 6.57 (2.60)\\
\bottomrule
\end{tabular}
\end{threeparttable}
\end{center}
\end{table}

\begin{table}[tbp]
\begin{center}
\begin{threeparttable}
\caption{\label{tab:percent-table}Percent and Average Percent of Frequency for Cue-Feature Part of Speech Combinations}
\begin{tabular}{llcccc}
\toprule
Cue Type & Feature Type & \% Raw & \% Root & $M$ (SD) Freq. Raw & $M$ (SD) Freq. Root\\
\midrule
Adjective & Adjective & 38.09 & 29.74 & 17.84 (16.47) & 30.02 (18.83)\\
 & Noun & 40.02 & 46.74 & 13.14 (14.96) & 29.71 (19.94)\\
 & Verb & 17.69 & 20.72 & 8.51 (9.78) & 26.88 (17.27)\\
 & Other & 4.20 & 2.80 & 15.17 (15.64) & 28.04 (15.54)\\
Noun & Adjective & 16.56 & 12.07 & 15.55 (15.17) & 31.20 (18.17)\\
 & Noun & 60.85 & 62.67 & 17.21 (17.01) & 33.26 (20.05)\\
 & Verb & 20.80 & 23.68 & 8.88 (9.73) & 31.01 (17.87)\\
 & Other & 1.79 & 1.58 & 17.06 (15.29) & 28.87 (17.14)\\
Verb & Adjective & 15.16 & 12.27 & 13.95 (13.98) & 30.03 (18.28)\\
 & Noun & 42.92 & 44.35 & 14.59 (14.92) & 29.59 (18.90)\\
 & Verb & 36.92 & 39.72 & 12.75 (14.85) & 30.43 (19.54)\\
 & Other & 5.00 & 3.66 & 19.16 (15.95) & 25.59 (19.54)\\
Other & Adjective & 20.80 & 20.32 & 16.61 (17.37) & 31.66 (19.51)\\
 & Noun & 42.74 & 39.03 & 16.77 (19.41) & 37.28 (25.94)\\
 & Verb & 19.66 & 23.93 & 7.18 (7.57) & 26.14 (19.38)\\
 & Other & 16.81 & 16.71 & 22.72 (16.69) & 30.70 (18.48)\\
Total & Adjective & 19.74 & 14.93 & 16.12 (15.57) & 30.75 (18.37)\\
 & Noun & 55.41 & 57.81 & 16.55 (16.74) & 32.58 (20.09)\\
 & Verb & 22.02 & 24.95 & 9.50 (10.91) & 30.29 (18.24)\\
 & Other & 2.82 & 2.31 & 17.76 (15.83) & 28.45 (16.83)\\
\bottomrule
\addlinespace
\end{tabular}
\begin{tablenotes}[para]
\normalsize{\textit{Note.} Raw words indicate original feature listed, while root words indicated translated feature. These data are only from the current project.}
\end{tablenotes}
\end{threeparttable}
\end{center}
\end{table}

The top cue-feature combinations for the reprocessed and new data
collection were then combined with the cue-feature combinations from
McRae et al. (2005) and Vinson and Vigliocco (2008). We included all the
cue-feature combinations listed in their supplemental files with the
feature in the raw feature column. If features could be translated into
root words with affixes, the same procedure as described above was
applied. The final file then included columns for the original dataset,
cue, feature, translated feature, frequency of the original feature,
frequency of the translated feature, sample size, and frequency
percentages for the original and translated feature. The cue-feature
file includes 69284 cue-raw feature combinations, where 48925 are from
our dataset, and 24449 of which are cue-translated feature combinations.

The final data processing step was to code affixes found on the original
features. Multiple affix codes were often needed for features, as
\emph{beautifully} would have been translated to \emph{beauty},
\emph{ful}, and \emph{ly} (the \emph{root, a1, and a2} columns; though,
three affix columns were created in total). The research team searched
lists of affixes online and collectively discussed how to code each
affix, and the complete coding system can be found online in our OSF
files. If an affix coding was unclear, the root and affix word were
discussed in a lab meeting. Table \ref{tab:affix-table} displays the
list of affix types, common examples for each type of affix, and the
percent of affixes that fell into each category. The percent values are
calculated on the overall affix list, as feature words could have up to
three different affixes. Generally, affixes were tagged in a one-to-one
match, however, special care was taken with numbers and verb tenses, and
the lead author checked these categories after lab member coding.
Features like cat\emph{s} would be coded as a number affix, while
features like walk\emph{s} would be coded as a third person verb.

In the final words file found online, we additionally added forward
strength (FSG) and backward strength (BSG) for investigation into
association overlap (Nelson et al., 2004). Forward strength indicates
the number of times a target word was listed in response to a cue word
in a free association task, which simply asks participants to name the
first word that comes to mind when presented with a cue word. Backward
strength is the number of times a cue word was listed with a target
word, as free association is directional (i.e., the number of times
\emph{cheese} is listed in response to \emph{cheddar} is not the same as
the number of times that \emph{cheddar} is listed in response to
\emph{cheese}). The last few columns indicate the word list a concept
was originally normed in to allow for matching to the original raw files
on the OSF page, along with the code for each school and time point of
collection.

\begin{table}[tbp]
\begin{center}
\begin{threeparttable}
\caption{\label{tab:affix-table}Example of Affix Coding and Percent of Affixes Found}
\begin{tabular}{llc}
\toprule
Affix Type & Example & Percent\\
\midrule
Actions/Processes & ion, ment, ble, ate, ize & 8.21\\
Characteristic & y, ous, nt, ful, ive, wise & 22.72\\
Location & under, sub, mid, inter & 0.44\\
Magnitude & er, est, over, super, extra & 1.31\\
Not & less, dis, un, non, in , im, ab & 2.76\\
Number & s, uni, bi, tri, semi & 28.31\\
Opposites/Wrong & mis, anti, de & 0.13\\
Past Tense & ed & 8.03\\
Person/Object & er, or, men, person, ess, ist & 7.23\\
Present Participle & ing & 14.03\\
Slang & bros, bike, bbq, diff, h2o & 0.12\\
Third Person & s & 6.16\\
Time & fore, pre, post, re & 0.54\\
\bottomrule
\end{tabular}
\end{threeparttable}
\end{center}
\end{table}

Both forms of the feature are provided for flexibility in calculating
overlap by using the original feature (raw), the translated feature
(root), and the affix overlap by code (affix). Cosine values were
calculated for each of these feature sets by using the following
formula:

\[
\frac{\sum_{i=1}^{n} A_i \times B_i} {\sqrt{\sum_{i=1}^{n} A_i^2} \times \sqrt{\sum_{i=1}^{n} B_i^2}}
\]

This formula is similar to a dot-product correlation, where \(A_i\) and
\(B_i\) indicate the overlapping feature frequency (normalized,
therefore, the percent) between cue A and cue B. The \emph{i} subscript
denotes the current cue, and when features match, the frequencies are
multiplied together and summed across all matches (\(\Sigma\)). For the
denominator, the feature frequency is first squared and summed from
\emph{i} to \emph{n} features for cue A and B. The square root of these
summation values is then multiplied together. In essence, the numerator
calculates the overlap of feature frequency for matching features, while
the denominator accounts for the entire feature frequency set for each
cue. Cosine values range from 0 (no overlapping features) to 1 (complete
overlapping features). With over four thousand cue words, just under
twenty million cue-cue cosine combinations can be calculated. In the
datasets presented online, we only included cue-cue combinations with a
feature overlap of at least two features, in order to reduce the large
quantity of zero and very low cosine values. This procedure additionally
allowed for online presentation of the data, as millions of cosines were
not feasible for our server. The complete feature list, along with our
code to calculate cosine, can be used to obtain values not presented in
our data if desired.

\subsection{Website}\label{website}

In addition to our OSF page, we present a revamped website for this data
at \url{http://www.wordnorms.com/}. The single word norms page includes
information about each of the cue words including cue set size,
concreteness, word frequency from multiple sources, length, full part of
speech, orthographic/phonographic neighborhood, and number of phonemes,
syllables, and morphemes. These values were taken from Nelson et al.
(2004), Balota et al. (2007), and Brysbaert and New (2009). A definition
of each of these variables is provided along with the minimum, maximum,
mean, and standard deviation of numeric values. The table is programmed
using Shiny apps (Chang, Cheng, Allaire, Xie, \& McPherson, 2017). Shiny
is an \emph{R} package that allows the creation of dynamic graphical
user interfaces for interactive web applications. The advantage to using
Shiny applications is data manipulation and visualization with the
additional bonus of up to date statistics for provided data (i.e., as
typos are fixed or data is updated, the web app will display the most
recent calculations). In addition to the variable table, users can
search and save filtered output using our Shiny search app. With this
app, you can filter for specific variable ranges and save the output in
a csv or Excel file. The complete data is also provided for download.

On the word pair norms page, all information about word-pair statistics
can be found. A second variable table is provided with semantic and
associative statistics. This dataset includes the cue and target words
from this project (cue-cue combinations), the root, raw, and affix
cosines described above, as well as the original Buchanan et al. (2013)
cosines. Additional semantic information includes Latent Semantic
Analysis (LSA; Landauer \& Dumais, 1997) and JCN (JCN stands for
Jiang-Conrath, see explanation below; Jiang \& Conrath, 1997) values
provided in the Maki, McKinley, and Thompson (2004) norms, along with
forward strength and backward strength (FSG; BSG) from the Nelson et al.
(2004) norms for association. The definitions, minimum, maximum, mean,
and standard deviations of these values are provided in the app. Again,
the searchable app includes all of these stimuli for cue-cue
combinations with two or more features in common, where you can filter
this data for experimental stimuli creation. The separation of single
and word-pair data (as well as cosine calculation reduction to cues with
two features in common) was practical, as the applications run slowly as
a factor of the number of rows and columns of data. On each page, we
link the data, applications, and source code so that others may use and
manipulate our work depending on their data creation or visualization
goals.

\section{Results}\label{results}

An examination of the results of the cue-feature lists indicated that
the new data collected was similar to the previous semantic feature
production norms. As shown in Table \ref{tab:feature-table}, the new
Mechanical Turk data showed roughly the same number of listed features
for each cue concept, usually between five to seven features. These
numbers represent, for each cue and part of speech, the average number
of distinct cue-feature pairs provided by participants after processing.
Table \ref{tab:percent-table} portrayed that adjective cues generally
included other adjectives or nouns as features, while noun cues were
predominately described by other nouns. Verb cues included a large
feature list of nouns and other verbs, followed by adjectives and other
word forms. Lastly, the other cue types generally elicited nouns and
verbs. Frequency percentages were generally between seven and twenty
percent when examining the raw words. These words included multiple
forms, as the percent increased to around thirty percent when features
were translated into their root words. Indeed, nearly half of the 48925
cue-feature pairs were repeated, as 24449 cue-feature pairs were unique
when examining translated features. Generally, because of the
translation process, word forms shifted towards nouns and verbs and away
from adjectives because adjectives are often formed by adding an affix
to a noun or verb.

36030 affix values were found, which arose from 4407 of the 4436 cue
concepts. 33052 first affixes were found, with 2832 second place
affixes, and 146 third place affixes. Table \ref{tab:affix-table} shows
the distribution of these affix values. Generally, numbers were the
largest category of affixes demonstrating that participants often
indicated the quantity of the feature when describing the cue word. The
second largest affix category was characteristics which denoted the
switch to or from a noun form of the feature word (i.e., \emph{angry} to
\emph{anger}). Verb tenses (past tense, present participle, and third
person) comprised a large set of affixes indicating the type of concept
or when a concept might be doing an action for a cue. Persons and
objects affixes were used about 7\% of the time on features to explain
cues, while actions and processes were added to the feature about 8\% of
the time.

\subsection{Divergent Validity}\label{divergent-validity}

\begin{table}[tbp]
\begin{center}
\begin{threeparttable}
\caption{\label{tab:divergent-table}Percent and Mean Overlap to the Free Association Norms}
\begin{tabular}{lcccccl}
\toprule
  & \% Overlap & $M$ FSG & $SD$ FSG & Min & Max & $r$\\
\midrule
Adjective & 51.86 & .12 & .15 & .01 & .94 & .36\\
Noun & 36.48 & .11 & .14 & .01 & .91 & .40\\
Verb & 32.15 & .11 & .13 & .01 & .94 & .44\\
Other & 44.44 & .13 & .18 & .01 & .88 & .09\\
Total & 37.47 & .11 & .14 & .01 & .94 & .39\\
All Buchanan cues & 52.12 & .11 & .14 & .01 & .94 & .41\\
McRae et al. cues & 23.50 & .10 & .14 & .01 & .91 & .28\\
Vinson \& Vigliocco cues & 15.19 & .09 & .13 & .01 & .88 & .38\\
Overlapping Cues & 27.26 & .09 & .14 & .01 & .88 & .30\\
\bottomrule
\addlinespace
\end{tabular}
\begin{tablenotes}[para]
\normalsize{\textit{Note.} Overlap was defined as the percent of cue-feature combinations from our feature list included in the Nelson et al. (2004) norms. FSG: Forward strength indicating the number of times a target was elicited after seeing a cue word. Correlation represents the relationship between frequency percent and forward strength.}
\end{tablenotes}
\end{threeparttable}
\end{center}
\end{table}

When collecting semantic feature production norms, there can be a
concern that the information produced will simply mimic the free
association norms, and thus, be a more of representation of association
(context) rather than semanticity (meaning). Association and semanticity
do overlap, however, the variables used to represent these concepts have
been shown to tap different underlying constructs (Maki \& Buchanan,
2008). Therefore, it is important to show that, while some overlap is
expected, the semantic feature production norms provide useful, separate
information from the free association norms. Table
\ref{tab:divergent-table} portrays the overlap with the Nelson et al.
(2004) norms. The percent of time a cue-feature combination was present
in the free association norms was calculated, along with the average
forward strength for those overlapping pairs. First, these values were
calculated on the complete dataset with the McRae et al. (2005) and
Vinson and Vigliocco (2008) norms--as we are presenting them as a
combined dataset--on the translated cue-feature set only. Because we
used the translated cue-feature set, repeated instances of cue-features
would occur (i.e., the original \emph{abandon-leave} and
\emph{abandon-leaving} becomes two lines when only using translated
\emph{abandon-leave}), and thus only the unique set was considered.
Second, we calculated these values on each dataset separately, as well
as for the 26 cues that overlapped in all three datasets.

The overall overlap between the database cue-feature sets and the free
association cue-target sets was approximately 37\%, ranging from 32\%
for verbs and nearly 52\% for adjectives. Similar to our previous
results, the range of the forward strength was large (.01 - .94),
however, the average forward strength was low for overlapping pairs,
\emph{M} = .11 (\emph{SD} = .14). These results indicated that while it
will always be difficult to separate association and meaning, the
dataset presented here represents a low association when examining
overlapping values, and more than 60\% of the data is completely
separate from the free association norms. The limitation to this finding
is the removal of idiosyncratic responses from the Nelson et al. (2004)
norms, but even if these were to be included in some form, the average
forward strength would still be quite low when comparing cue-feature
lists to cue-target lists. In examining these values by dataset, it
appears that the new norms have the highest overlap with the Nelson et
al. (2004) data, while the average, standard deviation, minimum, and
maximum values were roughly similar for each dataset and the overlapping
cues. This effect is likely driven by the inclusion of adjectives and
other forms of speech, which show higher overlaps than nouns and verbs,
which represent the cues present in McRae et al. (2005) and Vinson and
Vigliocco (2008).

In the last column of Table \ref{tab:divergent-table}, we calculated the
correlation between forward strength and the frequency percent for the
the root (translated) cue-feature pairs. This correlation provides
information the relation between the strength of the association and the
frequency of cue-feature mentions. Correlations were similar across
parts of speech except, notably, the other category included the lowest
relation. This result is likely because the instructions of a semantic
feature production task might exclude normal \enquote{first word that
pops into your mind} association task concepts. The correlations across
datasets and the overlapping cues were also similar, denoting that as
forward strength increased, the likelihood of the cue-feature mentions
also increased. In general, these cue-feature pairs were still of low
associative strength, as shown in the mean column of Table
\ref{tab:divergent-table}.

\subsection{Convergent Validity}\label{convergent-validity}

To examine the validity of cosine values, we calculated the average
cosine score between the new processing of the data for each of the
three feature production norms used in this project. Overlapping cues in
all of the three databases were found (\emph{n} = 188), and the average
cosine between their feature sets was examined. Buchanan et al. (2013)
and the new dataset are listed with the subscript B, while McRae et al.
(2005) is referred to with M and V for Vinson and Vigliocco (2008). For
root cosine values, we found high overlap between all three datasets:
\(M_{BM}\) = .67 (\emph{SD} = .14), \(M_{BV}\) = .66 (\emph{SD} = .18),
and \(M_{MV}\) = .72 (\emph{SD} = .11). The raw cosine values also
correlated, even though the McRae et al. (2005) and Vinson and Vigliocco
(2008) datasets were already mostly preprocessed for word stems:
\(M_{BM}\) = .55 (\emph{SD} = .15), \(M_{BV}\) = .54 (\emph{SD} = .20),
and \(M_{MV}\) = .45 (\emph{SD} = .19). Last, the affix cosines
overlapped similarly between Buchanan et al. (2013) and McRae et al.
(2005) datasets, \(M_{BM}\) = .43 (\emph{SD} = .29), but did not overlap
with the Vinson and Vigliocco (2008) datasets: \(M_{BV}\) = .04
(\emph{SD} = .14), and \(M_{MV}\) = .09 (\emph{SD} = .19), likely due to
Vinson and Vigliocco (2008) dataset preprocessing.

The correlation between root, raw, affix, previously found cosine,
Latent Semantic Analysis score (LSA), and Jiang-Conrath semantic
distance (JCN) were calculated to examine convergent validity. LSA is
one of the most well-known semantic memory models (Landauer \& Dumais,
1997; McRae \& Jones, 2013), wherein a large text corpus (i.e., many
texts) is used to create a word by document (i.e., each text) matrix.
From this matrix, words are weighted relative to their frequency, and
singular value decomposition is then used to select only the largest
semantic components. This process creates a word space that can then be
used to calculate the relation between two cues by examining the
patterns of their occurrence across documents, usually cosine or
correlation. JCN is calculated from an online dictionary (WordNet;
Fellbaum \& Felbaum, 1998), by measuring the semantic distance between
concepts in a hierarchical structure. JCN is backwards coded, as zero
values indicate close semantic neighbors (low dictionary distance) and
high values indicate low semantic relation. These two measures were
selected for convergent validity because they are well-cited measures of
semanticity. To examine if the type of processing impacted convergent
validity of the dataset, we calculcated the McRae et al. (2005) and
Vinson and Vigliocco (2008) cosine values based on their original
cue-feature matrices provided in their publications. These datasets were
coded for more complex features in a propositional style (\enquote{is
a}, \enquote{has a}), while our processing took a single word count
based approach. Therefore, providing the original processing
correlations allows one to examine if the cosine values provided are
covergent, as well as similarly correlated across other measures of
semanticity.

As shown in Table \ref{tab:correlation-table}, the intercorrelations
between the cosine measures (root, raw, affix) are high, especially
between our previous work and this dataset. We found that the
correlation between processing styles was high and matched the
intercorrelations between the new cosine measures (indicating convergent
validity of coding style). The small negative correlations between JCN
and cosine measures replicated previous findings (Buchanan et al.,
2013). LSA values showed small positive correlations with cosine values,
indicating some overlap with thematic information and semantic feature
overlap (Maki \& Buchanan, 2008). These correlations were slightly
different than our previous publication, likely because here we
restricted this cosine set to values with at least two features in
common. LSA and JCN correlations were lower than LSA-cosine and
JCN-cosine, but these values indicated that themes and dictionary
distance were similarly related to feature overlap. Last, the
correlation between propositional processing (\enquote{MV COS} column)
and JCN was higher than the new root cosine measure (-.39 versus -.18
respectively). JCN is created through a hierarchical dictionary with a
structure similar to the complex propositional coding provided in McRae
et al. (2005) and Vinson and Vigliocco (2008), and correspondingly, the
relation between them is stronger.

\begin{lltable}


\begin{TableNotes}[para]
\normalsize{\textit{Note.} Root, raw, and affix cosine values are from the current reprocessed dataset. PCOS indicates the cosine values in the original Buchanan et al. (2013) dataset. MV COS: Cosine values from only the McRae et al. (2005) and Vinson and Vigliocco (2008) data, JCN: Jiang-Conrath semantic distance, LSA: Latent Semantic Analysis score, FSG: Forward Strength, BSG: Backward Strength. Sample sizes for each correlation are presented in the top half of the table.}
\end{TableNotes}
\small{
\begin{longtable}{lccccccccl}\noalign{\getlongtablewidth\global\LTcapwidth=\longtablewidth}
\caption{\label{tab:correlation-table}Correlations and 95\% CI between Semantic and Associative Variables}\\
\toprule
  & Root & Raw & Affix & PCOS & MV COS & JCN & LSA & FSG & BSG\\
\midrule
Root & 1 & 208515 & 208515 & 83762 & 101446 & 5617 & 5590 & 6753 & 6685\\
Raw & .93 [.93,.93] & 1 & 208515 & 83762 & 101446 & 5617 & 5590 & 6753 & 6685\\
Affix & .50 [.50,.50] & .53 [.53,.54] & 1 & 83762 & 101446 & 5617 & 5590 & 6753 & 6685\\
PCOS & .94 [.94,.94] & .91 [.91,.91] & .49 [.48,.49] & 1 & 52342 & 2762 & 2759 & 3280 & 3243\\
MV COS & .84 [.84,.84] & .89 [.89,.89] & .46 [.45,.46] & .83 [.82,.83] & 1 & 1179 & 1179 & 1248 & 1232\\
JCN & -.18 [-.20,-.15] & -.22 [-.25,-.20] & -.17 [-.20,-.15] & -.22 [-.26,-.19] & -.39 [-.44,-.34] & 1 & 5590 & 5617 & 5617\\
LSA & .18 [.16,.21] & .15 [.12,.18] & .10 [.07,.13] & .21 [.18,.25] & .14 [.08,.19] & -.06 [-.08,-.03] & 1 & 5590 & 5590\\
FSG & .06 [.04,.08] & .04 [.01,.06] & .08 [.05,.10] & .10 [.06,.13] & .10 [.04,.15] & -.15 [-.18,-.13] & .24 [.22,.27] & 1 & 6685\\
BSG & .14 [.12,.16] & .15 [.13,.17] & .17 [.14,.19] & .18 [.15,.22] & .26 [.20,.31] & -.18 [-.21,-.16] & .26 [.23,.28] & .31 [.29,.33] & 1\\
\bottomrule
\addlinespace
\insertTableNotes
\end{longtable}
}
\end{lltable}

\subsection{Relation to Semantic
Priming}\label{relation-to-semantic-priming}

\begin{table}[tbp]
\begin{center}
\begin{threeparttable}
\caption{\label{tab:ldt-table}Lexical Decision Response Latencies' Correlation and 95\% CI with Semantic and Associative Variables}
\small{
\begin{tabular}{lcccc}
\toprule
Variable & \multicolumn{1}{c}{First 200} & \multicolumn{1}{c}{First 1200} & \multicolumn{1}{c}{Other 200} & \multicolumn{1}{c}{Other 1200}\\
\midrule
Root Cosine & .06 [.01,.12] & -.05 [-.10,.01] & .09 [.03,.14] & .09 [.03,.14]\\
Raw Cosine & .07 [.02,.12] & .05 [-.01,.10] & .09 [.04,.15] & .07 [.01,.12]\\
Affix Cosine & -.01 [-.06,.05] & .00 [-.05,.06] & .06 [.00,.11] & .04 [-.01,.10]\\
Target Root FSS & -.02 [-.07,.04] & -.31 [-.36,-.26] & -.03 [-.09,.02] & -.03 [-.08,.03]\\
Target Raw FSS & -.09 [-.15,-.04] & -.27 [-.32,-.22] & -.03 [-.08,.03] & -.02 [-.08,.03]\\
Target CSS & -.07 [-.12,-.02] & -.11 [-.16,-.06] & -.05 [-.10,.01] & .02 [-.04,.07]\\
Cue Root FSS & -.02 [-.07,.04] & -.32 [-.37,-.27] & .03 [-.02,.09] & .03 [-.02,.09]\\
Cue Raw FSS & .01 [-.04,.07] & -.34 [-.38,-.29] & .01 [-.05,.06] & .01 [-.04,.07]\\
Cue CSS & .16 [.11,.21] & -.23 [-.28,-.18] & .06 [.01,.12] & .01 [-.05,.06]\\
Forward Strength & -.12 [-.17,-.06] & -.12 [-.18,-.07] & .07 [.01,.12] & .04 [-.01,.10]\\
Backward Strength & .15 [.10,.20] & .10 [.04,.15] & .08 [.03,.14] & .04 [-.02,.10]\\
LSA & .05 [-.00,.11] & -.20 [-.26,-.15] & .13 [.08,.19] & .09 [.03,.14]\\
Jiang-Conrath & -.05 [-.11,.00] & .11 [.06,.17] & -.05 [-.11,.00] & .01 [-.04,.07]\\
\bottomrule
\addlinespace
\end{tabular}
}
\begin{tablenotes}[para]
\normalsize{\textit{Note.} First indicates first associate, other indicates other associate cue-target relation. 200 and 1200 ms represent the SOA, which is the time from the presentation of the cue to the target. CSS: Cue set size, FSS: Feature set size, LSA: Latent Semantic Analysis distance. Sample size is 1290 cue-target pairs for first associates and 1254 pairs for other associates.}
\end{tablenotes}
\end{threeparttable}
\end{center}
\end{table}

\begin{table}[tbp]
\begin{center}
\begin{threeparttable}
\caption{\label{tab:name-table}Naming Response Latencies' Correlation and 95\% CI with Semantic and Associative Variables}
\small{
\begin{tabular}{lcccc}
\toprule
Variable & \multicolumn{1}{c}{FA 200} & \multicolumn{1}{c}{FA 1200} & \multicolumn{1}{c}{OA 200} & \multicolumn{1}{c}{OA 1200}\\
\midrule
Root Cosine & -.02 [-.08,.03] & .10 [.05,.15] & -.00 [-.06,.05] & .06 [.00,.11]\\
Raw Cosine & -.02 [-.07,.04] & .11 [.06,.17] & -.01 [-.06,.05] & .05 [-.01,.10]\\
Affix Cosine & -.01 [-.07,.04] & .06 [.01,.11] & .03 [-.03,.08] & .01 [-.05,.06]\\
Target Root FSS & -.03 [-.09,.02] & -.03 [-.09,.02] & -.01 [-.07,.04] & .03 [-.03,.08]\\
Target Raw FSS & -.04 [-.09,.02] & -.02 [-.07,.04] & -.02 [-.08,.03] & .03 [-.02,.09]\\
Target CSS & -.06 [-.11,-.00] & -.04 [-.09,.02] & -.02 [-.08,.03] & .01 [-.04,.07]\\
Cue Root FSS & -.03 [-.09,.02] & -.00 [-.06,.05] & .02 [-.03,.08] & -.02 [-.07,.04]\\
Cue Raw FSS & -.01 [-.07,.04] & -.01 [-.07,.04] & .02 [-.04,.07] & -.02 [-.07,.04]\\
Cue CSS & -.01 [-.06,.05] & -.01 [-.07,.04] & -.01 [-.07,.04] & -.01 [-.06,.05]\\
Forward Strength & -.02 [-.08,.03] & .02 [-.03,.08] & .04 [-.01,.10] & .04 [-.01,.10]\\
Backward Strength & .10 [.05,.15] & .08 [.02,.13] & .11 [.06,.17] & .04 [-.02,.09]\\
LSA & .06 [.01,.12] & .03 [-.02,.09] & .06 [.00,.11] & .03 [-.03,.08]\\
Jiang-Conrath & -.05 [-.11,.00] & .00 [-.05,.06] & -.09 [-.14,-.03] & -.01 [-.06,.05]\\
\bottomrule
\addlinespace
\end{tabular}
}
\begin{tablenotes}[para]
\normalsize{\textit{Note.} First indicates first associate, other indicates other associate cue-target relation. 200 and 1200 ms represent the SOA, which is the time from the presentation of the cue to the target. CSS: Cue set size, FSS: Feature set size, LSA: Latent Semantic Analysis distance. Sample size is 1287 cue-target pairs for first associates and 1249 pairs for other associates.}
\end{tablenotes}
\end{threeparttable}
\end{center}
\end{table}

As a second examination of convergent validity, the correlation between
values calculated from these norms and the \emph{Z}-priming values from
the Semantic Priming Project were examined. The Semantic Priming Project
includes lexical decision and naming response latencies for priming at
200 and 1200 ms stimulus onset asynchronies (SOA). In these experiments,
participants were shown cue-target words that were either the first
associate of a concept or an other associate (second response or higher
in the Nelson et al. (2004) norms) with the delay between the cue and
target matching either 200 or 1200 ms (SOA). The response latency of the
target word in the related condition (either first or other associate)
was subtracted from the response latency in the unrelated condition to
create a priming response latency. Therefore, each target item received
four (two SOAs by two tasks: lexical decision or naming) priming times.
We selected the \emph{Z}-scored priming from the dataset to correlate
with our data, as Hutchison et al. (2013) demonstrated that the
\emph{Z}-scored data more accurately captures priming controlled for
individual differences in reaction times.

In addition to root, raw, and affix cosine, we additionally calculated
feature set size for the cue and target of the primed pairs. Feature set
size is the number of features listed by participants when creating the
norms for that concept. Because of the nature of our norms, we
calculated both feature set size for the raw, untranslated features, as
well as the translated features. The average feature set sizes for our
dataset can be found in Table \ref{tab:feature-table}. The last variable
included was cosine set size which was defined as the number of other
concepts each cue or target was nonzero paired with in the cosine
values. Feature set size indicates the number of features listed for
each cue or target, while cosine set size indicates the number of other
semantically related concepts for each cue or target. Feature and cue
set size are often called semantic richness, representing the
variability or extent of associated information for a cue (Buchanan,
Westbury, \& Burgess, 2001; Pexman, Hargreaves, Edwards, Henry, \&
Goodyear, 2007; Pexman, Hargreaves, Siakaluk, Bodner, \& Pope, 2008).
Several studies have showed the positive effects of semantic richness on
semantic tasks based on task demand (Duñabeitia, Avilés, \& Carreiras,
2008; Pexman et al., 2008; Yap, Pexman, Wellsby, Hargreaves, \& Huff,
2012; Yap, Tan, Pexman, \& Hargreaves, 2011), and thus, they were
included as important variables to examine.

Tables \ref{tab:ldt-table} (for the lexical decision task) and
\ref{tab:name-table} (for the naming task) display the correlations
between the new semantic variables described above, as well as forward
strength, backward strength, Latent Semantic Analysis score, and
Jiang-Conrath semantic distance for reference. Only cue-target pairs
with complete values were included in this analysis to allow for
comparison between correlations. For lexical decision priming, we found
small correlations between the root and raw cosine values and priming,
with the largest for first associates in the 200 ms condition. The
correlations decreased for the 1200 ms condition and the other associate
SOAs. These two variables are highly correlated, therefore, it is not
surprising that they have similar correlations with priming. Affix
cosine also was slightly related to priming, especially for first
associates in the 200 ms condition. Most of the cue and feature set
sizes were not related to priming, showing correlations close to zero in
most instances. Cue set size for the cue word was somewhat related to
200 ms priming, along with raw cue feature set size (for first
associates only). These correlations are small, but they are comparable
or greater than the correlations for association and other measures of
semantic or thematic relatedness. For naming, the results are less
consistent. Cosine values are related to 1200 ms naming in first
associates, but none of the feature or cue set sizes showed any
relationship with priming. Again, we see that many of the other
associative and semantic variables correspondingly do not correlate with
priming. In both naming and lexical decision priming, backward strength
has a small but consistent relationship with priming, which may indicate
the processing of the target back to the cue. Latent Semantic Analysis
score was also a small predictor of priming across conditions.

As mentioned in the Website section, we have provided the data to
calculate a broad range of information of linguistic information or
simply use the provided values. From our OSF page (also linked to
GitHub: \url{https://github.com/doomlab/Word-Norms-2}), you can find the
data at each stage of processing and final data from this manuscript.
Interested researchers could use our raw feature files to create their
own coding schemes (or ones similar to McRae et al. (2005)), use the
processed files to calculate set sizes for each cue or feature, and use
these files plus the cosine files to create their own experimental
stimuli (also avaliable as a Shiny app on
\url{http://www.wordnorms.com}). These data could also be used to
calculate other measures of interest, such as pointwise positive mutual
information, entropy, and random walk statistics (De Deyne, Navarro,
Perfors, \& Storms, 2016).

\section{Discussion}\label{discussion}

This research project focused on expanding the availability of English
semantic feature overlap norms, in an effort to provide more coverage of
concepts that occur in other large database projects like the Semantic
Priming and English Lexicon Projects. The number and breadth of
linguistic variables and normed databases has increased over the years,
however, researchers can still be limited by the concept overlap between
them. Projects like the Small World of Words provide newly expanded
datasets for association norms, and our work helps fill the voids for
corresponding semantic norms. To provide the largest dataset of similar
data, we combined the newly collected data with previous work by using
Buchanan et al. (2013), McRae et al. (2005), and Vinson and Vigliocco
(2008) together. These norms were reprocessed from previous work to
explore the impact of feature coding for feature overlap. As shown in
the correlation between root and raw cosines, the parsing of words to
root form created very similar results across other variables. This
finding does not imply that these cosine values are the same, as root
cosines were larger than their corresponding raw cosine. It does,
however, imply that the cue-feature coding can produce similar results
in raw or translated format. Because the correlation between the current
paper's cosine values and the previous cosine values was nearly 1, we
would suggest using the new values, simply for the increase in dataset
size.

Of particular interest was the information that is often lost when
translating raw features back to a root word. One surprising result in
this study was the sheer number of affixes present on each cue word.
With these values, we believe we have captured some of the nuance that
is often discarded in this type of research. Affix cosines were less
related to their feature root and raw counterparts, but also showed
small correlations with semantic priming. Potentially, affix overlap can
be used to add small, but meaningful predictive value to related
semantic phenomena. Further investigation into the compound prediction
of these variables is warranted to fully explore how these, and other
lexical variables, may be used to understand semantic priming. An
examination of the cosine values from the Semantic Priming Project
cue-target set indicates that these values were low, with many zeros
(i.e., no feature overlap between cues and targets). This restriction of
range of the cosine relatedness could explain the small correlations
with priming because the semantic priming was variable, but the cosine
values were not.

One important limitation of the instructions in this study is that
multiple senses of concepts were not distinguished. We did not wish to
prime participants for specific senses to capture the features for
multiple senses of a concept, however, this procedure could lead to
lower cosine values for concepts that might intuitively seem very
related. The feature production lists could be used to sort senses and
recalculate overlap values, but it is likely that feature information is
correspondingly mixed or sorted into small sublists in memory as well.
The addition of the coded affix information may help capture some of
those sense differences, as well as some of the spatial and relational
features that are not traditionally captured by simple feature
production. For example, by understanding the numbers or actors affixes,
we may gain more information about semanticity that is often regarded as
something to disregard in data processing.

We encourage readers to use the corresponding website associated with
these norms to download the data, explore the Shiny apps, and use the
options provided for controlled experimental stimuli creation. We
previously documented the limitations of feature production norms that
rely on on single word instances as their features (i.e., \emph{four}
and \emph{legs}), rather than combined phrase sets. One potential
limitation, then, is the inability to create fine distinctions between
cues; however, the small feature set sizes imply that the granulation of
features is large, since many distinguishing features are often never
listed in these tasks. For instance, \emph{dogs} are living creatures,
but \emph{has lungs} or \emph{has skin} would usually not be listed
during a feature production task, and thus, feature sets should not be
considered a complete snapshot of mental representation (Rogers \&
McClelland, 2004). Additionally, the cue-feature lists could be explored
for the type of cue-feature representation that is listed for each part
of speech (i.e., physical, functional, etc.) and the complexity in
coding could be increased or decreased depending on researcher goal. The
previous data and other norms were purposely combined in the recoded
format, so that researchers could use the entire set of available norms
which increases comparability across datasets. Given the strong
correlation between databases, we suspect that using single word
features does not reduce their reliability and validity. We found high
correlations between the different types of feature coding (i.e.,
complex/propositional versus single word/count), thus suggesting that
either dataset could be used for future work where the advantage of the
current project is the size of the norms.

\newpage

\section{References}\label{references}

\setlength{\parindent}{-0.5in} \setlength{\leftskip}{0.5in}

\hypertarget{refs}{}
\hypertarget{ref-Ashcraft1978a}{}
Ashcraft, M. H. (1978). Property norms for typical and atypical items
from 17 categories: A description and discussion. \emph{Memory \&
Cognition}, \emph{6}(3), 227--232.
doi:\href{https://doi.org/10.3758/BF03197450}{10.3758/BF03197450}

\hypertarget{ref-R-papaja}{}
Aust, F., \& Barth, M. (2018). \emph{papaja: Create APA manuscripts with
R Markdown}. Retrieved from \url{https://github.com/crsh/papaja}

\hypertarget{ref-Balota2007}{}
Balota, D. A., Yap, M. J., Hutchison, K. A., Cortese, M. J., Kessler,
B., Loftis, B., \ldots{} Treiman, R. (2007). The English lexicon
project. \emph{Behavior Research Methods}, \emph{39}(3), 445--459.
doi:\href{https://doi.org/10.3758/BF03193014}{10.3758/BF03193014}

\hypertarget{ref-Barsalou2003}{}
Barsalou, L. W. (2003). Abstraction in perceptual symbol systems.
\emph{Philosophical Transactions of the Royal Society B: Biological
Sciences}, \emph{358}(1435), 1177--1187.
doi:\href{https://doi.org/10.1098/rstb.2003.1319}{10.1098/rstb.2003.1319}

\hypertarget{ref-Bradley1980}{}
Bradley, D. (1980). Lexical representation of derivational relation. In
M. Aronoff \& M. L. Kean (Eds.), \emph{Juncture} (pp. 37--55). Saratoga,
CA: Anma Libri.

\hypertarget{ref-Brysbaert2009}{}
Brysbaert, M., \& New, B. (2009). Moving beyond Kučera and Francis: A
critical evaluation of current word frequency norms and the introduction
of a new and improved word frequency measure for American English.
\emph{Behavior Research Methods}, \emph{41}(4), 977--990.
doi:\href{https://doi.org/10.3758/BRM.41.4.977}{10.3758/BRM.41.4.977}

\hypertarget{ref-Brysbaert2013}{}
Brysbaert, M., Warriner, A. B., \& Kuperman, V. (2014). Concreteness
ratings for 40 thousand generally known English word lemmas.
\emph{Behavior Research Methods}, \emph{46}(3), 904--911.
doi:\href{https://doi.org/10.3758/s13428-013-0403-5}{10.3758/s13428-013-0403-5}

\hypertarget{ref-Buchanan2013}{}
Buchanan, E. M., Holmes, J. L., Teasley, M. L., \& Hutchison, K. A.
(2013). English semantic word-pair norms and a searchable Web portal for
experimental stimulus creation. \emph{Behavior Research Methods},
\emph{45}(3), 746--757.
doi:\href{https://doi.org/10.3758/s13428-012-0284-z}{10.3758/s13428-012-0284-z}

\hypertarget{ref-Buchanan2018}{}
Buchanan, E. M., Valentine, K. D., \& Maxwell, N. P. (2018). LAB:
Linguistic Annotated Bibliograpy - A searchable portal for normed
database information. Retrieved from \url{https://osf.io/r6y3n}

\hypertarget{ref-Buchanan2001}{}
Buchanan, L., Westbury, C., \& Burgess, C. (2001). Characterizing
semantic space: Neighborhood effects in word recognition.
\emph{Psychonomic Bulletin \& Review}, \emph{8}, 531--544.

\hypertarget{ref-Buhrmester2011}{}
Buhrmester, M., Kwang, T., \& Gosling, S. D. (2011). Amazon's Mechanical
Turk. \emph{Perspectives on Psychological Science}, \emph{6}(1), 3--5.
doi:\href{https://doi.org/10.1177/1745691610393980}{10.1177/1745691610393980}

\hypertarget{ref-Butterworth1983}{}
Butterworth, B. (1983). Lexical representation. In B. Butterworth (Ed.),
\emph{Language production, vol. ii: Development, writing and other
language processes} (pp. 257--294). London: Academic.

\hypertarget{ref-Caramazza1988}{}
Caramazza, A., Laudanna, A., \& Romani, C. (1988). Lexical access and
inflectional morphology. \emph{Cognition}, \emph{28}(3), 297--332.
doi:\href{https://doi.org/10.1016/0010-0277(88)90017-0}{10.1016/0010-0277(88)90017-0}

\hypertarget{ref-R-shiny}{}
Chang, W., Cheng, J., Allaire, J., Xie, Y., \& McPherson, J. (2017).
\emph{Shiny: Web application framework for r}. Retrieved from
\url{https://CRAN.R-project.org/package=shiny}

\hypertarget{ref-Collins1975}{}
Collins, A. M., \& Loftus, E. F. (1975). A spreading-activation theory
of semantic processing. \emph{Psychological Review}, \emph{82}(6),
407--428.
doi:\href{https://doi.org/10.1037/0033-295X.82.6.407}{10.1037/0033-295X.82.6.407}

\hypertarget{ref-Collins1969}{}
Collins, A. M., \& Quillian, M. R. (1969). Retrieval time from semantic
memory. \emph{Journal of Verbal Learning and Verbal Behavior},
\emph{8}(2), 240--247.
doi:\href{https://doi.org/10.1016/S0022-5371(69)80069-1}{10.1016/S0022-5371(69)80069-1}

\hypertarget{ref-Cree2003}{}
Cree, G. S., \& McRae, K. (2003). Analyzing the factors underlying the
structure and computation of the meaning of chipmunk, cherry, chisel,
cheese, and cello (and many other such concrete nouns). \emph{Journal of
Experimental Psychology: General}, \emph{132}(2), 163--201.
doi:\href{https://doi.org/10.1037/0096-3445.132.2.163}{10.1037/0096-3445.132.2.163}

\hypertarget{ref-Cree1999}{}
Cree, G. S., McRae, K., \& McNorgan, C. (1999). An attractor model of
lexical conceptual processing: Simulating semantic priming.
\emph{Cognitive Science}, \emph{23}, 371--414.
doi:\href{https://doi.org/10.1016/S0364-0213(99)00005-1}{10.1016/S0364-0213(99)00005-1}

\hypertarget{ref-DeDeyne2016}{}
De Deyne, S., Navarro, D. J., Perfors, A., \& Storms, G. (2016).
Structure at every scale: A semantic network account of the similarities
between unrelated concepts. \emph{Journal of Experimental Psychology:
General}, \emph{145}(9), 1228--1254.
doi:\href{https://doi.org/10.1037/xge0000192}{10.1037/xge0000192}

\hypertarget{ref-DeDeyne2008}{}
De Deyne, S., Verheyen, S., Ameel, E., Vanpaemel, W., Dry, M. J.,
Voorspoels, W., \& Storms, G. (2008). Exemplar by feature applicability
matrices and other Dutch normative data for semantic concepts.
\emph{Behavior Research Methods}, \emph{40}(4), 1030--1048.
doi:\href{https://doi.org/10.3758/BRM.40.4.1030}{10.3758/BRM.40.4.1030}

\hypertarget{ref-Devereux2014}{}
Devereux, B. J., Tyler, L. K., Geertzen, J., \& Randall, B. (2014). The
Centre for Speech , Language and the Brain ( CSLB ) concept property
norms, 1119--1127.
doi:\href{https://doi.org/10.3758/s13428-013-0420-4}{10.3758/s13428-013-0420-4}

\hypertarget{ref-Dewhurst1998}{}
Dewhurst, S. A., Hitch, G. J., \& Barry, C. (1998). Separate effects of
word frequency and age of acquisition in recognition and recall.
\emph{Journal of Experimental Psychology: Learning, Memory, and
Cognition}, \emph{24}(2), 284--298.
doi:\href{https://doi.org/10.1037/0278-7393.24.2.284}{10.1037/0278-7393.24.2.284}

\hypertarget{ref-Dunabeitia2008}{}
Duñabeitia, J. A., Avilés, A., \& Carreiras, M. (2008). NoA's ark:
Influence of the number of associates in visual word recognition.
\emph{Psychonomic Bulletin \& Review}, \emph{15}, 1072--1077.

\hypertarget{ref-Felbaum1998}{}
Fellbaum, C., \& Felbaum, C. (1998). \emph{WordNet: An electronic
lexical database}. Cambridge, MA: MIT Press.

\hypertarget{ref-Griffiths2007}{}
Griffiths, T. L., Steyvers, M., \& Tenenbaum, J. B. (2007). Topics in
semantic representation. \emph{Psychological Review}, \emph{114}(2),
211--244.
doi:\href{https://doi.org/10.1037/0033-295X.114.2.211}{10.1037/0033-295X.114.2.211}

\hypertarget{ref-Grondin2009}{}
Grondin, R., Lupker, S. J., \& McRae, K. (2009). Shared features
dominate semantic richness effects for concrete concepts. \emph{Journal
of Memory and Language}, \emph{60}(1), 1--19.
doi:\href{https://doi.org/10.1016/j.jml.2008.09.001}{10.1016/j.jml.2008.09.001}

\hypertarget{ref-Hutchison2013}{}
Hutchison, K. A., Balota, D. A., Neely, J. H., Cortese, M. J.,
Cohen-Shikora, E. R., Tse, C.-S., \ldots{} Buchanan, E. M. (2013). The
semantic priming project. \emph{Behavior Research Methods},
\emph{45}(4), 1099--1114.
doi:\href{https://doi.org/10.3758/s13428-012-0304-z}{10.3758/s13428-012-0304-z}

\hypertarget{ref-Jarvella1983}{}
Jarvella, R., \& Meijers, G. (1983). Recognizing morphemes in spoken
words: Some evidence for a stem-organized mental lexicon. In G. B.
Flores d'Arcaos \& R. Jarvella (Eds.), \emph{The process of language
understanding} (pp. 81--112). New York: Wiley.

\hypertarget{ref-Jiang1997}{}
Jiang, J. J., \& Conrath, D. W. (1997). Semantic similarity based on
corpus statistics and lexical taxonomy. \emph{Proceedings of
International Conference Research on Computational Linguistics (ROCLING
X)}. Retrieved from \url{http://arxiv.org/abs/cmp-lg/9709008}

\hypertarget{ref-Jones2007}{}
Jones, M. N., \& Mewhort, D. J. K. (2007). Representing word meaning and
order information in a composite holographic lexicon.
\emph{Psychological Review}, \emph{114}(1), 1--37.
doi:\href{https://doi.org/10.1037/0033-295X.114.1.1}{10.1037/0033-295X.114.1.1}

\hypertarget{ref-Jones2015a}{}
Jones, M. N., Willits, J., \& Dennis, S. (2015). Models of Semantic
Memory. \emph{Oxford Handbook of Mathematical and Computational
Psychology}, 232--254. Retrieved from
\url{http://psycnet.apa.org/psycinfo/2004-17297-001}

\hypertarget{ref-Keuleers2012}{}
Keuleers, E., Lacey, P., Rastle, K., \& Brysbaert, M. (2012). The
British Lexicon Project: Lexical decision data for 28,730 monosyllabic
and disyllabic English words. \emph{Behavior Research Methods},
\emph{44}(1), 287--304.
doi:\href{https://doi.org/10.3758/s13428-011-0118-4}{10.3758/s13428-011-0118-4}

\hypertarget{ref-Kounios2009}{}
Kounios, J., Green, D. L., Payne, L., Fleck, J. I., Grondin, R., \&
McRae, K. (2009). Semantic richness and the activation of concepts in
semantic memory: Evidence from event-related potentials. \emph{Brain
Research}, \emph{1282}, 95--102.
doi:\href{https://doi.org/10.1016/j.brainres.2009.05.092}{10.1016/j.brainres.2009.05.092}

\hypertarget{ref-Kremer2011a}{}
Kremer, G., \& Baroni, M. (2011). A set of semantic norms for German and
Italian. \emph{Behavior Research Methods}, \emph{43}(1), 97--109.
doi:\href{https://doi.org/10.3758/s13428-010-0028-x}{10.3758/s13428-010-0028-x}

\hypertarget{ref-Kuperman2012}{}
Kuperman, V., Stadthagen-Gonzalez, H., \& Brysbaert, M. (2012).
Age-of-acquisition ratings for 30,000 English words. \emph{Behavior
Research Methods}, \emph{44}(4), 978--990.
doi:\href{https://doi.org/10.3758/s13428-012-0210-4}{10.3758/s13428-012-0210-4}

\hypertarget{ref-Landauer1997}{}
Landauer, T. K., \& Dumais, S. T. (1997). A solution to Plato's problem:
The latent semantic analysis theory of acquisition, induction, and
representation of knowledge. \emph{Psychological Review}, \emph{104}(2),
211--240.
doi:\href{https://doi.org/10.1037//0033-295X.104.2.211}{10.1037//0033-295X.104.2.211}

\hypertarget{ref-Lenci2013}{}
Lenci, A., Baroni, M., Cazzolli, G., \& Marotta, G. (2013). BLIND: A set
of semantic feature norms from the congenitally blind. \emph{Behavior
Research Methods}, \emph{45}(4), 1218--1233.
doi:\href{https://doi.org/10.3758/s13428-013-0323-4}{10.3758/s13428-013-0323-4}

\hypertarget{ref-Lund1996}{}
Lund, K., \& Burgess, C. (1996). Hyperspace analogue to language (HAL):
A general model semantic representation. \emph{Brain and Cognition},
\emph{30}(3), 5--5.

\hypertarget{ref-MacKay1978}{}
MacKay, D. G. (1978). Derivational rules and the internal lexicon.
\emph{Journal of Verbal Learning and Verbal Behavior}, \emph{17},
61--71.

\hypertarget{ref-Maki2008}{}
Maki, W. S., \& Buchanan, E. M. (2008). Latent structure in measures of
associative, semantic, and thematic knowledge. \emph{Psychonomic
Bulletin \& Review}, \emph{15}(3), 598--603.
doi:\href{https://doi.org/10.3758/PBR.15.3.598}{10.3758/PBR.15.3.598}

\hypertarget{ref-Maki2004}{}
Maki, W. S., McKinley, L. N., \& Thompson, A. G. (2004). Semantic
distance norms computed from an electronic dictionary (WordNet).
\emph{Behavior Research Methods, Instruments, \& Computers},
\emph{36}(3), 421--431.
doi:\href{https://doi.org/10.3758/BF03195590}{10.3758/BF03195590}

\hypertarget{ref-Marslen-Wilson1994}{}
Marslen-Wilson, W., Tyler, L. K., Waksler, R., \& Older, L. (1994).
Morphology and meaning in the English mental lexicon.
\emph{Psychological Review}, \emph{101}(1), 3--33.
doi:\href{https://doi.org/10.1037/0033-295X.101.1.3}{10.1037/0033-295X.101.1.3}

\hypertarget{ref-McRae2013}{}
McRae, K., \& Jones, M. (2013). Semantic Memory. In D. Reisberg (Ed.),.
Oxford University Press.
doi:\href{https://doi.org/10.1093/oxfordhb/9780195376746.013.0014}{10.1093/oxfordhb/9780195376746.013.0014}

\hypertarget{ref-McRae2005}{}
McRae, K., Cree, G. S., Seidenberg, M. S., \& McNorgan, C. (2005).
Semantic feature production norms for a large set of living and
nonliving things. \emph{Behavior Research Methods}, \emph{37}(4),
547--559.
doi:\href{https://doi.org/10.3758/BF03192726}{10.3758/BF03192726}

\hypertarget{ref-McRae1997}{}
McRae, K., Sa, V. R. de, \& Seidenberg, M. S. (1997). On the nature and
scope of featural representations of word meaning. \emph{Journal of
Experimental Psychology: General}, \emph{126}(2), 99--130.
doi:\href{https://doi.org/10.1037/0096-3445.126.2.99}{10.1037/0096-3445.126.2.99}

\hypertarget{ref-Medin1989}{}
Medin, D. L. (1989). Concepts and conceptual structure. \emph{American
Psychologist}, \emph{44}(12), 1469--1481.
doi:\href{https://doi.org/10.1037/0003-066X.44.12.1469}{10.1037/0003-066X.44.12.1469}

\hypertarget{ref-Montefinese2013}{}
Montefinese, M., Ambrosini, E., Fairfield, B., \& Mammarella, N. (2013).
Semantic memory: A feature-based analysis and new norms for Italian.
\emph{Behavior Research Methods}, \emph{45}(2), 440--461.
doi:\href{https://doi.org/10.3758/s13428-012-0263-4}{10.3758/s13428-012-0263-4}

\hypertarget{ref-Montefinese2015}{}
Montefinese, M., Zannino, G. D., \& Ambrosini, E. (2015). Semantic
similarity between old and new items produces false alarms in
recognition memory. \emph{Psychological Research}, \emph{79}(5),
785--794.
doi:\href{https://doi.org/10.1007/s00426-014-0615-z}{10.1007/s00426-014-0615-z}

\hypertarget{ref-Moss1995}{}
Moss, H. E. H., Ostrin, R. K. R., Tyler, I., Marlsen-Wilson, W., Tyler,
L. K., \& Marslen-Wilson, W. D. (1995). Accessing different types of
lexical semantic information: Evidence from priming. \emph{Journal of
Experimental Psychology: Learning, Memory, and Cognition}, \emph{21}(4),
863--883.
doi:\href{https://doi.org/10.1037/\%200278-7393.21.4.863}{10.1037/ 0278-7393.21.4.863}

\hypertarget{ref-Moss2002}{}
Moss, H. E., Tyler, L. K., \& Devlin, J. T. (2002). The emergence of
category-specific deficits in a distribuited semantic system. In E.
Forde \& G. Humphreys (Eds.), \emph{Category-specificity in mind and
brain} (pp. 115--145). CRC Press.

\hypertarget{ref-Nelson2004}{}
Nelson, D. L., McEvoy, C. L., \& Schreiber, T. A. (2004). The University
of South Florida free association, rhyme, and word fragment norms.
\emph{Behavior Research Methods, Instruments, \& Computers},
\emph{36}(3), 402--407.
doi:\href{https://doi.org/10.3758/BF03195588}{10.3758/BF03195588}

\hypertarget{ref-New2007}{}
New, B., Brysbaert, M., Veronis, J., \& Pallier, C. (2007). The use of
film subtitles to estimate word frequencies. \emph{Applied
Psycholinguistics}, \emph{28}(4), 661--677.
doi:\href{https://doi.org/10.1017/S014271640707035X}{10.1017/S014271640707035X}

\hypertarget{ref-Pexman2007}{}
Pexman, P. M., Hargreaves, I. S., Edwards, J. D., Henry, L. C., \&
Goodyear, B. G. (2007). The Neural Consequences of Semantic Richness
When More Comes to Mind , Less Activation Is Observed.
\emph{Psychological Science}, \emph{18}(5), 401--406.

\hypertarget{ref-Pexman2008}{}
Pexman, P. M., Hargreaves, I. S., Siakaluk, P. D., Bodner, G. E., \&
Pope, J. (2008). There are many ways to be rich : Effects of three
measures of semantic, \emph{15}(1), 161--167.
doi:\href{https://doi.org/10.3758/PBR.15.1.161}{10.3758/PBR.15.1.161}

\hypertarget{ref-Pexman2003}{}
Pexman, P. M., Holyk, G. G., \& Monfils, M.-H. (2003).
Number-of-features effects and semantic processing. \emph{Memory \&
Cognition}, \emph{31}(6), 842--855.
doi:\href{https://doi.org/10.3758/BF03196439}{10.3758/BF03196439}

\hypertarget{ref-Porter2001}{}
Porter, M. (2001). Snowball: A language for stemming algorithms -
Snowball. Retrieved from
\url{https://snowballstem.org/texts/introduction.html}

\hypertarget{ref-R-base}{}
R Core Team. (2017). \emph{R: A language and environment for statistical
computing}. Vienna, Austria: R Foundation for Statistical Computing.
Retrieved from \url{https://www.R-project.org/}

\hypertarget{ref-Reverberi2004}{}
Reverberi, C., Capitani, E., \& Laiacona, E. (2004). Variabili semantico
lessicali relative a tutti gli elementi di una categoria semantica:
Indagine su soggetti normali italiani per la categoria ``frutta".
\emph{Giornale Italiano Di Psicologia}, \emph{31}, 497--522.

\hypertarget{ref-Riordan2011}{}
Riordan, B., \& Jones, M. N. (2011). Redundancy in perceptual and
linguistic experience: Comparing feature-based and distributional models
of semantic representation. \emph{Topics in Cognitive Science},
\emph{3}(2), 303--345.
doi:\href{https://doi.org/10.1111/j.1756-8765.2010.01111.x}{10.1111/j.1756-8765.2010.01111.x}

\hypertarget{ref-Rogers2004}{}
Rogers, T. T., \& McClelland, J. L. (2004). \emph{Semantic cognition: A
parallel distributed processing approach}. MIT Press.

\hypertarget{ref-Rosch1975}{}
Rosch, E., \& Mervis, C. B. (1975). Family resemblances: Studies in the
internal structure of categories. \emph{Cognitive Psychology},
\emph{7}(4), 573--605.
doi:\href{https://doi.org/10.1016/0010-0285(75)90024-9}{10.1016/0010-0285(75)90024-9}

\hypertarget{ref-Ruts2004}{}
Ruts, W., De Deyne, S., Ameel, E., Vanpaemel, W., Verbeemen, T., \&
Storms, G. (2004). Dutch norm data for 13 semantic categories and 338
exemplars. \emph{Behavior Research Methods, Instruments, \& Computers},
\emph{36}(3), 506--515.
doi:\href{https://doi.org/10.3758/BF03195597}{10.3758/BF03195597}

\hypertarget{ref-Smith1974}{}
Smith, E. E., Shoben, E. J., \& Rips, L. J. (1974). Structure and
process in semantic memory: A featural model for semantic decisions.
\emph{Psychological Review}, \emph{81}(3), 214--241.
doi:\href{https://doi.org/10.1037/h0036351}{10.1037/h0036351}

\hypertarget{ref-Stein2009}{}
Stein, L., \& de Azevedo Gomes, C. (2009). Normas Brasileiras para
listas de palavras associadas: Associação semântica, concretude,
frequência e emocionalidade. \emph{Psicologia: Teoria E Pesquisa},
\emph{25}, 537--546.
doi:\href{https://doi.org/10.1590/S0102-37722009000400009}{10.1590/S0102-37722009000400009}

\hypertarget{ref-Toglia2009}{}
Toglia, M. P. (2009). Withstanding the test of time: The 1978 semantic
word norms. \emph{Behavior Research Methods}, \emph{41}(2), 531--533.
doi:\href{https://doi.org/10.3758/BRM.41.2.531}{10.3758/BRM.41.2.531}

\hypertarget{ref-Toglia1978}{}
Toglia, M. P., \& Battig, W. F. (1978). \emph{Handbook of semantic word
norms}. Hillside, NJ: Earlbaum.

\hypertarget{ref-Vieth2014}{}
Vieth, H. E., Mcmahon, K. L., \& Zubicaray, G. I. D. (2014). The roles
of shared vs . distinctive conceptual features in lexical access,
\emph{5}(September), 1--12.
doi:\href{https://doi.org/10.3389/fpsyg.2014.01014}{10.3389/fpsyg.2014.01014}

\hypertarget{ref-Vigliocco2005}{}
Vigliocco, G., Vinson, D. P., \& Siri, S. (2005). Semantic and
grammatical class effects in naming actions. \emph{Cognition},
\emph{94}, 91--100.
doi:\href{https://doi.org/10.1016/j.cognition.2004.06.004}{10.1016/j.cognition.2004.06.004}

\hypertarget{ref-Vigliocco2002}{}
Vigliocco, G., Vinson, D. P., Damian, M. M. F., \& Levelt, W. (2002).
Semantic distance effects on object and action naming. \emph{Cognition},
\emph{85}, 61--69.
doi:\href{https://doi.org/10.1016/S0010-0277(02)00107-5}{10.1016/S0010-0277(02)00107-5}

\hypertarget{ref-Vigliocco2004}{}
Vigliocco, G., Vinson, D. P., Lewis, W., \& Garrett, M. F. (2004).
Representing the meanings of object and action words: The featural and
unitary semantic space hypothesis. \emph{Cognitive Psychology},
\emph{48}(4), 422--488.
doi:\href{https://doi.org/10.1016/j.cogpsych.2003.09.001}{10.1016/j.cogpsych.2003.09.001}

\hypertarget{ref-Vinson2002}{}
Vinson, D. P., \& Vigliocco, G. (2002). A semantic analysis of noun-verb
dissociations in aphasia. \emph{Journal of Neurolinguistics}, \emph{15},
317--351.
doi:\href{https://doi.org/10.1016/S0911-6044(01)00037-9}{10.1016/S0911-6044(01)00037-9}

\hypertarget{ref-Vinson2008}{}
Vinson, D. P., \& Vigliocco, G. (2008). Semantic feature production
norms for a large set of objects and events. \emph{Behavior Research
Methods}, \emph{40}(1), 183--190.
doi:\href{https://doi.org/10.3758/BRM.40.1.183}{10.3758/BRM.40.1.183}

\hypertarget{ref-Vinson2003}{}
Vinson, D. P., Vigliocco, G., Cappa, S., \& Siri, S. (2003). The
breakdown of semantic knowledge: Insights from a statistical model of
meaning representation. \emph{Brain and Language}, \emph{86}(3),
347--365.
doi:\href{https://doi.org/10.1016/S0093-934X(03)00144-5}{10.1016/S0093-934X(03)00144-5}

\hypertarget{ref-Vivas2017}{}
Vivas, J., Vivas, L., Comesaña, A., Coni, A. G., \& Vorano, A. (2017).
Spanish semantic feature production norms for 400 concrete concepts.
\emph{Behavior Research Methods}, \emph{49}(3), 1095--1106.
doi:\href{https://doi.org/10.3758/s13428-016-0777-2}{10.3758/s13428-016-0777-2}

\hypertarget{ref-Warriner2013}{}
Warriner, A. B., Kuperman, V., \& Brysbaert, M. (2013). Norms of
valence, arousal, and dominance for 13,915 English lemmas.
\emph{Behavior Research Methods}, \emph{45}(4), 1191--1207.
doi:\href{https://doi.org/10.3758/s13428-012-0314-x}{10.3758/s13428-012-0314-x}

\hypertarget{ref-Yap2016}{}
Yap, M. J., \& Pexman, P. M. (2016). Semantic richness effects in
syntactic classification: The role of feedback. \emph{Frontiers in
Psychology}, \emph{7}(July), 1394.
doi:\href{https://doi.org/10.3389/fpsyg.2016.01394}{10.3389/fpsyg.2016.01394}

\hypertarget{ref-Yap2015}{}
Yap, M. J., Lim, G. Y., \& Pexman, P. M. (2015). Semantic richness
effects in lexical decision: The role of feedback. \emph{Memory \&
Cognition}, \emph{43}(8), 1148--1167.
doi:\href{https://doi.org/10.3758/s13421-015-0536-0}{10.3758/s13421-015-0536-0}

\hypertarget{ref-Yap2012}{}
Yap, M. J., Pexman, P. M., Wellsby, M., Hargreaves, I. S., \& Huff, M.
J. (2012). An abundance of riches : cross-task comparisons of semantic
richness effects in visual word recognition. \emph{Frontiers in Human
Neuroscience}, \emph{6}, 1--10.
doi:\href{https://doi.org/10.3389/fnhum.2012.00072}{10.3389/fnhum.2012.00072}

\hypertarget{ref-Yap2011}{}
Yap, M. J., Tan, S. E., Pexman, P. M., \& Hargreaves, I. S. (2011). Is
more always better? Effects of semantic richness on lexical decision,
speeded pronunciation, and semantic classification. \emph{Psychonomic
Bulletin and Review}, \emph{18}(4), 742--750.
doi:\href{https://doi.org/10.3758/s13423-011-0092-y}{10.3758/s13423-011-0092-y}


\end{document}
